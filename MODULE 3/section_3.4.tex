\subsection{Properties of the Cross Product}

Let $\mathbf{a},\mathbf{b},\mathbf{c}\in\mathbb{R}^3$ and $k$ be a scalar.

\subsubsection{Linearity and Distributive Laws}

\begin{conceptbox}
    Cross product satisfies:
    \begin{enumerate}[label=(\alph*)]
        \item \textbf{Scalar Linearity:}
              \[
                  (k\mathbf{a})\times\mathbf{b} = k(\mathbf{a}\times\mathbf{b}) = \mathbf{a}\times(k\mathbf{b})
              \]
        \item \textbf{Distributive Property:}
              \[
                  \mathbf{a}\times(\mathbf{b}+\mathbf{c}) = \mathbf{a}\times\mathbf{b} + \mathbf{a}\times\mathbf{c}
              \]
              \[
                  (\mathbf{a}+\mathbf{b})\times\mathbf{c} = \mathbf{a}\times\mathbf{c} + \mathbf{b}\times\mathbf{c}
              \]
    \end{enumerate}
\end{conceptbox}

\subsubsection{Anti-Commutativity}

\begin{conceptbox}
    The cross product is \textbf{anti-commutative}:
    \[
        \mathbf{a}\times\mathbf{b}=-(\mathbf{b}\times\mathbf{a}).
    \]
\end{conceptbox}

\subsubsection{Not Associative}

\begin{noteBox}
    Cross product is \textbf{not associative}. In general:
    \[
        \mathbf{a}\times(\mathbf{b}\times\mathbf{c}) \neq (\mathbf{a}\times\mathbf{b})\times\mathbf{c}.
    \]
    So parentheses matter.
\end{noteBox}

\begin{examplebox}
    Verify anti-commutativity using an easy pair:
    \[
        \mathbf{a}=\mathbf{i}=[1,0,0],\qquad \mathbf{b}=\mathbf{j}=[0,1,0].
    \]

    We know:
    \[
        \mathbf{i}\times\mathbf{j}=\mathbf{k}.
    \]
    Switching the order:
    \[
        \mathbf{j}\times\mathbf{i}=-\mathbf{k}.
    \]
    So,
    \[
        \mathbf{i}\times\mathbf{j}=-(\mathbf{j}\times\mathbf{i}).
    \]
\end{examplebox}

\begin{examplebox}
    Demonstrate non-associativity using basis vectors:

    \[
        \mathbf{i}\times(\mathbf{j}\times\mathbf{k})
        =
        \mathbf{i}\times(\mathbf{i})
        =
        \mathbf{0}.
    \]

    But
    \[
        (\mathbf{i}\times\mathbf{j})\times\mathbf{k}
        =
        \mathbf{k}\times\mathbf{k}
        =
        \mathbf{0}.
    \]

    This example gives equality, so it doesn't disprove associativity.

    To show non-associativity more clearly, use:
    \[
        \mathbf{a}=[1,1,0],\quad \mathbf{b}=[1,0,1],\quad \mathbf{c}=[0,1,1].
    \]

    Compute $\mathbf{b}\times\mathbf{c}$:
    \[
        \mathbf{b}\times\mathbf{c}
        =
        \big[0\cdot 1-1\cdot 1,\; 1\cdot 0-1\cdot 1,\; 1\cdot 1-0\cdot 0\big]
        =
        [-1,-1,1].
    \]
    Then
    \[
        \mathbf{a}\times(\mathbf{b}\times\mathbf{c})
        =
        [1,1,0]\times[-1,-1,1]
        =
        [1(-?)-0(-?),\;\dots]
        =
        [1,-1,0].
    \]

    Now compute $\mathbf{a}\times\mathbf{b}$:
    \[
        \mathbf{a}\times\mathbf{b}
        =
        [1,1,0]\times[1,0,1]
        =
        [1\cdot 1-0\cdot 0,\;0\cdot 1-1\cdot 1,\;1\cdot 0-1\cdot 1]
        =
        [1,-1,-1].
    \]
    Then
    \[
        (\mathbf{a}\times\mathbf{b})\times\mathbf{c}
        =
        [1,-1,-1]\times[0,1,1]
        =
        [0,\,-1,\,1].
    \]

    Since
    \[
        \mathbf{a}\times(\mathbf{b}\times\mathbf{c})=[1,-1,0]
        \neq
        (\mathbf{a}\times\mathbf{b})\times\mathbf{c}=[0,-1,1],
    \]
    cross product is not associative.
\end{examplebox}
