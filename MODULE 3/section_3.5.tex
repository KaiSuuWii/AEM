\subsection{Scalar Triple Product and Volume Interpretation}

\subsubsection{Definition}

\begin{conceptbox}
    The \textbf{scalar triple product} of $\mathbf{a},\mathbf{b},\mathbf{c}\in\mathbb{R}^3$ is defined as
    \[
        (\mathbf{a},\mathbf{b},\mathbf{c})=\mathbf{a}\cdot(\mathbf{b}\times\mathbf{c}).
    \]
    It produces a \textbf{scalar} (a real number).
\end{conceptbox}

\subsubsection{Determinant Form}

If
\[
    \mathbf{a}=[a_1,a_2,a_3],\quad
    \mathbf{b}=[b_1,b_2,b_3],\quad
    \mathbf{c}=[c_1,c_2,c_3],
\]
then
\[
    \mathbf{a}\cdot(\mathbf{b}\times\mathbf{c})
    =
    \begin{vmatrix}
        a_1 & a_2 & a_3 \\
        b_1 & b_2 & b_3 \\
        c_1 & c_2 & c_3
    \end{vmatrix}.
\]

\subsubsection{Geometric Meaning: Volume}

\begin{noteBox}
    The absolute value of the scalar triple product gives the \textbf{volume of the parallelepiped}
    formed by $\mathbf{a},\mathbf{b},\mathbf{c}$:
    \[
        V = \left|\mathbf{a}\cdot(\mathbf{b}\times\mathbf{c})\right|.
    \]
\end{noteBox}

If the volume is zero, the vectors lie in the same plane (they are coplanar), meaning they are linearly dependent in $\mathbb{R}^3$.

\begin{examplebox}
    Let
    \[
        \mathbf{a}=[1,2,3],\quad
        \mathbf{b}=[2,0,1],\quad
        \mathbf{c}=[1,1,0].
    \]

    Step 1: Compute $\mathbf{b}\times\mathbf{c}$:
    \[
        \mathbf{b}\times\mathbf{c}
        =
        [\,0\cdot 0-1\cdot 1,\; 1\cdot 1-2\cdot 0,\; 2\cdot 1-0\cdot 1\,]
        =
        [-1,1,2].
    \]

    Step 2: Dot with $\mathbf{a}$:
    \[
        \mathbf{a}\cdot(\mathbf{b}\times\mathbf{c})
        =
        [1,2,3]\cdot[-1,1,2]
        =
        1(-1)+2(1)+3(2)
        =
        -1+2+6
        =
        7.
    \]

    Thus, the volume of the parallelepiped is
    \[
        V=|7|=7 \text{ cubic units}.
    \]
\end{examplebox}

\begin{examplebox}
    If $\mathbf{a}=[1,2,3]$, $\mathbf{b}=[2,4,6]$, and $\mathbf{c}=[0,1,0]$,
    notice that $\mathbf{b}=2\mathbf{a}$, so $\mathbf{a}$ and $\mathbf{b}$ are parallel.

    Then $\mathbf{a},\mathbf{b},\mathbf{c}$ are coplanar, and the parallelepiped collapses into a flat shape.
    Hence,
    \[
        \mathbf{a}\cdot(\mathbf{b}\times\mathbf{c})=0
        \quad\Rightarrow\quad
        V=0.
    \]
\end{examplebox}
