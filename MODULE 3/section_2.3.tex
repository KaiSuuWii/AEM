\subsection{Angle and Orthogonality}

\subsubsection{Angle Between Two Vectors}

If $\mathbf{a}\ne\mathbf{0}$ and $\mathbf{b}\ne\mathbf{0}$, then
\[
    \cos\gamma=\frac{\mathbf{a}\cdot\mathbf{b}}{\|\mathbf{a}\|\|\mathbf{b}\|},
    \qquad
    \gamma=\arccos\left(\frac{\mathbf{a}\cdot\mathbf{b}}{\|\mathbf{a}\|\|\mathbf{b}\|}\right).
\]

\begin{examplebox}
    Let $\mathbf{a}=[1,2,0]$ and $\mathbf{b}=[3,-2,1]$.

    From earlier, $\mathbf{a}\cdot\mathbf{b}=-1$.
    Also,
    \[
        \|\mathbf{a}\|=\sqrt{1^2+2^2+0^2}=\sqrt{5},\qquad
        \|\mathbf{b}\|=\sqrt{3^2+(-2)^2+1^2}=\sqrt{14}.
    \]
    Thus,
    \[
        \cos\gamma=\frac{-1}{\sqrt{5}\sqrt{14}}=\frac{-1}{\sqrt{70}}.
    \]
\end{examplebox}

\subsubsection{Orthogonality}

\begin{conceptbox}
    Two vectors $\mathbf{a}$ and $\mathbf{b}$ are \textbf{orthogonal} (perpendicular) if and only if
    \[
        \mathbf{a}\cdot\mathbf{b}=0.
    \]
\end{conceptbox}

\begin{examplebox}
    Check if $\mathbf{u}=[2,-1]$ and $\mathbf{v}=[1,2]$ are orthogonal:
    \[
        \mathbf{u}\cdot\mathbf{v}=2(1)+(-1)(2)=2-2=0.
    \]
    Hence, $\mathbf{u}\perp\mathbf{v}$.
\end{examplebox}
