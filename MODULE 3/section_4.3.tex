\subsection{Parallelism and Perpendicularity}

Understanding whether lines and planes are parallel or perpendicular is essential in
engineering applications such as design, architecture, robotics, and machine alignment.

\subsubsection{Parallel Lines}

\begin{conceptbox}
    Two lines are parallel if their direction vectors are scalar multiples of each other.
\end{conceptbox}

If $\mathbf{v}_1$ and $\mathbf{v}_2$ are direction vectors:
\[
    \mathbf{v}_1 = k\mathbf{v}_2
\]
for some scalar $k$, then the lines are parallel.

\begin{examplebox}
    Check if lines with direction vectors
    \[
        \mathbf{v}_1=[2,4,-1],\qquad \mathbf{v}_2=[1,2,-0.5]
    \]
    are parallel.

    Observe:
    \[
        \mathbf{v}_2=\left[\frac{2}{2},\frac{4}{2},\frac{-1}{2}\right]
        =[1,2,-0.5].
    \]

    Thus,
    \[
        \mathbf{v}_1=2\mathbf{v}_2.
    \]
    So the lines are parallel.
\end{examplebox}

\subsubsection{Perpendicular Lines}

\begin{conceptbox}
    Two lines are perpendicular if their direction vectors satisfy:
    \[
        \mathbf{v}_1\cdot\mathbf{v}_2=0.
    \]
\end{conceptbox}

\begin{examplebox}
    Check if the vectors
    \[
        \mathbf{v}_1=[3,-2,1],\qquad \mathbf{v}_2=[2,3,0]
    \]
    are perpendicular.

    \[
        \mathbf{v}_1\cdot\mathbf{v}_2=3(2)+(-2)(3)+1(0)=6-6+0=0.
    \]

    Thus, the lines are perpendicular.
\end{examplebox}

\subsubsection{Parallel Planes}

\begin{conceptbox}
    Two planes are parallel if their normal vectors are scalar multiples.
\end{conceptbox}

For planes:
\[
    A_1x+B_1y+C_1z=D_1
\]
\[
    A_2x+B_2y+C_2z=D_2
\]

they are parallel if:
\[
    [A_1,B_1,C_1]=k[A_2,B_2,C_2].
\]

\subsubsection{Perpendicular Planes}

\begin{conceptbox}
    Two planes are perpendicular if their normal vectors are orthogonal:
    \[
        \mathbf{n}_1\cdot\mathbf{n}_2=0.
    \]
\end{conceptbox}

\begin{examplebox}
    Determine whether the planes
    \[
        2x-y+3z=5
    \]
    and
    \[
        x+2y-z=4
    \]
    are perpendicular.

    Normal vectors:
    \[
        \mathbf{n}_1=[2,-1,3],\qquad \mathbf{n}_2=[1,2,-1].
    \]

    Dot product:
    \[
        \mathbf{n}_1\cdot\mathbf{n}_2=2(1)+(-1)(2)+3(-1)=2-2-3=-3.
    \]

    Since the dot product is not zero, the planes are not perpendicular.
\end{examplebox}
