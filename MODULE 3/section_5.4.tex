\subsection{Linear Independence}

Linear independence is a key concept because it tells us whether vectors provide new information
or if one is redundant.

\begin{conceptbox}
    Vectors $\mathbf{v}_1,\mathbf{v}_2,\dots,\mathbf{v}_k$ are \textbf{linearly independent} if the only solution to:
    \[
        c_1\mathbf{v}_1+c_2\mathbf{v}_2+\dots+c_k\mathbf{v}_k=\mathbf{0}
    \]
    is
    \[
        c_1=c_2=\dots=c_k=0.
    \]
\end{conceptbox}

If there exists a nonzero solution, then the vectors are \textbf{linearly dependent}.

\begin{examplebox}
    Determine if the vectors
    \[
        \mathbf{v}_1=[1,2],\qquad \mathbf{v}_2=[2,4]
    \]
    are linearly independent.

    We check if:
    \[
        c_1[1,2]+c_2[2,4]=[0,0].
    \]

    This gives the system:
    \[
        c_1+2c_2=0
    \]
    \[
        2c_1+4c_2=0.
    \]

    The second equation is just twice the first, so there are infinitely many solutions.
    For example, if $c_2=1$, then $c_1=-2$.

    Thus, there is a nontrivial solution, so the vectors are linearly dependent.
\end{examplebox}

\begin{examplebox}
    Determine if the vectors
    \[
        \mathbf{u}=[1,0,2],\qquad \mathbf{v}=[0,1,3],\qquad \mathbf{w}=[1,1,0]
    \]
    are linearly independent.

    We solve:
    \[
        c_1\mathbf{u}+c_2\mathbf{v}+c_3\mathbf{w}=\mathbf{0}.
    \]

    That is:
    \[
        c_1[1,0,2]+c_2[0,1,3]+c_3[1,1,0]=[0,0,0].
    \]

    Component-wise:
    \[
        c_1+c_3=0
    \]
    \[
        c_2+c_3=0
    \]
    \[
        2c_1+3c_2=0.
    \]

    From the first two equations:
    \[
        c_1=-c_3,\qquad c_2=-c_3.
    \]

    Substitute into the third:
    \[
        2(-c_3)+3(-c_3)=0
        \Rightarrow -5c_3=0
        \Rightarrow c_3=0.
    \]

    Then $c_1=0$ and $c_2=0$.

    Thus the only solution is the trivial solution, so the vectors are linearly independent.
\end{examplebox}
