\subsection{Vector Applications}

Vectors appear naturally in engineering and physics because many real-world quantities require both magnitude and direction.

\subsubsection{Displacement}

A displacement from point $P(x_1,y_1,z_1)$ to point $Q(x_2,y_2,z_2)$ is represented as:
\[
    \mathbf{d} = [x_2-x_1,\; y_2-y_1,\; z_2-z_1].
\]

\subsubsection{Velocity}

Velocity is a vector that describes both speed and direction of motion.

\subsubsection{Force and Resultant Force}

\begin{noteBox}
    In statics and dynamics, forces are vectors. The \textbf{resultant force} is obtained by adding all forces acting on a body:
    \[
        \mathbf{R} = \mathbf{F}_1 + \mathbf{F}_2 + \dots + \mathbf{F}_n.
    \]
    If $\mathbf{R} = \mathbf{0}$, then the system is in equilibrium.
\end{noteBox}

\begin{examplebox}
    Suppose two forces act on an object:
    \[
        \mathbf{F}_1 = [3,2], \qquad \mathbf{F}_2 = [1,-4].
    \]

    Then the resultant force is:
    \[
        \mathbf{R} = \mathbf{F}_1 + \mathbf{F}_2 = [3+1,\;2-4] = [4,-2].
    \]

    The magnitude of the resultant is:
    \[
        \|\mathbf{R}\| = \sqrt{4^2 + (-2)^2} = \sqrt{20}.
    \]
\end{examplebox}