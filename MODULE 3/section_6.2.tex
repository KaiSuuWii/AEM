\subsection{Projection Review}

Before applying Gram-Schmidt, we must recall the concept of projection.

\subsubsection{Vector Projection Formula}

\begin{conceptbox}
    The projection of $\mathbf{v}$ onto $\mathbf{u}$ is given by:
    \[
        \mathrm{proj}_{\mathbf{u}}(\mathbf{v})
        =
        \left(\frac{\mathbf{v}\cdot\mathbf{u}}{\mathbf{u}\cdot\mathbf{u}}\right)\mathbf{u}.
    \]
\end{conceptbox}

This formula gives the component of $\mathbf{v}$ that lies in the direction of $\mathbf{u}$.

\subsubsection{Geometric Interpretation}

\begin{noteBox}
    Projection is essentially ``shadowing'' a vector onto another direction.
    In engineering, projection is used in force decomposition, signal approximation, and coordinate transformations.
\end{noteBox}

\begin{examplebox}
    Let
    \[
        \mathbf{v}=[3,4],\qquad \mathbf{u}=[1,2].
    \]

    Compute the projection of $\mathbf{v}$ onto $\mathbf{u}$.

    First compute the dot products:
    \[
        \mathbf{v}\cdot\mathbf{u}=3(1)+4(2)=11
    \]
    \[
        \mathbf{u}\cdot\mathbf{u}=1^2+2^2=5.
    \]

    Thus,
    \[
        \mathrm{proj}_{\mathbf{u}}(\mathbf{v})
        =
        \left(\frac{11}{5}\right)[1,2]
        =
        \left[\frac{11}{5},\frac{22}{5}\right].
    \]

    So the part of $\mathbf{v}$ in the direction of $\mathbf{u}$ is:
    \[
        \mathrm{proj}_{\mathbf{u}}(\mathbf{v})=\left[\frac{11}{5},\frac{22}{5}\right].
    \]
\end{examplebox}
