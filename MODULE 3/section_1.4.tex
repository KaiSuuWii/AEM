
\subsection{Standard Basis and Component Form}

\subsubsection{Standard Basis Vectors}

\begin{conceptbox}
    The \textbf{standard basis vectors} in $\mathbb{R}^n$ are the vectors:
    \[
        \mathbf{e}_1 = [1,0,0,\dots,0],\quad
        \mathbf{e}_2 = [0,1,0,\dots,0],\quad \dots,\quad
        \mathbf{e}_n = [0,0,0,\dots,1].
    \]
\end{conceptbox}

Any vector $\mathbf{v} \in \mathbb{R}^n$ can be written as a linear combination of these basis vectors:
\[
    \mathbf{v} = v_1\mathbf{e}_1 + v_2\mathbf{e}_2 + \dots + v_n\mathbf{e}_n.
\]

\subsubsection{Component Form in $\mathbb{R}^3$}

In $\mathbb{R}^3$, the standard basis is often written as:
\[
    \mathbf{i} = [1,0,0],\quad
    \mathbf{j} = [0,1,0],\quad
    \mathbf{k} = [0,0,1].
\]

Thus any vector can be written as:
\[
    \mathbf{v} = v_1\mathbf{i} + v_2\mathbf{j} + v_3\mathbf{k}.
\]

\begin{examplebox}
    Let the vector be
    \[
        \mathbf{a}=[5,-3,2].
    \]

    Using the standard basis vectors
    \[
        \mathbf{i}=[1,0,0],\quad
        \mathbf{j}=[0,1,0],\quad
        \mathbf{k}=[0,0,1],
    \]
    we can rewrite $\mathbf{a}$ in component form as:
    \[
        \mathbf{a}=5\mathbf{i}-3\mathbf{j}+2\mathbf{k}.
    \]

    This representation is useful in physics and engineering because each term represents
    a vector component along an axis direction.
\end{examplebox}
