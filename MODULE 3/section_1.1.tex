\section{Vectors in \texorpdfstring{$\mathbb{R}^n$}{R^n}}

\subsection{Definition and Representation of Vectors}

\begin{conceptbox}
    A \textbf{vector} is a quantity that has both \textbf{magnitude} and \textbf{direction}.
    It is commonly represented as an ordered list of real numbers:
    \[
        \mathbf{v} = [v_1, v_2, v_3, \dots, v_n]
    \]
    where $\mathbf{v} \in \mathbb{R}^n$.
\end{conceptbox}

A \textbf{scalar} is a quantity described only by magnitude (a single number), such as temperature, mass, or time.

Vectors can be interpreted geometrically as directed line segments (arrows), and algebraically as ordered tuples.

\subsubsection{Position Vector}

\begin{conceptbox}
    The \textbf{position vector} of a point $P(x_1,x_2,\dots,x_n)$ is the vector drawn from the origin to the point:
    \[
        \mathbf{r} = [x_1, x_2, \dots, x_n]
    \]
\end{conceptbox}

\begin{examplebox}
    A point in 3D space is given by
    \[
        P(2,-1,5).
    \]

    The position vector of $P$ is the vector from the origin to $P$:
    \[
        \mathbf{r} = [2,-1,5].
    \]

    This means the vector moves:
    \begin{itemize}
        \item 2 units in the $x$-direction,
        \item 1 unit in the negative $y$-direction,
        \item 5 units in the $z$-direction.
    \end{itemize}
\end{examplebox}
