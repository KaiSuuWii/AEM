\subsection{Basis and Dimension}

\subsubsection{Basis}

\begin{conceptbox}
    A set of vectors $\{\mathbf{v}_1,\mathbf{v}_2,\dots,\mathbf{v}_k\}$ is a \textbf{basis} for a vector space $V$ if:
    \begin{enumerate}[label=(\alph*)]
        \item The vectors span $V$
        \item The vectors are linearly independent
    \end{enumerate}
\end{conceptbox}

This means every vector in $V$ can be written uniquely as a linear combination of the basis vectors.

\subsubsection{Dimension}

\begin{conceptbox}
    The \textbf{dimension} of a vector space is the number of vectors in any basis of the space.
\end{conceptbox}

For example:
\[
    \dim(\mathbb{R}^2)=2,\qquad \dim(\mathbb{R}^3)=3.
\]

\begin{examplebox}
    Show that the vectors
    \[
        \mathbf{v}_1=[1,0],\qquad \mathbf{v}_2=[0,1]
    \]
    form a basis for $\mathbb{R}^2$.

    \textbf{Step 1: Check spanning.}

    Any vector $[x,y]\in\mathbb{R}^2$ can be written as:
    \[
        [x,y]=x[1,0]+y[0,1].
    \]
    So they span $\mathbb{R}^2$.

    \textbf{Step 2: Check independence.}

    Solve:
    \[
        c_1[1,0]+c_2[0,1]=[0,0].
    \]

    This implies:
    \[
        c_1=0,\qquad c_2=0.
    \]

    So they are linearly independent.

    Thus $\{\mathbf{v}_1,\mathbf{v}_2\}$ is a basis for $\mathbb{R}^2$ and $\dim(\mathbb{R}^2)=2$.
\end{examplebox}

\begin{noteBox}
    A basis is like a coordinate system. In engineering, it is used in signal decomposition, Fourier series, state-space models, and machine learning feature spaces.
\end{noteBox}
