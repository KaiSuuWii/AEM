\subsection{Worked Example in $\mathbb{R}^3$}

\begin{examplebox}
    Apply Gram-Schmidt to the vectors
    \[
        \mathbf{v}_1=[1,1,0],\qquad
        \mathbf{v}_2=[1,0,1],\qquad
        \mathbf{v}_3=[0,1,1].
    \]

    \textbf{Step 1: Compute $\mathbf{u}_1$}
    \[
        \mathbf{u}_1=\mathbf{v}_1=[1,1,0].
    \]

    \textbf{Step 2: Compute $\mathbf{u}_2$}

    First compute:
    \[
        \mathrm{proj}_{\mathbf{u}_1}(\mathbf{v}_2)
        =
        \left(\frac{\mathbf{v}_2\cdot\mathbf{u}_1}{\mathbf{u}_1\cdot\mathbf{u}_1}\right)\mathbf{u}_1.
    \]

    Dot products:
    \[
        \mathbf{v}_2\cdot\mathbf{u}_1=[1,0,1]\cdot[1,1,0]=1(1)+0(1)+1(0)=1
    \]
    \[
        \mathbf{u}_1\cdot\mathbf{u}_1=[1,1,0]\cdot[1,1,0]=1+1+0=2.
    \]

    Thus:
    \[
        \mathrm{proj}_{\mathbf{u}_1}(\mathbf{v}_2)
        =
        \left(\frac{1}{2}\right)[1,1,0]
        =
        \left[\frac{1}{2},\frac{1}{2},0\right].
    \]

    Now subtract:
    \[
        \mathbf{u}_2=\mathbf{v}_2-\mathrm{proj}_{\mathbf{u}_1}(\mathbf{v}_2)
        =
        [1,0,1]-\left[\frac{1}{2},\frac{1}{2},0\right]
        =
        \left[\frac{1}{2},-\frac{1}{2},1\right].
    \]

    \textbf{Step 3: Compute $\mathbf{u}_3$}

    We compute:
    \[
        \mathbf{u}_3=\mathbf{v}_3-\mathrm{proj}_{\mathbf{u}_1}(\mathbf{v}_3)-\mathrm{proj}_{\mathbf{u}_2}(\mathbf{v}_3).
    \]

    First projection:
    \[
        \mathbf{v}_3\cdot\mathbf{u}_1=[0,1,1]\cdot[1,1,0]=0(1)+1(1)+1(0)=1
    \]
    \[
        \mathbf{u}_1\cdot\mathbf{u}_1=2
    \]
    \[
        \mathrm{proj}_{\mathbf{u}_1}(\mathbf{v}_3)
        =
        \left(\frac{1}{2}\right)[1,1,0]
        =
        \left[\frac{1}{2},\frac{1}{2},0\right].
    \]

    Second projection requires:
    \[
        \mathbf{v}_3\cdot\mathbf{u}_2=[0,1,1]\cdot\left[\frac{1}{2},-\frac{1}{2},1\right]
        =
        0\left(\frac{1}{2}\right)+1\left(-\frac{1}{2}\right)+1(1)
        =
        \frac{1}{2}.
    \]

    Now compute $\mathbf{u}_2\cdot\mathbf{u}_2$:
    \[
        \mathbf{u}_2\cdot\mathbf{u}_2
        =
        \left(\frac{1}{2}\right)^2+\left(-\frac{1}{2}\right)^2+(1)^2
        =
        \frac{1}{4}+\frac{1}{4}+1
        =
        \frac{3}{2}.
    \]

    Thus:
    \[
        \mathrm{proj}_{\mathbf{u}_2}(\mathbf{v}_3)
        =
        \left(\frac{\frac{1}{2}}{\frac{3}{2}}\right)\mathbf{u}_2
        =
        \left(\frac{1}{3}\right)\left[\frac{1}{2},-\frac{1}{2},1\right]
        =
        \left[\frac{1}{6},-\frac{1}{6},\frac{1}{3}\right].
    \]

    Now compute $\mathbf{u}_3$:
    \[
        \mathbf{u}_3
        =
        [0,1,1]
        -
        \left[\frac{1}{2},\frac{1}{2},0\right]
        -
        \left[\frac{1}{6},-\frac{1}{6},\frac{1}{3}\right].
    \]

    Combine the subtractions:
    \[
        \mathbf{u}_3
        =
        \left[
            0-\frac{1}{2}-\frac{1}{6},\;
            1-\frac{1}{2}+\frac{1}{6},\;
            1-0-\frac{1}{3}
            \right].
    \]

    \[
        \mathbf{u}_3
        =
        \left[
            -\frac{2}{3},\;
            \frac{2}{3},\;
            \frac{2}{3}
            \right].
    \]

    Thus, the orthogonal set is:
    \[
        \mathbf{u}_1=[1,1,0],\quad
        \mathbf{u}_2=\left[\frac{1}{2},-\frac{1}{2},1\right],\quad
        \mathbf{u}_3=\left[-\frac{2}{3},\frac{2}{3},\frac{2}{3}\right].
    \]

    \textbf{(Optional) Check orthogonality:}

    \[
        \mathbf{u}_1\cdot\mathbf{u}_2
        =
        [1,1,0]\cdot\left[\frac{1}{2},-\frac{1}{2},1\right]
        =
        \frac{1}{2}-\frac{1}{2}+0
        =
        0.
    \]

    \[
        \mathbf{u}_1\cdot\mathbf{u}_3
        =
        [1,1,0]\cdot\left[-\frac{2}{3},\frac{2}{3},\frac{2}{3}\right]
        =
        -\frac{2}{3}+\frac{2}{3}+0
        =
        0.
    \]

    So $\mathbf{u}_1$ is orthogonal to both $\mathbf{u}_2$ and $\mathbf{u}_3$.
\end{examplebox}

\begin{noteBox}
    The orthonormal version can be obtained by dividing each $\mathbf{u}_i$ by its magnitude.
    This is often done in numerical methods and QR factorization.
\end{noteBox}
