\subsection{Applications}

\subsubsection{Work Done by a Constant Force}

\begin{conceptbox}
    If a constant force $\mathbf{F}$ displaces an object by $\mathbf{d}$, then the \textbf{work} done is
    \[
        W=\mathbf{F}\cdot\mathbf{d}=\|\mathbf{F}\|\,\|\mathbf{d}\|\cos\gamma.
    \]
\end{conceptbox}

\begin{examplebox}
    A force $\mathbf{F}=[6,-3,0]$ moves an object by $\mathbf{d}=[2,5,0]$.

    \[
        W=\mathbf{F}\cdot\mathbf{d}=6(2)+(-3)(5)+0(0)=12-15=-3.
    \]
    Since $W<0$, the force opposes the displacement (work is done \emph{against} the force).
\end{examplebox}

\subsubsection{Decomposing a Vector into Parallel and Perpendicular Parts}

\begin{noteBox}
    In engineering (statics/dynamics), it is common to decompose a force into:
    \begin{itemize}
        \item a component \textbf{parallel} to a surface or direction
        \item a component \textbf{perpendicular} (normal) to the surface
    \end{itemize}
    This is done using projection.
\end{noteBox}

If $\mathbf{b}\ne\mathbf{0}$, the component of $\mathbf{a}$ parallel to $\mathbf{b}$ is
\[
    \mathbf{a}_{\parallel}=\mathrm{proj}_{\mathbf{b}}(\mathbf{a}),
\]
and the perpendicular component is
\[
    \mathbf{a}_{\perp}=\mathbf{a}-\mathbf{a}_{\parallel}.
\]
