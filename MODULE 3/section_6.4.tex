\subsection{Orthonormalization}

The Gram-Schmidt process produces an \textbf{orthogonal set}.
To convert this into an \textbf{orthonormal set}, we simply normalize each vector.

\subsubsection{Normalization}

\begin{conceptbox}
    Given a nonzero vector $\mathbf{u}$, its corresponding unit vector is:
    \[
        \mathbf{e}=\frac{\mathbf{u}}{\|\mathbf{u}\|}.
    \]
\end{conceptbox}

Thus, if Gram-Schmidt produces:
\[
    \mathbf{u}_1,\mathbf{u}_2,\dots,\mathbf{u}_n,
\]
then the orthonormal vectors are:
\[
    \mathbf{e}_1=\frac{\mathbf{u}_1}{\|\mathbf{u}_1\|},\quad
    \mathbf{e}_2=\frac{\mathbf{u}_2}{\|\mathbf{u}_2\|},\quad \dots,\quad
    \mathbf{e}_n=\frac{\mathbf{u}_n}{\|\mathbf{u}_n\|}.
\]

\begin{noteBox}
    Orthonormal vectors simplify calculations dramatically because:
    \[
        \mathbf{e}_i\cdot\mathbf{e}_j=
        \begin{cases}
            1, & i=j     \\
            0, & i\neq j
        \end{cases}
    \]
\end{noteBox}

\begin{examplebox}
    Let $\mathbf{u}=[6,-2,3]$.

    Compute its unit vector.

    \[
        \|\mathbf{u}\|=\sqrt{6^2+(-2)^2+3^2}
        =\sqrt{36+4+9}
        =\sqrt{49}=7.
    \]

    Thus the normalized vector is:
    \[
        \mathbf{e}=\frac{1}{7}[6,-2,3]
        =
        \left[\frac{6}{7},-\frac{2}{7},\frac{3}{7}\right].
    \]
\end{examplebox}
