\subsection{Coordinates Relative to a Basis}

\subsubsection{Coordinate Representation}

\begin{conceptbox}
    Let $B=\{\mathbf{v}_1,\mathbf{v}_2,\dots,\mathbf{v}_n\}$ be a basis for a vector space $V$.
    Any vector $\mathbf{x}\in V$ can be written uniquely as:
    \[
        \mathbf{x}=c_1\mathbf{v}_1+c_2\mathbf{v}_2+\dots+c_n\mathbf{v}_n.
    \]
    The scalars $c_1,c_2,\dots,c_n$ are called the \textbf{coordinates of $\mathbf{x}$ relative to the basis $B$}.
\end{conceptbox}

We often write the coordinate vector as:
\[
    [\mathbf{x}]_B=
    \begin{bmatrix}
        c_1    \\
        c_2    \\
        \vdots \\
        c_n
    \end{bmatrix}.
\]

\subsubsection{Why Coordinates Change With Basis}

In standard coordinates, vectors are written using $\mathbf{e}_1,\mathbf{e}_2,\dots$.
But if we use a different basis, the coordinates of the same vector may change.

\begin{examplebox}
    Let the basis in $\mathbb{R}^2$ be:
    \[
        B=\left\{
        \mathbf{v}_1=[1,1],\;
        \mathbf{v}_2=[1,-1]
        \right\}.
    \]

    Write the vector
    \[
        \mathbf{x}=[4,2]
    \]
    as a linear combination of $\mathbf{v}_1$ and $\mathbf{v}_2$.

    We want:
    \[
        \mathbf{x}=c_1\mathbf{v}_1+c_2\mathbf{v}_2.
    \]

    Substitute:
    \[
        [4,2]=c_1[1,1]+c_2[1,-1].
    \]

    Component-wise:
    \[
        4=c_1+c_2
    \]
    \[
        2=c_1-c_2.
    \]

    Add the equations:
    \[
        4+2=2c_1
        \Rightarrow 6=2c_1
        \Rightarrow c_1=3.
    \]

    Then:
    \[
        4=3+c_2
        \Rightarrow c_2=1.
    \]

    Thus:
    \[
        \mathbf{x}=3\mathbf{v}_1+1\mathbf{v}_2.
    \]

    So the coordinate vector of $\mathbf{x}$ relative to basis $B$ is:
    \[
        [\mathbf{x}]_B=
        \begin{bmatrix}
            3 \\
            1
        \end{bmatrix}.
    \]
\end{examplebox}

\begin{noteBox}
    In engineering, using different bases is extremely useful. For example, Fourier analysis changes a signal from the time domain basis into a frequency domain basis.
\end{noteBox}
