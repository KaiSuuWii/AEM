\subsection{Gram-Schmidt Algorithm}

Suppose we are given a set of linearly independent vectors:
\[
    \{\mathbf{v}_1,\mathbf{v}_2,\dots,\mathbf{v}_n\}.
\]

The goal is to generate an orthogonal set:
\[
    \{\mathbf{u}_1,\mathbf{u}_2,\dots,\mathbf{u}_n\}
\]
that spans the same subspace.

\subsubsection{Step-by-Step Process}

\begin{conceptbox}
    The Gram-Schmidt process is defined as:
    \[
        \mathbf{u}_1 = \mathbf{v}_1
    \]
    \[
        \mathbf{u}_2 = \mathbf{v}_2 - \mathrm{proj}_{\mathbf{u}_1}(\mathbf{v}_2)
    \]
    \[
        \mathbf{u}_3 = \mathbf{v}_3 - \mathrm{proj}_{\mathbf{u}_1}(\mathbf{v}_3) - \mathrm{proj}_{\mathbf{u}_2}(\mathbf{v}_3)
    \]
    and so on.
\end{conceptbox}

In general:
\[
    \mathbf{u}_k
    =
    \mathbf{v}_k
    -
    \sum_{j=1}^{k-1}\mathrm{proj}_{\mathbf{u}_j}(\mathbf{v}_k).
\]

\subsubsection{Why It Works}

Each new vector $\mathbf{u}_k$ is obtained by subtracting from $\mathbf{v}_k$ its components in the directions
of all previous $\mathbf{u}_1,\mathbf{u}_2,\dots,\mathbf{u}_{k-1}$.

Thus, the remaining part is orthogonal to all previous vectors.

\begin{noteBox}
    Gram-Schmidt is like ``removing overlap'' from vectors.
    It forces the new vector to be perpendicular to all previously formed vectors.
\end{noteBox}
