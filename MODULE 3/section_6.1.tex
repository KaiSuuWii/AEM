\section{Gram-Schmidt Orthogonalization}

\subsection{Motivation and Concept}

In many engineering and mathematical applications, it is useful to represent vectors using
\textbf{orthogonal} or \textbf{orthonormal} bases instead of arbitrary bases.

This is because orthogonal bases make computations easier, especially in:
\begin{itemize}
    \item projections
    \item least squares approximation
    \item Fourier analysis
    \item signal processing
    \item numerical computations
\end{itemize}

\begin{conceptbox}
    A set of vectors $\{\mathbf{v}_1,\mathbf{v}_2,\dots,\mathbf{v}_k\}$ is called
    \textbf{orthogonal} if
    \[
        \mathbf{v}_i\cdot\mathbf{v}_j = 0 \quad \text{for } i\neq j.
    \]
\end{conceptbox}

\begin{conceptbox}
    A set of vectors is called \textbf{orthonormal} if:
    \begin{enumerate}[label=(\alph*)]
        \item the vectors are orthogonal, and
        \item each vector has magnitude 1.
    \end{enumerate}

    That is,
    \[
        \mathbf{v}_i\cdot\mathbf{v}_j =
        \begin{cases}
            1, & i=j,     \\
            0, & i\neq j.
        \end{cases}
    \]
\end{conceptbox}

\begin{noteBox}
    Orthonormal bases are extremely useful because they simplify dot products:
    \[
        \mathbf{x}=c_1\mathbf{u}_1+\cdots+c_n\mathbf{u}_n
        \quad \Rightarrow \quad
        c_i=\mathbf{x}\cdot\mathbf{u}_i
    \]
    when $\{\mathbf{u}_1,\dots,\mathbf{u}_n\}$ is orthonormal.
\end{noteBox}

The \textbf{Gram-Schmidt process} is an algorithm that converts a set of linearly independent vectors
into an orthogonal (or orthonormal) set spanning the same subspace.
