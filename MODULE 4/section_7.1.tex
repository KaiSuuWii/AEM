\section{Cryptography}

\subsection{Hill Cipher Encryption (Matrix-Based Encryption)}

The \textbf{Hill Cipher} is a classical block cipher that uses matrix multiplication
over modular arithmetic. It is a beautiful application of linear algebra, even though
it is not secure by modern standards.

%------------------------------------------------

\subsubsection*{Alphabet Mapping and Modular Arithmetic}

We map letters to integers:

\[
    A=0,\; B=1,\; \dots,\; Z=25.
\]

All computations are done modulo 26:

\[
    y \equiv Kx \pmod{26}.
\]

\begin{conceptbox}
    In the Hill cipher, encryption is a linear transformation over the ring $\mathbb{Z}_{26}$.
\end{conceptbox}

\subsubsection*{Handling Negative Values}

When reducing modulo 26, negative numbers are converted by adding 26 repeatedly:

\[
    -3 \equiv 23 \pmod{26},\quad -29 \equiv -29 + 26 = -3 \equiv 23 \pmod{26}.
\]

%------------------------------------------------

\subsubsection*{Block Size and Padding}

Choose an $n \times n$ key matrix $K$. The plaintext is grouped into blocks of size $n$.

\begin{itemize}
    \item If the message length is not a multiple of $n$, we add padding (commonly $X=23$).
    \item Each block becomes a column vector $x \in \mathbb{Z}_{26}^n$.
\end{itemize}

Encryption for each block is:

\[
    y = Kx \pmod{26}.
\]

%------------------------------------------------

\subsubsection*{Key Matrix Requirement}

Not every matrix works as a key. We must be able to decrypt.

\begin{conceptbox}
    The key matrix $K$ must be invertible modulo 26, otherwise decryption is impossible.
\end{conceptbox}

(We will formalize this condition in the next subsection.)

%------------------------------------------------
\subsubsection*{Example 1 (2 $\times$ 2): Digraph Encryption}

Let

\[
    K =
    \begin{bmatrix}
        3 & 3 \\
        2 & 5
    \end{bmatrix}.
\]

Encrypt the plaintext block ``HI''.

\[
    H = 7,\quad I = 8
    \quad\Rightarrow\quad
    x =
    \begin{bmatrix}
        7 \\
        8
    \end{bmatrix}.
\]

Compute:

\[
    y = Kx =
    \begin{bmatrix}
        3 & 3 \\
        2 & 5
    \end{bmatrix}
    \begin{bmatrix}
        7 \\
        8
    \end{bmatrix}
    =
    \begin{bmatrix}
        45 \\
        54
    \end{bmatrix}.
\]

Reduce mod 26:

\[
    45 \equiv 19,\quad 54 \equiv 2 \pmod{26}.
\]

Thus,

\[
    y =
    \begin{bmatrix}
        19 \\
        2
    \end{bmatrix}
    \Rightarrow
    (T,C).
\]

Ciphertext is \textbf{TC}.

%------------------------------------------------
\subsubsection*{Example 2 (2 $\times$ 2): Multi-block Encryption with Padding}

Let the same key matrix $K$ be used. Encrypt ``HELP''.

Convert letters:

\[
    H=7,\; E=4,\; L=11,\; P=15.
\]

Block into pairs (since $n=2$):

\[
    x_1 =
    \begin{bmatrix}
        7 \\4
    \end{bmatrix},
    \quad
    x_2 =
    \begin{bmatrix}
        11 \\15
    \end{bmatrix}.
\]

Encrypt first block:

\[
    y_1 = Kx_1 =
    \begin{bmatrix}
        3(7)+3(4) \\
        2(7)+5(4)
    \end{bmatrix}
    =
    \begin{bmatrix}
        33 \\
        34
    \end{bmatrix}
    \equiv
    \begin{bmatrix}
        7 \\
        8
    \end{bmatrix}
    \pmod{26}.
\]

So $y_1 \Rightarrow (H,I)$.

Encrypt second block:

\[
    y_2 = Kx_2 =
    \begin{bmatrix}
        3(11)+3(15) \\
        2(11)+5(15)
    \end{bmatrix}
    =
    \begin{bmatrix}
        78 \\
        97
    \end{bmatrix}
    \equiv
    \begin{bmatrix}
        0 \\
        19
    \end{bmatrix}
    \pmod{26}.
\]

So $y_2 \Rightarrow (A,T)$.

Ciphertext: \textbf{HIAT}.

\begin{noteBox}
    This example shows that each plaintext block is transformed independently
    using the same linear map $K$ in $\mathbb{Z}_{26}^n$.
\end{noteBox}

%------------------------------------------------
\subsubsection*{Example 3 (3 $\times$ 3): Trigraph Encryption (More Complex)}

Let the key matrix be:

\[
    K =
    \begin{bmatrix}
        6  & 24 & 1  \\
        13 & 16 & 10 \\
        20 & 17 & 15
    \end{bmatrix}.
\]

Encrypt the plaintext ``ACT'' (a classic trigraph example).

\[
    A=0,\quad C=2,\quad T=19
    \Rightarrow
    x=
    \begin{bmatrix}
        0 \\2\\19
    \end{bmatrix}.
\]

Compute $y = Kx$:

\[
    y =
    \begin{bmatrix}
        6(0)+24(2)+1(19)   \\
        13(0)+16(2)+10(19) \\
        20(0)+17(2)+15(19)
    \end{bmatrix}
    =
    \begin{bmatrix}
        67  \\
        222 \\
        319
    \end{bmatrix}.
\]

Reduce mod 26:

\[
    67 \equiv 15,\quad
    222 \equiv 14,\quad
    319 \equiv 7
    \pmod{26}.
\]

Thus,

\[
    y=
    \begin{bmatrix}
        15 \\
        14 \\
        7
    \end{bmatrix}
    \Rightarrow (P,O,H).
\]

Ciphertext is \textbf{POH}.

\begin{conceptbox}
    A larger block size $n$ increases diffusion: each ciphertext letter depends on multiple plaintext letters.
\end{conceptbox}

%------------------------------------------------
\subsubsection*{Practical Implementation Notes}

\begin{itemize}
    \item Always choose a key matrix $K$ that is invertible modulo 26.
    \item Agree on block size and padding rule (e.g., pad with X).
    \item Encryption is fast: it is just matrix multiplication mod 26.
\end{itemize}