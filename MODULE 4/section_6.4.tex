\subsection{Pivoting and Numerical Stability}

LU without pivoting may fail if a pivot is zero.

\subsubsection*{Example}

\[
    A =
    \begin{bmatrix}
        0 & 2 \\
        1 & 3
    \end{bmatrix}.
\]

The first pivot is zero.

We interchange rows:

\[
    P A =
    \begin{bmatrix}
        1 & 3 \\
        0 & 2
    \end{bmatrix}.
\]

\begin{conceptbox}
    With pivoting:

    \[
        PA = LU.
    \]
\end{conceptbox}

$P$ is a permutation matrix.

%------------------------------------------------

\subsubsection*{Why Pivoting Is Important}

\begin{itemize}
    \item Avoids division by zero.
    \item Reduces round-off error.
    \item Improves numerical stability.
\end{itemize}

%------------------------------------------------

\subsubsection*{Partial Pivoting}

Choose the largest absolute value in the pivot column.

This leads to:

\[
    PA = LU.
\]

\begin{noteBox}
    In practical computation (MATLAB, NumPy),
    LU decomposition always uses pivoting.
\end{noteBox}