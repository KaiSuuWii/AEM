\subsection{Powers of Special Matrices}

\subsubsection*{1. Diagonal Matrices}

\begin{conceptbox}
    If
    \[
        D =
        \begin{bmatrix}
            \lambda_1 & 0         \\
            0         & \lambda_2
        \end{bmatrix},
    \]
    then
    \[
        D^n =
        \begin{bmatrix}
            \lambda_1^n & 0           \\
            0           & \lambda_2^n
        \end{bmatrix}.
    \]
\end{conceptbox}

Powers of diagonal matrices are computed by raising each diagonal entry to the $n$th power.

%----------------------------------------

\subsubsection*{2. Identity Matrix}

\[
    I^n = I.
\]

%----------------------------------------

\subsubsection*{3. Nilpotent Matrix}

\begin{conceptbox}
    A matrix $N$ is \textbf{nilpotent} if
    \[
        N^k = 0
    \]
    for some positive integer $k$.
\end{conceptbox}

\begin{examplebox}
    \[
        N =
        \begin{bmatrix}
            0 & 1 \\
            0 & 0
        \end{bmatrix}.
    \]

    \[
        N^2 =
        \begin{bmatrix}
            0 & 0 \\
            0 & 0
        \end{bmatrix}.
    \]

    Thus $N$ is nilpotent of index 2.
\end{examplebox}