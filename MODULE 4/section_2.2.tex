\subsection{Eigenvectors and the Structure of the Solution Space}

After computing eigenvalues from

\[
    \det(A - \lambda I) = 0,
\]

we determine eigenvectors by solving

\[
    (A - \lambda I)v = 0.
\]

\begin{conceptbox}
    For a fixed eigenvalue $\lambda$, the set of all eigenvectors (together with the zero vector)
    forms the null space of $(A - \lambda I)$.
\end{conceptbox}

Thus eigenvectors are obtained by solving a homogeneous linear system.

\subsubsection*{Geometric Meaning of the Equation}

The equation

\[
    Av = \lambda v
\]

means:

\begin{itemize}
    \item $v$ is scaled but not rotated.
    \item $v$ is an invariant direction under transformation $A$.
\end{itemize}

If $\lambda > 1$ → stretching \\
If $0 < \lambda < 1$ → contraction \\
If $\lambda < 0$ → direction reversal

%----------------------------------------

\subsubsection*{Dimension of the Eigenspace}

\begin{conceptbox}
    The set
    \[
        E_\lambda = \{v \neq 0 \mid (A - \lambda I)v = 0\}
    \]
    is called the \textbf{eigenspace} corresponding to $\lambda$.
\end{conceptbox}

Its dimension equals:

\[
    \text{dim}(\ker(A - \lambda I)).
\]

This dimension is crucial for determining diagonalizability.

%----------------------------------------

\subsubsection*{Example: Full Computation}

Let

\[
    A =
    \begin{bmatrix}
        4 & 1 \\
        2 & 3
    \end{bmatrix}.
\]

Eigenvalues:

\[
    \lambda_1 = 5, \quad \lambda_2 = 2.
\]

For $\lambda = 5$:

\[
    A - 5I =
    \begin{bmatrix}
        -1 & 1  \\
        2  & -2
    \end{bmatrix}.
\]

Row reduce:

\[
    \begin{bmatrix}
        -1 & 1  \\
        2  & -2
    \end{bmatrix}
    \rightarrow
    \begin{bmatrix}
        1 & -1 \\
        0 & 0
    \end{bmatrix}.
\]

Thus:

\[
    x - y = 0 \Rightarrow x = y.
\]

Eigenvectors:

\[
    v_1 =
    \begin{bmatrix}
        1 \\
        1
    \end{bmatrix}.
\]

We always report eigenvectors up to scalar multiples.

\begin{noteBox}
    Each eigenvalue produces its own linear subspace.
    Different eigenspaces intersect only at the zero vector.
\end{noteBox}