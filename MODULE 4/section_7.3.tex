\subsection{Security Analysis and Structural Weaknesses}

Although mathematically elegant, the Hill cipher is fundamentally insecure.
Its weakness comes from the fact that encryption is a \textbf{linear transformation}.

%------------------------------------------------
\subsubsection*{Linearity of the Cipher}

Encryption is defined as:

\[
    y \equiv Kx \pmod{26}.
\]

This is a linear map over $\mathbb{Z}_{26}^n$.

\begin{conceptbox}
    Because encryption is linear, the entire cipher can be described
    as solving a system of linear equations.
\end{conceptbox}

This linearity makes the system vulnerable.

%------------------------------------------------
\subsubsection*{Known-Plaintext Attack (Core Weakness)}

Suppose an attacker knows:

\[
    x_1 \mapsto y_1, \quad
    x_2 \mapsto y_2, \quad
    \dots, \quad
    x_n \mapsto y_n.
\]

Stack these into matrices:

\[
    X = [x_1 \; x_2 \; \dots \; x_n],
    \quad
    Y = [y_1 \; y_2 \; \dots \; y_n].
\]

From encryption:

\[
    Y \equiv KX \pmod{26}.
\]

If $X$ is invertible modulo 26, then:

\[
    K \equiv YX^{-1} \pmod{26}.
\]

\begin{conceptbox}
    If an attacker obtains $n$ independent plaintext–ciphertext pairs
    (for an $n \times n$ key),
    they can recover the key matrix exactly.
\end{conceptbox}

%------------------------------------------------
\subsubsection*{Concrete Attack Example (2 $\times$ 2 Key)}

Suppose the attacker knows:

Plaintext blocks:
\[
    x_1 =
    \begin{bmatrix}
        7 \\
        8
    \end{bmatrix},
    \quad
    x_2 =
    \begin{bmatrix}
        4 \\
        11
    \end{bmatrix}.
\]

Ciphertext blocks:
\[
    y_1 =
    \begin{bmatrix}
        19 \\
        2
    \end{bmatrix},
    \quad
    y_2 =
    \begin{bmatrix}
        7 \\
        8
    \end{bmatrix}.
\]

Construct matrices:

\[
    X =
    \begin{bmatrix}
        7 & 4  \\
        8 & 11
    \end{bmatrix},
    \quad
    Y =
    \begin{bmatrix}
        19 & 7 \\
        2  & 8
    \end{bmatrix}.
\]

Then:

\[
    K \equiv YX^{-1} \pmod{26}.
\]

The attacker simply computes a matrix inverse modulo 26 —
exactly the same operation used for decryption.

Thus the encryption key is fully recovered.

%------------------------------------------------
\subsubsection*{Why Larger Block Size Does Not Fix It}

One might think using a $3 \times 3$ or $4 \times 4$ matrix improves security.

However:

\begin{itemize}
    \item A $3 \times 3$ key requires only 3 independent plaintext blocks.
    \item A $4 \times 4$ key requires only 4 independent blocks.
\end{itemize}

Once enough linear equations are collected,
the key is determined uniquely.

\begin{conceptbox}
    Increasing matrix size increases work slightly,
    but does not remove the fundamental linear vulnerability.
\end{conceptbox}

%------------------------------------------------
\subsubsection*{Structural Reason for Failure}

The Hill cipher operates entirely within:

\[
    \text{Linear Algebra over } \mathbb{Z}_{26}.
\]

Modern cryptography avoids purely linear systems because:

\begin{itemize}
    \item Linear systems are predictable.
    \item They can be solved with Gaussian elimination.
    \item They have no avalanche effect (small changes do not create chaotic output).
\end{itemize}

%------------------------------------------------
\subsubsection*{Comparison with Modern Cryptography}

Modern systems (AES, RSA, ECC) use:

\begin{itemize}
    \item Nonlinear substitutions
    \item Finite field arithmetic
    \item Modular exponentiation
    \item Elliptic curves
\end{itemize}

These prevent direct linear reconstruction.

%------------------------------------------------
\subsubsection*{Educational Value of the Hill Cipher}

Despite its weakness, the Hill cipher is valuable because it demonstrates:

\begin{itemize}
    \item Matrix multiplication
    \item Modular inverses
    \item Determinant-based invertibility
    \item Linear transformations
    \item Real-world application of LU-style solving
\end{itemize}

\begin{noteBox}
    The Hill cipher is not secure — but it is a powerful teaching tool
    for understanding linear algebra in cryptographic systems.
\end{noteBox}