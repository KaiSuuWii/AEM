\section{Eigenvalues and Eigenvectors}

\subsection{Eigenvalues and the Characteristic Equation}

Consider a square matrix $A \in \mathbb{R}^{n \times n}$.

\begin{conceptbox}
    A nonzero vector $v$ is called an \textbf{eigenvector} of $A$ if
    \[
        Av = \lambda v
    \]
    for some scalar $\lambda$.

    The scalar $\lambda$ is called an \textbf{eigenvalue} of $A$.
\end{conceptbox}

\subsubsection*{Interpretation}

The equation

\[
    Av = \lambda v
\]

means:

\begin{itemize}
    \item The transformation $A$ does not change the direction of $v$.
    \item It only scales $v$ by the factor $\lambda$.
\end{itemize}

Thus, eigenvectors represent \textbf{invariant directions} of a linear transformation.

\subsubsection*{Deriving the Characteristic Equation}

Starting from:

\[
    Av = \lambda v
\]

\[
    Av - \lambda v = 0
\]

\[
    (A - \lambda I)v = 0
\]

For a nontrivial solution ($v \neq 0$) to exist, we require:

\[
    \det(A - \lambda I) = 0.
\]

\begin{conceptbox}
    The equation
    \[
        \det(A - \lambda I) = 0
    \]
    is called the \textbf{characteristic equation}.
\end{conceptbox}

The resulting polynomial in $\lambda$ is called the \textbf{characteristic polynomial}.

%----------------------------------------
\subsubsection*{Example: 2 $\times$ 2 Matrix}

Let

\[
    A =
    \begin{bmatrix}
        4 & 1 \\
        2 & 3
    \end{bmatrix}.
\]

Compute:

\[
    A - \lambda I =
    \begin{bmatrix}
        4 - \lambda & 1           \\
        2           & 3 - \lambda
    \end{bmatrix}.
\]

\[
    \det(A - \lambda I)
    =
    (4 - \lambda)(3 - \lambda) - 2.
\]

\[
    =
    \lambda^2 - 7\lambda + 10.
\]

Set equal to zero:

\[
    \lambda^2 - 7\lambda + 10 = 0.
\]

\[
    (\lambda - 5)(\lambda - 2) = 0.
\]

\[
    \lambda_1 = 5, \quad \lambda_2 = 2.
\]

\begin{noteBox}
    A $2 \times 2$ matrix always has a quadratic characteristic polynomial.
\end{noteBox}