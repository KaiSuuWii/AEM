\subsection{Conditions for Diagonalizability}

\begin{conceptbox}
    An $n \times n$ matrix is diagonalizable
    if and only if it has $n$ linearly independent eigenvectors.
\end{conceptbox}

This is equivalent to:

\[
    \text{Geometric multiplicity} = \text{Algebraic multiplicity}
\]

for every eigenvalue.

%------------------------------------------------

\subsubsection*{Important Cases}

\subsubsection*{1. Distinct Eigenvalues}

\begin{conceptbox}
    If a matrix has $n$ distinct eigenvalues,
    it is automatically diagonalizable.
\end{conceptbox}

\subsubsection*{2. Repeated Eigenvalues}

Repeated eigenvalues require checking the dimension of eigenspaces.

\subsubsection*{3. Symmetric Matrices}
\begin{conceptbox}
    Every real symmetric matrix is diagonalizable by an orthogonal matrix.
\end{conceptbox}

That is,

\[
    A = Q D Q^T.
\]

%------------------------------------------------

\subsubsection*{When Diagonalization Fails}

If geometric multiplicity is strictly less than algebraic multiplicity,
the matrix is defective and cannot be diagonalized.