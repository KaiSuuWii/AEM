\subsection{Orthogonal Diagonalization and the Spectral Theorem}

Orthogonal matrices play a central role in diagonalizing symmetric matrices.

\begin{conceptbox}
    \textbf{Spectral Theorem (Real Case)}

    If $A$ is a real symmetric matrix, then:

    \[
        A = Q D Q^T,
    \]

    where:
    \begin{itemize}
        \item $Q$ is orthogonal,
        \item $D$ is diagonal,
        \item The diagonal entries are eigenvalues of $A$.
    \end{itemize}
\end{conceptbox}

%------------------------------------------------

\subsubsection*{Why Symmetry Matters}

If $A = A^T$, then:

\begin{itemize}
    \item All eigenvalues are real.
    \item Eigenvectors corresponding to distinct eigenvalues are orthogonal.
\end{itemize}

%------------------------------------------------

\subsubsection*{Energy Interpretation}

For symmetric $A$:

\[
    x^T A x
\]

represents a quadratic form.

Orthogonal diagonalization transforms it into:

\[
    x^T A x
    =
    y^T D y.
\]

This simplifies:

\begin{itemize}
    \item Principal axis transformation
    \item Optimization problems
    \item PCA
\end{itemize}

%------------------------------------------------

\subsubsection*{Engineering Importance}

\begin{noteBox}
    Orthogonal diagonalization appears in:

    \begin{itemize}
        \item Vibration mode analysis
        \item Structural mechanics
        \item Principal component analysis
        \item Image compression
        \item Least squares methods
    \end{itemize}
\end{noteBox}