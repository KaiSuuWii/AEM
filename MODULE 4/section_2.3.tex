\subsection{Defective Matrices and Diagonalizability}

\subsubsection*{Diagonalizability}

\begin{conceptbox}
    A matrix $A$ is \textbf{diagonalizable} if there exists an invertible matrix $P$ such that

    \[
        A = PDP^{-1},
    \]

    where $D$ is a diagonal matrix containing eigenvalues of $A$.
\end{conceptbox}

\begin{conceptbox}
    A matrix is diagonalizable if and only if it has $n$ linearly independent eigenvectors.
\end{conceptbox}

This means:

\[
    \text{Geometric multiplicity} = \text{Algebraic multiplicity}
    \quad \text{for all eigenvalues}.
\]


\subsubsection*{Example: Defective Matrix}

Consider:

\[
    A =
    \begin{bmatrix}
        2 & 1 \\
        0 & 2
    \end{bmatrix}.
\]

Characteristic equation:

\[
    \det(A - \lambda I)
    =
    (2-\lambda)^2.
\]

So the eigenvalue is:

\[
    \lambda = 2 \quad \text{(algebraic multiplicity 2)}.
\]

Now compute eigenvectors:

\[
    A - 2I =
    \begin{bmatrix}
        0 & 1 \\
        0 & 0
    \end{bmatrix}.
\]

Solve:

\[
    \begin{bmatrix}
        0 & 1 \\
        0 & 0
    \end{bmatrix}
    \begin{bmatrix}
        x \\
        y
    \end{bmatrix}
    =
    \begin{bmatrix}
        0 \\
        0
    \end{bmatrix}.
\]

This gives:

\[
    y = 0.
\]

Thus eigenvectors are:

\[
    v =
    \begin{bmatrix}
        1 \\
        0
    \end{bmatrix}.
\]

There is only one independent eigenvector.

\begin{conceptbox}
    Since geometric multiplicity (1) is less than algebraic multiplicity (2),
    the matrix is \textbf{defective} and not diagonalizable.
\end{conceptbox}