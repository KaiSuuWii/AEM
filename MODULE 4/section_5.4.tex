\subsection{Applications and Structural Meaning}

\subsubsection*{1. Matrix Powers}

Diagonalization simplifies:

\[
    A^n.
\]

\subsubsection*{2. Differential Equations}

For

\[
    x' = Ax,
\]

if

\[
    A = P D P^{-1},
\]

then:

\[
    x(t) = P e^{Dt} P^{-1} x_0.
\]

Since:

\[
    e^{Dt} =
    \begin{bmatrix}
        e^{\lambda_1 t} & 0               \\
        0               & e^{\lambda_2 t}
    \end{bmatrix}.
\]

%------------------------------------------------

\subsubsection*{3. Decoupling of Systems}

Diagonalization decouples coupled systems
into independent scalar equations.

%------------------------------------------------

\subsubsection*{4. Geometric Interpretation}

The transformation:

\[
    A = P D P^{-1}
\]

means:

\begin{enumerate}
    \item Change basis using $P^{-1}$.
    \item Apply independent scaling using $D$.
    \item Transform back using $P$.
\end{enumerate}

\begin{conceptbox}
    Diagonalization reveals the intrinsic structure of a linear transformation.
\end{conceptbox}

%------------------------------------------------

\subsubsection*{Engineering Importance}

\begin{noteBox}
    Diagonalization is fundamental in:

    \begin{itemize}
        \item Vibrations and modal analysis
        \item Control theory
        \item Principal Component Analysis (PCA)
        \item Quantum mechanics
        \item Signal processing
    \end{itemize}
\end{noteBox}