\subsection{Decryption and Modular Matrix Inverses}

To decrypt Hill cipher ciphertext, we need the inverse of the key matrix \emph{in modular arithmetic}.

If encryption is:

\[
    y \equiv Kx \pmod{26},
\]

then decryption is:

\[
    x \equiv K^{-1}y \pmod{26}.
\]

\begin{conceptbox}
    Decryption is only possible if $K$ has a modular inverse modulo 26.
\end{conceptbox}

%------------------------------------------------
\subsubsection*{Invertibility Condition (Key Feasibility)}

In ordinary linear algebra, $K$ is invertible if $\det(K)\neq 0$.

In modular arithmetic, the correct condition is:

\begin{conceptbox}
    A matrix $K$ is invertible modulo 26 if and only if
    \[
        \gcd(\det K, 26) = 1.
    \]
\end{conceptbox}

Why?

Because we need $(\det K)^{-1}$ modulo 26 to compute $K^{-1}$,
and a modular inverse exists only when the number is relatively prime to 26.

\begin{noteBox}
    If $\gcd(\det K, 26)\neq 1$, then $\det K$ has no modular inverse,
    so $K^{-1}$ does not exist in $\mathbb{Z}_{26}$.
\end{noteBox}

%------------------------------------------------
\subsubsection*{Modular Inverse of an Integer (Extended Euclidean Idea)}

To find $a^{-1} \pmod{26}$, we want a number $t$ such that:

\[
    at \equiv 1 \pmod{26}.
\]

This is equivalent to:

\[
    at - 26k = 1
\]

for some integer $k$. This is what the extended Euclidean algorithm finds.

\begin{examplebox}
    Find $9^{-1} \pmod{26}$.

    We test multiples:
    \[
        9\cdot 3 = 27 \equiv 1 \pmod{26}.
    \]

    Thus:
    \[
        9^{-1} \equiv 3 \pmod{26}.
    \]
\end{examplebox}

%------------------------------------------------
\subsubsection*{Inverse of a $2 \times 2$ Matrix Modulo 26}

Let

\[
    K=
    \begin{bmatrix}
        a & b \\
        c & d
    \end{bmatrix}.
\]

In ordinary algebra:

\[
    K^{-1}=\frac{1}{ad-bc}
    \begin{bmatrix}
        d  & -b \\
        -c & a
    \end{bmatrix}.
\]

In modular arithmetic we replace division by multiplication with the modular inverse of the determinant:

\begin{conceptbox}
    \[
        K^{-1} \equiv (\det K)^{-1}\,\operatorname{adj}(K) \pmod{26}.
    \]
\end{conceptbox}

where:

\[
    \det K = ad-bc,
    \quad
    \operatorname{adj}(K)=
    \begin{bmatrix}
        d  & -b \\
        -c & a
    \end{bmatrix}.
\]

%------------------------------------------------
\subsubsection*{Example: Full Decryption for the $2 \times 2$ Key}

Use the key:

\[
    K=
    \begin{bmatrix}
        3 & 3 \\
        2 & 5
    \end{bmatrix},
    \quad
    \det K = 9.
\]

We already found:

\[
    9^{-1}\equiv 3 \pmod{26}.
\]

Adjugate matrix:

\[
    \operatorname{adj}(K)=
    \begin{bmatrix}
        5  & -3 \\
        -2 & 3
    \end{bmatrix}.
\]

Now compute $K^{-1}$:

\[
    K^{-1} \equiv 3
    \begin{bmatrix}
        5  & -3 \\
        -2 & 3
    \end{bmatrix}
    =
    \begin{bmatrix}
        15 & -9 \\
        -6 & 9
    \end{bmatrix}
    \pmod{26}.
\]

Reduce each entry mod 26:

\[
    15\equiv 15,\quad
    -9\equiv 17,\quad
    -6\equiv 20,\quad
    9\equiv 9.
\]

Thus:

\[
    K^{-1}\equiv
    \begin{bmatrix}
        15 & 17 \\
        20 & 9
    \end{bmatrix}
    \pmod{26}.
\]

\begin{examplebox}
    Decrypt the ciphertext block ``TC''.

    \[
        T=19,\quad C=2
        \Rightarrow
        y=
        \begin{bmatrix}
            19 \\2
        \end{bmatrix}.
    \]

    Compute:

    \[
        x \equiv K^{-1}y
        =
        \begin{bmatrix}
            15 & 17 \\
            20 & 9
        \end{bmatrix}
        \begin{bmatrix}
            19 \\2
        \end{bmatrix}
        =
        \begin{bmatrix}
            15(19)+17(2) \\
            20(19)+9(2)
        \end{bmatrix}
        =
        \begin{bmatrix}
            319 \\
            398
        \end{bmatrix}
        \pmod{26}.
    \]

    Reduce:

    \[
        319 \equiv 7,\quad 398 \equiv 8 \pmod{26}.
    \]

    So:

    \[
        x=
        \begin{bmatrix}
            7 \\8
        \end{bmatrix}
        \Rightarrow (H,I).
    \]

    We recovered the plaintext \textbf{HI}.
\end{examplebox}

%------------------------------------------------
\subsubsection*{How About $3 \times 3$ Inverses? (Important Insight)}

For a $3\times 3$ key matrix, the same principle holds:

\[
    K^{-1} \equiv (\det K)^{-1}\,\operatorname{adj}(K)\pmod{26}.
\]

But now:
\begin{itemize}
    \item $\det K$ is computed from the $3\times 3$ determinant formula,
    \item $\operatorname{adj}(K)$ requires cofactors (a full cofactor matrix transpose),
    \item then multiply by $(\det K)^{-1}$ modulo 26.
\end{itemize}

\begin{noteBox}
    Conceptually, decryption is still ``multiply by the inverse,'' but
    the computation is heavier for larger block sizes.
    That is why computers are used in practice.
\end{noteBox}

%------------------------------------------------
\subsubsection*{Common Student Pitfalls}

\begin{itemize}
    \item Forgetting to reduce values modulo 26 at the end.
    \item Mishandling negative entries (always convert to $0$--$25$).
    \item Choosing a key matrix with $\gcd(\det K, 26)\neq 1$ (non-invertible).
\end{itemize}