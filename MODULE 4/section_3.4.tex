\subsection{Applications to Recurrence Relations and Dynamical Systems}

\subsubsection*{1. Discrete Dynamical Systems}

Consider:

\[
    x_{k+1} = Ax_k.
\]

Iterating:

\[
    x_k = A^k x_0.
\]

Thus understanding $A^k$ determines the long-term behavior.

%----------------------------------------

\subsubsection*{2. Example: Fibonacci Sequence}

The Fibonacci recurrence:

\[
    F_{n+1} = F_n + F_{n-1}
\]

can be written as:

\[
    \begin{bmatrix}
        F_{n+1} \\
        F_n
    \end{bmatrix}
    =
    \begin{bmatrix}
        1 & 1 \\
        1 & 0
    \end{bmatrix}
    \begin{bmatrix}
        F_n \\
        F_{n-1}
    \end{bmatrix}.
\]

Thus:

\[
    \begin{bmatrix}
        F_{n} \\
        F_{n-1}
    \end{bmatrix}
    =
    A^{n-1}
    \begin{bmatrix}
        1 \\
        0
    \end{bmatrix}.
\]

Using eigenvalues:

\[
    \lambda = \frac{1 \pm \sqrt{5}}{2},
\]

we obtain Binet’s formula.

%----------------------------------------

\subsubsection*{3. Stability Insight}

\begin{conceptbox}
    If all eigenvalues satisfy $|\lambda| < 1$, then

    \[
        A^n \rightarrow 0
        \quad \text{as} \quad n \rightarrow \infty.
    \]
\end{conceptbox}

\begin{noteBox}
    Matrix powers allow prediction of long-term behavior in:
    \begin{itemize}
        \item Population models
        \item Control systems
        \item Markov chains
        \item Economic growth models
    \end{itemize}
\end{noteBox}