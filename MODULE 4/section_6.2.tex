\subsection{LU Decomposition (No Pivoting)}

Suppose we perform Gaussian elimination on:

\[
    A =
    \begin{bmatrix}
        2 & 3 \\
        4 & 7
    \end{bmatrix}.
\]

%------------------------------------------------

\subsubsection*{Step 1: Elimination}

We eliminate the entry below the pivot $a_{11}$.

The multiplier is:

\[
    m_{21} = \frac{a_{21}}{a_{11}} = \frac{4}{2} = 2.
\]

Apply the row operation:

\[
    R_2 \rightarrow R_2 - 2R_1.
\]

This produces the upper triangular matrix:

\[
    U =
    \begin{bmatrix}
        2 & 3 \\
        0 & 1
    \end{bmatrix}.
\]

%------------------------------------------------

\subsubsection*{How Does This Produce LU?}

Gaussian elimination can be written as matrix multiplication.

The row operation

\[
    R_2 \rightarrow R_2 - 2R_1
\]

is equivalent to multiplying $A$ on the left by the elimination matrix:

\[
    E =
    \begin{bmatrix}
        1  & 0 \\
        -2 & 1
    \end{bmatrix}.
\]

Indeed,

\[
    EA =
    \begin{bmatrix}
        1  & 0 \\
        -2 & 1
    \end{bmatrix}
    \begin{bmatrix}
        2 & 3 \\
        4 & 7
    \end{bmatrix}
    =
    \begin{bmatrix}
        2 & 3 \\
        0 & 1
    \end{bmatrix}
    =
    U.
\]

Thus,

\[
    EA = U.
\]

%------------------------------------------------

\subsubsection*{Solving for $A$}

From

\[
    EA = U,
\]

we solve for $A$:

\[
    A = E^{-1} U.
\]

Now compute the inverse of $E$:

\[
    E^{-1} =
    \begin{bmatrix}
        1 & 0 \\
        2 & 1
    \end{bmatrix}.
\]

Notice:

\begin{itemize}
    \item The inverse simply changes $-2$ to $+2$.
    \item This number $2$ is exactly the elimination multiplier.
\end{itemize}

We define:

\[
    L = E^{-1}.
\]

Thus:

\[
    A = LU.
\]

%------------------------------------------------

\subsubsection*{Why Does $L$ Look Like This?}

\[
    L =
    \begin{bmatrix}
        1 & 0 \\
        2 & 1
    \end{bmatrix}.
\]

Properties:

\begin{itemize}
    \item Ones on the diagonal
    \item Elimination multipliers below the diagonal
    \item Zeros above the diagonal
\end{itemize}

\begin{conceptbox}
    Matrix $L$ stores the elimination multipliers.
    Matrix $U$ is the resulting upper triangular matrix.
\end{conceptbox}

%------------------------------------------------

\subsubsection*{General $3 \times 3$ Structure}

For a $3 \times 3$ matrix:

\[
    L =
    \begin{bmatrix}
        1         & 0         & 0 \\
        \ell_{21} & 1         & 0 \\
        \ell_{31} & \ell_{32} & 1
    \end{bmatrix},
    \quad
    U =
    \begin{bmatrix}
        u_{11} & u_{12} & u_{13} \\
        0      & u_{22} & u_{23} \\
        0      & 0      & u_{33}
    \end{bmatrix}.
\]

Where:

\begin{itemize}
    \item $\ell_{21}, \ell_{31}, \ell_{32}$ are elimination multipliers.
    \item $U$ contains the final pivoted matrix after elimination.
\end{itemize}

%------------------------------------------------

\subsubsection*{Big Picture Insight}

Gaussian elimination can always be viewed as:

\[
    E_k \cdots E_2 E_1 A = U.
\]

Taking inverses:

\[
    A = E_1^{-1} E_2^{-1} \cdots E_k^{-1} U.
\]

The product of all inverse elimination matrices forms $L$.

\[
    A = LU.
\]

\begin{noteBox}
    LU decomposition is simply Gaussian elimination written in matrix form.
\end{noteBox}