\section{Orthogonal Matrices}

\subsection{Definition of Orthogonal Matrices}

\begin{conceptbox}
    A square matrix $Q \in \mathbb{R}^{n \times n}$ is called \textbf{orthogonal} if

    \[
        Q^T Q = I.
    \]

    Equivalently,

    \[
        Q^{-1} = Q^T.
    \]
\end{conceptbox}

Thus the transpose of an orthogonal matrix is its inverse.

%------------------------------------------------

\subsubsection*{Interpretation}

If $Q$ is orthogonal, then:

\[
    Q^T Q = I \Rightarrow Q^T = Q^{-1}.
\]

This means:

\begin{itemize}
    \item Applying $Q$ and then $Q^T$ returns the original vector.
    \item Orthogonal matrices represent reversible transformations.
\end{itemize}

%------------------------------------------------

\subsubsection*{Example}

Consider

\[
    Q =
    \begin{bmatrix}
        \cos\theta & -\sin\theta \\
        \sin\theta & \cos\theta
    \end{bmatrix}.
\]

Compute:

\[
    Q^T Q =
    \begin{bmatrix}
        \cos\theta  & \sin\theta \\
        -\sin\theta & \cos\theta
    \end{bmatrix}
    \begin{bmatrix}
        \cos\theta & -\sin\theta \\
        \sin\theta & \cos\theta
    \end{bmatrix}
    =
    I.
\]

Thus rotation matrices are orthogonal.

\begin{noteBox}
    Orthogonal matrices generalize rotations and reflections in $\mathbb{R}^n$.
\end{noteBox}