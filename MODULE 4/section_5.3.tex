\subsection{Procedure for Diagonalization}

\subsubsection*{Step-by-Step Method}

\begin{enumerate}
    \item Find eigenvalues from $\det(A - \lambda I)=0$.
    \item Compute eigenvectors for each eigenvalue.
    \item Form matrix $P$ using eigenvectors as columns.
    \item Form $D$ using eigenvalues along the diagonal.
\end{enumerate}

%------------------------------------------------

\subsubsection*{Full Example with Verification}

Let

\[
    A =
    \begin{bmatrix}
        4 & 1 \\
        2 & 3
    \end{bmatrix}.
\]

We previously computed:

\[
    \lambda_1 = 5, \quad \lambda_2 = 2.
\]

Eigenvectors:

\[
    v_1 =
    \begin{bmatrix}
        1 \\
        1
    \end{bmatrix},
    \quad
    v_2 =
    \begin{bmatrix}
        1 \\
        -2
    \end{bmatrix}.
\]

Construct:

\[
    P =
    \begin{bmatrix}
        1 & 1  \\
        1 & -2
    \end{bmatrix},
    \qquad
    D =
    \begin{bmatrix}
        5 & 0 \\
        0 & 2
    \end{bmatrix}.
\]

Compute determinant of $P$:

\[
    \det(P) = (1)(-2) - (1)(1) = -3.
\]

Thus,

\[
    P^{-1}
    =
    \frac{1}{-3}
    \begin{bmatrix}
        -2 & -1 \\
        -1 & 1
    \end{bmatrix}.
\]

%------------------------------------------------
\subsubsection*{Verification of $A = PDP^{-1}$}

First compute $PD$:

\[
    PD =
    \begin{bmatrix}
        1 & 1  \\
        1 & -2
    \end{bmatrix}
    \begin{bmatrix}
        5 & 0 \\
        0 & 2
    \end{bmatrix}
    =
    \begin{bmatrix}
        5 & 2  \\
        5 & -4
    \end{bmatrix}.
\]

Now multiply by $P^{-1}$:

\[
    PDP^{-1}
    =
    \begin{bmatrix}
        5 & 2  \\
        5 & -4
    \end{bmatrix}
    \frac{1}{-3}
    \begin{bmatrix}
        -2 & -1 \\
        -1 & 1
    \end{bmatrix}.
\]

Compute product inside first:

\[
    =
    \frac{1}{-3}
    \begin{bmatrix}
        (5)(-2)+(2)(-1)  & (5)(-1)+(2)(1)  \\
        (5)(-2)+(-4)(-1) & (5)(-1)+(-4)(1)
    \end{bmatrix}.
\]

\[
    =
    \frac{1}{-3}
    \begin{bmatrix}
        -10 -2 & -5 +2 \\
        -10 +4 & -5 -4
    \end{bmatrix}.
\]

\[
    =
    \frac{1}{-3}
    \begin{bmatrix}
        -12 & -3 \\
        -6  & -9
    \end{bmatrix}.
\]

\[
    =
    \begin{bmatrix}
        4 & 1 \\
        2 & 3
    \end{bmatrix}.
\]

Thus,

\[
    A = P D P^{-1}.
\]

\begin{conceptbox}
    We have verified that $A$ is diagonalizable.
\end{conceptbox}

%------------------------------------------------

\subsubsection*{Interpretation}

Diagonalization shows that:

\[
    A = P D P^{-1}
\]

means:

\begin{enumerate}
    \item Transform coordinates using $P^{-1}$.
    \item Apply independent scalings (5 and 2) via $D$.
    \item Transform back using $P$.
\end{enumerate}

The transformation $A$ acts as pure scaling in the eigenvector directions.

%------------------------------------------------

\subsubsection*{Power Computation}

Now powers are simple:

\[
    A^n = P D^n P^{-1}.
\]

\[
    =
    P
    \begin{bmatrix}
        5^n & 0   \\
        0   & 2^n
    \end{bmatrix}
    P^{-1}.
\]

\begin{noteBox}
    Instead of multiplying $A$ repeatedly,
    we raise eigenvalues to the $n$th power.
    This is the primary computational advantage of diagonalization.
\end{noteBox}