\section{Diagonalization}

\subsection{Definition of Diagonalization}

\begin{conceptbox}
    A square matrix $A$ is \textbf{diagonalizable} if there exists an invertible matrix $P$ such that

    \[
        A = P D P^{-1},
    \]

    where $D$ is a diagonal matrix.
\end{conceptbox}

The diagonal entries of $D$ are the eigenvalues of $A$.

%------------------------------------------------

\subsubsection*{Meaning of the Decomposition}

If

\[
    A = P D P^{-1},
\]

then:

\[
    P^{-1} A P = D.
\]

This means:

\begin{itemize}
    \item In a new coordinate system defined by $P$,
    \item The transformation $A$ acts as simple scaling.
\end{itemize}

Diagonal matrices are easy to compute with:

\[
    D^n =
    \begin{bmatrix}
        \lambda_1^n & 0           & \cdots & 0           \\
        0           & \lambda_2^n & \cdots & 0           \\
        \vdots      & \vdots      & \ddots & \vdots      \\
        0           & 0           & \cdots & \lambda_n^n
    \end{bmatrix}.
\]

%------------------------------------------------

\subsubsection*{Interpretation}

Diagonalization transforms a complicated linear transformation
into independent scaling along eigenvector directions.