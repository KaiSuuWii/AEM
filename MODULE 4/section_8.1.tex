\section{Error-Correcting Codes}

\subsection{Linear Codes and Generator Matrices}

Digital communication is subject to noise.
Error-correcting codes allow detection and correction of transmission errors.

%------------------------------------------------
\subsubsection*{Finite Field $\mathbb{Z}_2$}

Most classical codes operate over the binary field:

\[
    \mathbb{Z}_2 = \{0,1\}
\]

with arithmetic modulo 2:

\[
    1+1 \equiv 0.
\]

\begin{conceptbox}
    Binary linear codes are vector spaces over $\mathbb{Z}_2$.
\end{conceptbox}

%------------------------------------------------
\subsubsection*{Definition of a Linear Code}

A linear $(n,k)$ code is a $k$-dimensional subspace of $\mathbb{Z}_2^n$.

\begin{itemize}
    \item $k$ = number of message bits
    \item $n$ = number of transmitted bits
\end{itemize}

Thus:

\[
    \text{Redundancy} = n-k.
\]

%------------------------------------------------
\subsubsection*{Generator Matrix}

\begin{conceptbox}
    A generator matrix $G$ is a $k \times n$ matrix
    whose rows form a basis for the code.
\end{conceptbox}

Encoding is performed by:

\[
    c = mG
\]

where:
\begin{itemize}
    \item $m$ = message vector (row vector)
    \item $c$ = codeword
\end{itemize}

%------------------------------------------------
\subsubsection*{Example (3,2) Code}

Let:

\[
    G =
    \begin{bmatrix}
        1 & 0 & 1 \\
        0 & 1 & 1
    \end{bmatrix}.
\]

Encode message:

\[
    m = [1 \; 0].
\]

\[
    c = mG =
    [1 \; 0]
    \begin{bmatrix}
        1 & 0 & 1 \\
        0 & 1 & 1
    \end{bmatrix}
    =
    [1 \; 0 \; 1].
\]

All arithmetic is modulo 2.

%------------------------------------------------
\subsubsection*{Geometric View}

Codewords form a subspace of $\mathbb{Z}_2^n$.

\begin{noteBox}
    Linear algebra structure ensures:
    \begin{itemize}
        \item Sum of codewords is a codeword
        \item Scalar multiples remain in the code
    \end{itemize}
\end{noteBox}