\subsection{Eigenvalues in Applications and Stability}

Eigenvalues determine the long-term behavior of systems.

%----------------------------------------

\subsubsection*{Discrete Dynamical Systems}

Consider:

\[
    x_{k+1} = Ax_k.
\]

If $A$ has eigenvalues $\lambda_1, \lambda_2, \dots$:

\begin{itemize}
    \item If $|\lambda_i| < 1$, solutions decay.
    \item If $|\lambda_i| > 1$, solutions grow.
    \item If $|\lambda_i| = 1$, oscillatory behavior.
\end{itemize}

\begin{conceptbox}
    Stability of the system depends on the magnitude of eigenvalues.
\end{conceptbox}

%----------------------------------------

\subsubsection*{Vibration and Natural Frequencies}

In mechanical systems:

\[
    Kx = \lambda Mx,
\]

where:
\begin{itemize}
    \item $K$ = stiffness matrix
    \item $M$ = mass matrix
\end{itemize}

Eigenvalues determine natural frequencies:

\[
    \omega = \sqrt{\lambda}.
\]

%----------------------------------------

\subsubsection*{Engineering Insight}

\begin{noteBox}
    Eigenvalues appear in:
    \begin{itemize}
        \item Structural analysis
        \item Control systems
        \item Signal processing
        \item Population models
        \item Machine learning (PCA)
    \end{itemize}
\end{noteBox}

%----------------------------------------

\subsubsection*{Summary of Eigenvalue Behavior}

\[
    \begin{array}{c|c}
        \textbf{Eigenvalue} & \textbf{Behavior}  \\ \hline
        |\lambda| < 1       & Decay              \\
        |\lambda| > 1       & Growth             \\
        \lambda < 0         & Direction reversal \\
        \lambda = a \pm bi  & Rotation-scaling
    \end{array}
\]