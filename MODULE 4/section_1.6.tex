\subsection{Gaussian Elimination}

Gaussian elimination is a systematic procedure for solving a system of linear equations
written in matrix form:

\[
    Ax = b.
\]

The idea is to transform the augmented matrix $[A \mid b]$ into an upper triangular
(or row echelon) form using elementary row operations.

\subsubsection*{Elementary Row Operations}

\begin{itemize}
    \item $R_i \leftrightarrow R_j$ \hfill (Swap two rows)
    \item $R_i \rightarrow cR_i$, $c \neq 0$ \hfill (Scale a row)
    \item $R_i \rightarrow R_i + cR_j$ \hfill (Row replacement)
\end{itemize}

\begin{conceptbox}
    Gaussian elimination transforms a system into upper triangular form,
    after which the solution is obtained by back substitution.
\end{conceptbox}

%-----------------------------------
\subsubsection*{Example 1: 3 $\times$ 3 System with Unique Solution}

Solve:

\[
    \begin{aligned}
        x + y + z  & = 6 \\
        2x - y + z & = 3 \\
        x + 2y - z & = 3
    \end{aligned}
\]

Augmented matrix:

\[
    \left[
        \begin{array}{ccc|c}
            1 & 1  & 1  & 6 \\
            2 & -1 & 1  & 3 \\
            1 & 2  & -1 & 3
        \end{array}
        \right]
\]

Step 1: Eliminate below first pivot

\[
    R_2 \rightarrow R_2 - 2R_1
    \quad
    R_3 \rightarrow R_3 - R_1
\]

\[
    \left[
        \begin{array}{ccc|c}
            1 & 1  & 1  & 6  \\
            0 & -3 & -1 & -9 \\
            0 & 1  & -2 & -3
        \end{array}
        \right]
\]

Step 2: Eliminate below second pivot

\[
    R_3 \rightarrow R_3 + \frac{1}{3}R_2
\]

\[
    \left[
        \begin{array}{ccc|c}
            1 & 1  & 1            & 6  \\
            0 & -3 & -1           & -9 \\
            0 & 0  & -\frac{7}{3} & -6
        \end{array}
        \right]
\]

Now perform back substitution:

\[
    z = \frac{18}{7}, \quad
    y = \frac{9}{7}, \quad
    x = \frac{15}{7}.
\]

\begin{noteBox}
    Since we obtained three pivots (one in each row), the system has a unique solution.
\end{noteBox}

%-----------------------------------
\subsubsection*{Example 2: System with Infinitely Many Solutions}

Consider:

\[
    \begin{aligned}
        x + y + z    & = 2 \\
        2x + 2y + 2z & = 4
    \end{aligned}
\]

Augmented matrix:

\[
    \left[
        \begin{array}{ccc|c}
            1 & 1 & 1 & 2 \\
            2 & 2 & 2 & 4
        \end{array}
        \right]
\]

Eliminate:

\[
    R_2 \rightarrow R_2 - 2R_1
\]

\[
    \left[
        \begin{array}{ccc|c}
            1 & 1 & 1 & 2 \\
            0 & 0 & 0 & 0
        \end{array}
        \right]
\]

The second row becomes all zeros.

This corresponds to:

\[
    x + y + z = 2.
\]

We may assign free variables:

\[
    y = s, \quad z = t.
\]

Then:

\[
    x = 2 - s - t.
\]

\begin{conceptbox}
    If at least one variable is free (no pivot in its column),
    the system has infinitely many solutions.
\end{conceptbox}

%-----------------------------------
\subsubsection*{Example 3: Inconsistent System (No Solution)}

Consider:

\[
    \begin{aligned}
        x + y & = 1 \\
        x + y & = 3
    \end{aligned}
\]

Augmented matrix:

\[
    \left[
        \begin{array}{cc|c}
            1 & 1 & 1 \\
            1 & 1 & 3
        \end{array}
        \right]
\]

Eliminate:

\[
    R_2 \rightarrow R_2 - R_1
\]

\[
    \left[
        \begin{array}{cc|c}
            1 & 1 & 1 \\
            0 & 0 & 2
        \end{array}
        \right]
\]

The second row represents:

\[
    0 = 2,
\]

which is impossible.

\begin{conceptbox}
    If a row reduces to
    \[
        [0 \; 0 \; \cdots \; 0 \mid c], \quad c \neq 0,
    \]
    the system is inconsistent and has no solution.
\end{conceptbox}

%-----------------------------------
\subsubsection*{Pivoting and Numerical Stability}

If a pivot element is zero (or very small), we interchange rows.

\begin{examplebox}
    \[
        \left[
            \begin{array}{cc|c}
                0 & 1 & 2 \\
                3 & 4 & 5
            \end{array}
            \right]
    \]

    Swap rows to obtain a nonzero pivot:

    \[
        R_1 \leftrightarrow R_2.
    \]
\end{examplebox}

\begin{noteBox}
    Row swapping to improve numerical stability is called \textbf{partial pivoting}.
    It is essential in computational linear algebra.
\end{noteBox}

%-----------------------------------
\subsubsection*{Rank and Solution Classification}

\begin{conceptbox}
    The number of pivots obtained in row echelon form is called the \textbf{rank} of the matrix.
\end{conceptbox}

Let:
\[
    \text{rank}(A) = r.
\]

\begin{itemize}
    \item If $r = n$ (number of variables) $\Rightarrow$ unique solution.
    \item If $r < n$ and consistent $\Rightarrow$ infinitely many solutions.
    \item If $\text{rank}(A) < \text{rank}([A|b])$ $\Rightarrow$ no solution.
\end{itemize}

%-----------------------------------
\subsubsection*{Engineering Interpretation}

\begin{noteBox}
    In engineering applications:
    \begin{itemize}
        \item Unique solution $\Rightarrow$ well-determined physical system.
        \item Infinite solutions $\Rightarrow$ underdetermined system (degrees of freedom).
        \item No solution $\Rightarrow$ inconsistent measurements or incompatible constraints.
    \end{itemize}
\end{noteBox}