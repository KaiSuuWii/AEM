\subsection{Computing Powers Using Eigenvalues}

If a matrix is diagonalizable, powers can be computed efficiently.

\begin{conceptbox}
    If
    \[
        A = PDP^{-1},
    \]
    then
    \[
        A^n = PD^nP^{-1}.
    \]
\end{conceptbox}

Since $D$ is diagonal:

\[
    D^n =
    \begin{bmatrix}
        \lambda_1^n & 0           \\
        0           & \lambda_2^n
    \end{bmatrix}.
\]

%----------------------------------------

\subsubsection*{Example}

Let

\[
    A =
    \begin{bmatrix}
        4 & 1 \\
        2 & 3
    \end{bmatrix}.
\]

Suppose eigenvalues are:

\[
    \lambda_1 = 5, \quad \lambda_2 = 2.
\]

Then

\[
    A = PDP^{-1}
    \quad \Rightarrow \quad
    A^n = P
    \begin{bmatrix}
        5^n & 0   \\
        0   & 2^n
    \end{bmatrix}
    P^{-1}.
\]

\begin{noteBox}
    This avoids repeated multiplication and is essential for large powers.
\end{noteBox}