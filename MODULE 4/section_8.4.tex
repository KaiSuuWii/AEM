\subsection{Syndrome Detection and Error Correction}

When a codeword is transmitted over a noisy channel,
errors may occur.

Let:

\[
    r = x + e,
\]

where:
\begin{itemize}
    \item $x$ = transmitted codeword
    \item $e$ = error vector
\end{itemize}

%------------------------------------------------
\subsubsection*{Syndrome Definition}

\begin{conceptbox}
    The syndrome is defined as:

    \[
        s = Hr \pmod{2}.
    \]
\end{conceptbox}

If $r$ is a valid codeword, then:

\[
    Hr = 0.
\]

If $Hr \neq 0$, an error occurred.

%------------------------------------------------
\subsubsection*{Example 1: Single-Bit Error Correction}

Suppose the transmitted codeword was:

\[
    x = 0110011.
\]

Assume bit 6 flips during transmission:

\[
    r = 0110001.
\]

Compute syndrome:

\[
    s = Hr.
\]

Using:

\[
    H =
    \begin{bmatrix}
        1 & 0 & 1 & 0 & 1 & 0 & 1 \\
        0 & 1 & 1 & 0 & 0 & 1 & 1 \\
        0 & 0 & 0 & 1 & 1 & 1 & 1
    \end{bmatrix},
    \quad
    r =
    \begin{bmatrix}
        0 \\1\\1\\0\\0\\0\\1
    \end{bmatrix}.
\]

Compute row-by-row (mod 2).

\paragraph{First syndrome bit:}
\[
    s_1 = r_1 + r_3 + r_5 + r_7
    = 0 + 1 + 0 + 1
    = 2 \equiv 0.
\]

\paragraph{Second syndrome bit:}
\[
    s_2 = r_2 + r_3 + r_6 + r_7
    = 1 + 1 + 0 + 1
    = 3 \equiv 1.
\]

\paragraph{Third syndrome bit:}
\[
    s_3 = r_4 + r_5 + r_6 + r_7
    = 0 + 0 + 0 + 1
    = 1.
\]

Thus:

\[
    s =
    \begin{bmatrix}
        0 \\1\\1
    \end{bmatrix}.
\]

Now compare with columns of $H$.

Column 6 of $H$ is:

\[
    \begin{bmatrix}
        0 \\1\\1
    \end{bmatrix}.
\]

\begin{conceptbox}
    The syndrome equals the column of $H$ corresponding
    to the error position.
\end{conceptbox}

Therefore, error occurred in bit 6.

Correct the error by flipping bit 6:

\[
    0110001 \rightarrow 0110011.
\]

Recovered codeword is correct.

%------------------------------------------------
\subsubsection*{Example 2: No Error Case}

If:

\[
    r = 0110011,
\]

then:

\[
    s = Hr = 0.
\]

Thus no detected error.

%------------------------------------------------
\subsubsection*{Example 3: Double Error Case}

Suppose two bits flip:

\[
    r = 0010001.
\]

Compute:

\[
    s = Hr.
\]

The syndrome will be nonzero,
but it will correspond to a single column of $H$,
causing incorrect correction.

\begin{noteBox}
    Hamming (7,4) corrects all single-bit errors,
    but may mis-correct double-bit errors.
\end{noteBox}