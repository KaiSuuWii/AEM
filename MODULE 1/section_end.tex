
\subsection*{1.5 Common Series and Power Series Formulas}

This section summarizes the most frequently used series formulas encountered in
Module 1. These formulas are used as building blocks for constructing Taylor and
Maclaurin series and for analyzing convergence.

\bigskip
\begin{conceptbox}
    \subsubsection*{1. Geometric Series}

    \[
        \sum_{n=0}^{\infty} c_n \quad \text{where} \ c_n = ar^n
    \]

    \begin{itemize}
        \item Converges if $|r| < 1$
        \item Diverges if $|r| \ge 1$
    \end{itemize}

    When convergent,
    \[
        \sum_{n=0}^{\infty} ar^n = \frac{a}{1-r}.
    \]
\end{conceptbox}

\bigskip

\begin{conceptbox}
    \subsubsection*{2. Harmonic Series}

    \[
        \sum_{n=1}^{\infty} \frac{1}{n}
    \]

    This series diverges, even though
    \[
        \lim_{n \to \infty} \frac{1}{n} = 0.
    \]
\end{conceptbox}

\bigskip

\begin{conceptbox}
    \subsubsection*{3. $p$-Series}

    \[
        \sum_{n=1}^{\infty} \frac{1}{n^p}
    \]

    \begin{itemize}
        \item Converges if $p > 1$
        \item Diverges if $p \le 1$
    \end{itemize}
\end{conceptbox}

\bigskip

\begin{conceptbox}
    \subsubsection*{4. Taylor Series Formula}

    If a function $f(x)$ has derivatives of all orders at $x=a$, then its Taylor series
    about $x=a$ is
    \[
        f(x) = \sum_{n=0}^{\infty} \frac{f^{(n)}(a)}{n!}(x-a)^n.
    \]
\end{conceptbox}

\bigskip

\begin{conceptbox}
    \subsubsection*{5. Maclaurin Series Formula}

    The Maclaurin series is the Taylor series centered at $a=0$:
    \[
        f(x) = \sum_{n=0}^{\infty} \frac{f^{(n)}(0)}{n!}x^n.
    \]
\end{conceptbox}

\bigskip

\begin{conceptbox}
    \subsubsection*{6. Common Maclaurin Series}

    \paragraph{$e^x$}
    \[
        e^x = \sum_{n=0}^{\infty} \frac{x^n}{n!},
        \qquad R = \infty
    \]

    \paragraph{$\sin x$}
    \[
        \sin x = \sum_{n=0}^{\infty} (-1)^n \frac{x^{2n+1}}{(2n+1)!},
        \qquad R = \infty
    \]

    \paragraph{$\cos x$}
    \[
        \cos x = \sum_{n=0}^{\infty} (-1)^n \frac{x^{2n}}{(2n)!},
        \qquad R = \infty
    \]

    \paragraph{$\displaystyle \frac{1}{1-x}$}
    \[
        \frac{1}{1-x} = \sum_{n=0}^{\infty} x^n,
        \qquad |x| < 1
    \]

    \paragraph{$\ln(1+x)$}
    \[
        \ln(1+x) = \sum_{n=1}^{\infty} (-1)^{n+1} \frac{x^n}{n},
        \qquad -1 < x \le 1
    \]
\end{conceptbox}

\bigskip

\subsubsection*{7. Using Series Formulas}

\begin{itemize}
    \item Known series may be shifted using $(x-a)$.
    \item Constants may be factored out of a series.
    \item Algebraic substitutions can be used to generate new series.
    \item The interval of convergence must always be adjusted accordingly.
\end{itemize}