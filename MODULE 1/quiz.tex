\subsection*{Quiz: Power Series and Series Expansions}

\subsubsection*{Part I: Convergence of Power Series}

For each power series, determine whether it converges or diverges.
If it converges, find the radius of convergence (RoC) and interval of convergence (IoC).

\begin{enumerate}
    \item $\displaystyle \sum_{n=0}^{\infty} \frac{(x-1)^n}{n!}$

    \item $\displaystyle \sum_{n=1}^{\infty} \frac{(x+2)^n}{n}$

    \item $\displaystyle \sum_{n=0}^{\infty} n!(x-3)^n$

    \item $\displaystyle \sum_{n=1}^{\infty} \frac{(x-2)^n}{\sqrt[n]{n}}$

    \item $\displaystyle \sum_{n=1}^{\infty} \frac{(-1)^n (x-4)^n}{n}$
\end{enumerate}

---

\subsubsection*{Part II: Series Representations}

Find a power series representation for each function.

\begin{enumerate}
    \setcounter{enumi}{5}
    \item $f(x)=\dfrac{1}{1-(x-2)}$

    \item $f(x)=e^{x-1}$

    \item $f(x)=\cos x$
\end{enumerate}

---

\subsubsection*{Solutions}

{\color{red}

\paragraph{1.}
Using the Ratio Test,
\[
    \lim_{n\to\infty}\left|\frac{(x-1)^{n+1}/(n+1)!}{(x-1)^n/n!}\right|
    = \lim_{n\to\infty}\frac{|x-1|}{n+1}=0
\]
Thus, the series converges for all $x$.
\[
    R=\infty, \quad \text{IoC }=(-\infty,\infty)
\]

\begin{center}
    \begin{tikzpicture}
        \begin{axis}[axis lines=middle, xmin=-3,xmax=5,ymin=0,ymax=6]
            \addplot[blue,domain=-3:5,samples=100]{exp(x)};
        \end{axis}
    \end{tikzpicture}
\end{center}

---

\paragraph{2.}
Applying the Ratio Test,
\[
    \lim_{n\to\infty}\left|\frac{(x+2)^{n+1}/(n+1)}{(x+2)^n/n}\right|=|x+2|
\]
So the series converges when $|x+2|<1$.

Endpoint testing:
\[
    x=-3:\sum\frac{(-1)^n}{n} \text{ converges}
\]
\[
    x=-1:\sum\frac{1}{n} \text{ diverges}
\]

\[
    R=1, \quad \text{IoC }=[-3,-1)
\]

\begin{center}
    \begin{tikzpicture}
        \begin{axis}[axis lines=middle, xmin=-4,xmax=0,ymin=-2,ymax=2]
            \addplot[blue,domain=-3: -1]{ln(1+x+2)};
        \end{axis}
    \end{tikzpicture}
\end{center}

---

\paragraph{3.}
Using the Ratio Test,
\[
    \lim_{n\to\infty}\frac{(n+1)!|x-3|^{n+1}}{n!|x-3|^n}
    =(n+1)|x-3|
\]
This limit diverges for all $x\neq3$.
\[
    R=0, \quad \text{IoC }=\{3\}
\]

---

\paragraph{4.}
Applying the Root Test,
\[
    \lim_{n\to\infty}\sqrt[n]{\left|\frac{(x-2)^n}{\sqrt[n]{n}}\right|}
    =|x-2|
\]
So convergence occurs for $|x-2|<1$.

Testing endpoints shows divergence.
\[
    R=1, \quad \text{IoC }=(1,3)
\]

\begin{center}
    \begin{tikzpicture}
        \begin{axis}[axis lines=middle, xmin=0,xmax=4,ymin=-2,ymax=2]
            \addplot[blue,domain=1:3]{x-2};
        \end{axis}
    \end{tikzpicture}
\end{center}

---

\paragraph{5.}
The Ratio Test gives $|x-4|<1$.

Endpoint testing:
\[
    x=3,\; x=5 \Rightarrow \text{alternating harmonic series (convergent)}
\]

\[
    R=1, \quad \text{IoC }=[3,5]
\]

\paragraph{6.}
We begin by recalling the geometric series formula:
\[
    \frac{1}{1-r} = \sum_{n=0}^{\infty} r^n, \quad |r|<1.
\]

The given function is
\[
    f(x) = \frac{1}{1-(x-2)}.
\]

By comparing with the geometric series, we identify
\[
    r = x-2.
\]

Substituting into the formula gives
\[
    \frac{1}{1-(x-2)} = \sum_{n=0}^{\infty} (x-2)^n.
\]

For the series to converge, we must have
\[
    |x-2|<1.
\]

Thus, the power series representation is
\[
    \boxed{
        \frac{1}{1-(x-2)} = \sum_{n=0}^{\infty} (x-2)^n
    }
    \quad \text{for } |x-2|<1.
\]

\paragraph{7.}
The exponential function $e^x$ has the well-known power series expansion
\[
    e^x = \sum_{n=0}^{\infty} \frac{x^n}{n!}.
\]

To represent $e^{x-1}$ as a power series, we substitute $(x-1)$ in place of $x$:
\[
    e^{x-1} = \sum_{n=0}^{\infty} \frac{(x-1)^n}{n!}.
\]

Since the exponential series converges for all real values of its argument, this series converges for every $x$.

Therefore,
\[
    \boxed{
        e^{x-1} = \sum_{n=0}^{\infty} \frac{(x-1)^n}{n!}
    }
    \quad \text{for all } x.
\]

\paragraph{8.}
The cosine function has the Maclaurin series expansion
\[
    \cos x = \sum_{n=0}^{\infty} (-1)^n \frac{x^{2n}}{(2n)!}.
\]

This expansion is obtained by evaluating the derivatives of $\cos x$ at $x=0$:
\[
    \cos(0)=1, \quad \cos'(0)=0, \quad \cos''(0)=-1, \quad \cos^{(4)}(0)=1, \dots
\]

Only even-powered terms appear in the series, and the signs alternate.

Because the factorial in the denominator grows rapidly, the series converges for all real values of $x$.

Hence,
\[
    \boxed{
        \cos x = \sum_{n=0}^{\infty} (-1)^n \frac{x^{2n}}{(2n)!}
    }
    \quad \text{for all } x.
\]

} % end red color
