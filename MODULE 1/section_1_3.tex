\subsection*{1.3 Absolute Convergence}

A series $\sum a_n$ is \textbf{absolutely convergent} if
\[
    \sum |a_n| \text{ converges.}
\]
Absolute convergence guarantees convergence of the original series.

\paragraph{Example 1}
\[
    \sum_{n=1}^{\infty} \frac{(-1)^n}{n^2}
\]
The absolute series
\[
    \sum_{n=1}^{\infty} \frac{1}{n^2}
\]
converges; hence the given series is absolutely convergent.

\paragraph{Example 2}
\[
    \sum_{n=1}^{\infty} (-1)^n \frac{1}{n}
\]
The absolute series diverges, but the original series converges conditionally.

\paragraph{Example 3}
\[
    \sum_{n=1}^{\infty} (-1)^n \frac{n}{n^2+1}
\]
The absolute series behaves like $\sum \frac{1}{n}$ and diverges. The original series converges conditionally.
