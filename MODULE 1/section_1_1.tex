\subsection*{1.1 Power Series and Their Convergence}

\subsubsection*{Definition of a Power Series}

A \textbf{power series} is an infinite series of the form
\[
    \sum_{n=0}^{\infty} c_n (x-x_0)^n
\]
where:
\begin{itemize}
    \item $c_n$ are constant coefficients,
    \item $x_0$ is a fixed real number called the \textbf{center} of the series,
    \item $(x-x_0)^n$ represents powers of the variable $x$.
\end{itemize}

Unlike ordinary numerical series, a power series depends on the value of $x$. As a result, a power series may converge for some values of $x$ and diverge for others.

\bigskip

\subsubsection*{General Behavior of Power Series}

For a given power series
\[
    \sum_{n=0}^{\infty} c_n (x-x_0)^n,
\]
there exists a real number $R \ge 0$, called the \textbf{radius of convergence}, such that:

\[
    \begin{cases}
        \text{The series converges absolutely if } |x-x_0| < R, \\
        \text{The series diverges if } |x-x_0| > R,             \\
        \text{The series may converge or diverge if } |x-x_0| = R.
    \end{cases}
\]

The radius of convergence can be determined from the coefficients of the series through:

\begin{center}
    \begin{tabular}{cc}
        \boxed{(a) \quad R = \dfrac{1}{\displaystyle \lim_{n \to \infty} \sqrt[n]{|c_n|}}} &
        \boxed{(b) \quad R = \dfrac{1}{\displaystyle \lim_{n \to \infty} \left| \dfrac{c_{n+1}}{c_n} \right|}}
    \end{tabular}
\end{center}

provided the limit exists.

The interval
\[
    (x_0 - R,\, x_0 + R)
\]
together with any endpoints where the series converges is called the \textbf{interval of convergence}.

\bigskip

\subsubsection*{Convergence, Divergence, and Absolute Convergence}

Let
\[
    \sum_{n=0}^{\infty} c_n (x-x_0)^n
\]
be a power series.

\begin{itemize}
    \item The series is said to \textbf{converge} at a value $x=b$ if the numerical series
          \[
              \sum_{n=0}^{\infty} c_n (b-x_0)^n
          \]
          converges.

    \item The series \textbf{diverges} at $x=b$ if the corresponding numerical series diverges.

    \item The series is \textbf{absolutely convergent} at $x=b$ if
          \[
              \sum_{n=0}^{\infty} \left| a_n (b-x_0)^n \right|
          \]
          converges.
\end{itemize}

\textbf{Important Result:}
If a power series converges at a point $x=b$, then it converges absolutely for all values of $x$ such that $|x-x_0| < |b-x_0|$.

\bigskip

\subsubsection*{Focus on Convergence of Power Series}

The convergence of a power series depends primarily on the distance of $x$ from the center $x_0$. The farther $x$ is from $x_0$, the more likely the series is to diverge.

To determine where a power series converges, the following steps are followed:

\begin{enumerate}
    \item Apply a convergence test to find the radius of convergence $R$.
    \item Determine the interval $|x-x_0| < R$.
    \item Test the endpoints $x = x_0 \pm R$ separately.
\end{enumerate}

\bigskip

\subsubsection*{Convergence Tests for Power Series}

\paragraph{1. Ratio Test}

Let the power series be
\[
    \sum_{n=0}^{\infty} c_n (x-x_0)^n,
\]
and define
\[
    a_n = c_n (x-x_0)^n.
\]

Compute
\[
    \lim_{n \to \infty} \left| \frac{a_{n+1}}{a_n} \right|.
\]

\begin{itemize}
    \item If the limit is less than 1, the series converges absolutely.
    \item If the limit is greater than 1, the series diverges.
    \item If the limit equals 1, the test is inconclusive.
\end{itemize}

\subsubsection*{Examples: Radius and Interval of Convergence Using the Ratio Test}

\paragraph{Example 1: Power Series of $e^x$}

Consider the power series
\[
    \sum_{n=0}^{\infty} \frac{x^n}{n!}.
\]

Let
\[
    a_n = \frac{x^n}{n!}.
\]

Applying the Ratio Test,
\[
    \lim_{n \to \infty} \left| \frac{a_{n+1}}{a_n} \right|
    = \lim_{n \to \infty} \left| \frac{x^{n+1}}{(n+1)!} \cdot \frac{n!}{x^n} \right|
    = \lim_{n \to \infty} \frac{|x|}{n+1} = 0.
\]

Since the limit is zero for all real values of $x$, the series converges for every $x$.

\[
    \boxed{R = \infty}
\]

Thus, the interval of convergence is
\[
    \boxed{(-\infty, \infty)}.
\]

\bigskip

\paragraph{Example 2: Geometric Series $\displaystyle \frac{1}{1-x}$}

Consider the power series
\[
    \sum_{n=0}^{\infty} x^n.
\]

Let
\[
    a_n = x^n.
\]

Applying the Ratio Test,
\[
    \lim_{n \to \infty} \left| \frac{a_{n+1}}{a_n} \right|
    = \lim_{n \to \infty} \left| \frac{x^{n+1}}{x^n} \right|
    = |x|.
\]

For convergence,
\[
    |x| < 1.
\]

Thus, the radius of convergence is
\[
    \boxed{R = 1}.
\]

\textbf{Endpoint Testing:}

\begin{itemize}
    \item At $x = -1$:
          \[
              \sum_{n=0}^{\infty} (-1)^n \quad \text{diverges}.
          \]

    \item At $x = 1$:
          \[
              \sum_{n=0}^{\infty} 1 \quad \text{diverges}.
          \]
\end{itemize}

Hence, the interval of convergence is
\[
    \boxed{(-1,\,1)}.
\]

\bigskip

\paragraph{Example 3: Series Involving Factorials}

Consider the power series
\[
    \sum_{n=0}^{\infty} n! \, x^n.
\]

Let
\[
    a_n = n! \, x^n.
\]

Applying the Ratio Test,
\[
    \lim_{n \to \infty} \left| \frac{a_{n+1}}{a_n} \right|
    = \lim_{n \to \infty} \left| \frac{(n+1)! x^{n+1}}{n! x^n} \right|
    = \lim_{n \to \infty} (n+1)|x|.
\]

For convergence,
\[
    (n+1)|x| < 1.
\]

As $n \to \infty$, this inequality holds only when $x = 0$.

\[
    \boxed{R = 0}.
\]

Thus, the series converges only at $x = 0$, and the interval of convergence is
\[
    \boxed{\{0\}}.
\]

\bigskip

\textbf{Summary of Results:}

\begin{center}
    \begin{tabular}{|c|c|c|}
        \hline
        \textbf{Series}                     & \textbf{Radius of Convergence} & \textbf{Interval of Convergence} \\
        \hline
        $\displaystyle \sum \frac{x^n}{n!}$ & $R = \infty$                   & $(-\infty, \infty)$              \\
        \hline
        $\displaystyle \sum x^n$            & $R = 1$                        & $(-1, 1)$                        \\
        \hline
        $\displaystyle \sum n! x^n$         & $R = 0$                        & $\{0\}$                          \\
        \hline
    \end{tabular}
\end{center}


\paragraph{2. Root Test}

The \textbf{Root Test} may also be applied:
\[
    \lim_{n \to \infty} \sqrt[n]{|a_n|}.
\]

The conclusions are the same as those of the Ratio Test.

\paragraph{3. Endpoint Testing}

Once the radius of convergence $R$ is found, the series must be tested at
\[
    x = a - R \quad \text{and} \quad x = a + R.
\]

At these endpoints, the power series reduces to a numerical series. Common tests used include:
\begin{itemize}
    \item $p$-series test
    \item Alternating series test
    \item Comparison test
\end{itemize}
