\subsection*{1.1 Power Series and Their Convergence}

\subsubsection*{Definition of a Power Series}

A \textbf{power series} is an infinite series of the form
\[
    \sum_{n=0}^{\infty} c_n (x-x_0)^n
\]
where:
\begin{itemize}
    \item $c_n$ are constant coefficients,
    \item $x_0$ is a fixed real number called the \textbf{center} of the series,
    \item $(x-x_0)^n$ represents powers of the variable $x$.
\end{itemize}

Unlike ordinary numerical series, a power series depends on the value of $x$. As a result, a power series may converge for some values of $x$ and diverge for others.

\bigskip

\subsubsection*{General Behavior of Power Series}

For a given power series
\[
    \sum_{n=0}^{\infty} c_n (x-x_0)^n,
\]
there exists a real number $R \ge 0$, called the \textbf{radius of convergence}, such that:

\[
    \begin{cases}
        \text{The series converges absolutely if } |x-x_0| < R, \\
        \text{The series diverges if } |x-x_0| > R,             \\
        \text{The series may converge or diverge if } |x-x_0| = R.
    \end{cases}
\]

The radius of convergence can be determined from the coefficients of the series through:

\begin{center}
    \begin{tabular}{cc}
        \boxed{(a) \quad R = \dfrac{1}{\displaystyle \lim_{n \to \infty} \sqrt[n]{|c_n|}}} &
        \boxed{(b) \quad R = \dfrac{1}{\displaystyle \lim_{n \to \infty} \left| \dfrac{c_{n+1}}{c_n} \right|}}
    \end{tabular}
\end{center}

provided the limit exists.

\bigskip

The interval
\[
    (x_0 - R,\, x_0 + R)
\]
together with any endpoints where the series converges is called the \textbf{interval of convergence}.

\bigskip

\subsubsection*{Convergence, Divergence, and Absolute Convergence}

Let
\[
    \sum_{n=0}^{\infty} c_n (x-x_0)^n
\]
be a power series.

\begin{itemize}
    \item The series is said to \textbf{converge} at a value $x=b$ if the numerical series
          \[
              \sum_{n=0}^{\infty} c_n (b-x_0)^n
          \]
          converges.

    \item The series \textbf{diverges} at $x=b$ if the corresponding numerical series diverges.

    \item The series is \textbf{absolutely convergent} at $x=b$ if
          \[
              \sum_{n=0}^{\infty} \left| c_n (b-x_0)^n \right|
          \]
          converges.
\end{itemize}

\textbf{Important Result:}
If a power series converges at a point $x=b$, then it converges absolutely for all values of $x$ such that $|x-x_0| < |b-x_0|$.

\bigskip

\subsubsection*{Focus on Convergence of Power Series}

The convergence of a power series depends primarily on the distance of $x$ from the center $x_0$. The farther $x$ is from $x_0$, the more likely the series is to diverge.

To determine where a power series converges, the following steps are followed:

\begin{enumerate}
    \item Apply a convergence test to find the radius of convergence $R$.
    \item Determine the interval $|x-x_0| < R$.
    \item Test the endpoints $x = x_0 \pm R$ separately.
\end{enumerate}

\bigskip

\subsubsection*{Convergence Tests for Power Series}

\paragraph{1. Ratio Test}

Let the power series be
\[
    \sum_{n=0}^{\infty} c_n (x-x_0)^n,
\]
and define
\[
    a_n = c_n (x-x_0)^n.
\]

Compute
\[
    \lim_{n \to \infty} \left| \frac{a_{n+1}}{a_n} \right|.
\]

\begin{itemize}
    \item If the limit is less than 1, the series converges absolutely.
    \item If the limit is greater than 1, the series diverges.
    \item If the limit equals 1, the test is inconclusive.
\end{itemize}

\subsubsection*{Examples: Radius and Interval of Convergence Using the Ratio Test}

\begin{examplebox}
    \paragraph{Example 1: Power Series of $e^x$}

    Consider the power series
    \[
        \sum_{n=0}^{\infty} \frac{x^n}{n!}.
    \]

    Let
    \[
        a_n = \frac{x^n}{n!}.
    \]

    Applying the Ratio Test,
    \[
        \lim_{n \to \infty} \left| \frac{a_{n+1}}{a_n} \right|
        = \lim_{n \to \infty} \left| \frac{x^{n+1}}{(n+1)!} \cdot \frac{n!}{x^n} \right|
        = \lim_{n \to \infty} \frac{|x|}{n+1} = 0.
    \]

    Since the limit is zero for all real values of $x$, the series converges for every $x$.

    \[
        \boxed{R = \infty}
    \]

    Thus, the interval of convergence is
    \[
        \boxed{(-\infty, \infty)}.
    \]
\end{examplebox}

\bigskip

\begin{examplebox}
    \paragraph{Example 2: Geometric Series $\displaystyle \frac{1}{1-x}$}

    Consider the power series
    \[
        \sum_{n=0}^{\infty} x^n.
    \]

    Let
    \[
        a_n = x^n.
    \]

    Applying the Ratio Test,
    \[
        \lim_{n \to \infty} \left| \frac{a_{n+1}}{a_n} \right|
        = \lim_{n \to \infty} \left| \frac{x^{n+1}}{x^n} \right|
        = |x|.
    \]

    For convergence,
    \[
        |x| < 1.
    \]

    Thus, the radius of convergence is
    \[
        \boxed{R = 1}.
    \]

    \textbf{Endpoint Testing:}

    \begin{itemize}
        \item At $x = -1$:
              \[
                  \sum_{n=0}^{\infty} (-1)^n \quad \text{diverges}.
              \]

        \item At $x = 1$:
              \[
                  \sum_{n=0}^{\infty} 1 \quad \text{diverges}.
              \]
    \end{itemize}

    Hence, the interval of convergence is
    \[
        \boxed{(-1,\,1)}.
    \]
\end{examplebox}

\bigskip

\begin{examplebox}
    \paragraph{Example 3: Series Involving Factorials}

    Consider the power series
    \[
        \sum_{n=0}^{\infty} n! \, x^n.
    \]

    Let
    \[
        a_n = n! \, x^n.
    \]

    Applying the Ratio Test,
    \[
        \lim_{n \to \infty} \left| \frac{a_{n+1}}{a_n} \right|
        = \lim_{n \to \infty} \left| \frac{(n+1)! x^{n+1}}{n! x^n} \right|
        = \lim_{n \to \infty} (n+1)|x|.
    \]

    For convergence,
    \[
        (n+1)|x| < 1.
    \]

    As $n \to \infty$, this inequality holds only when $x = 0$.

    \[
        \boxed{R = 0}.
    \]

    Thus, the series converges only at $x = 0$, and the interval of convergence is
    \[
        \boxed{\{0\}}.
    \]
\end{examplebox}

\bigskip

\textbf{Summary of Results:}

\begin{center}
    \begin{tabular}{|c|c|c|}
        \hline
        \textbf{Series}                     & \textbf{Radius of Convergence} & \textbf{Interval of Convergence} \\
        \hline
        $\displaystyle \sum \frac{x^n}{n!}$ & $R = \infty$                   & $(-\infty, \infty)$              \\
        \hline
        $\displaystyle \sum x^n$            & $R = 1$                        & $(-1, 1)$                        \\
        \hline
        $\displaystyle \sum n! x^n$         & $R = 0$                        & $\{0\}$                          \\
        \hline
    \end{tabular}
\end{center}


\paragraph{2. Root Test}

The \textbf{Root Test} may also be applied:
\[
    \lim_{n \to \infty} \sqrt[n]{|a_n|}.
\]

The conclusions are the same as those of the Ratio Test.

\begin{examplebox}
    \paragraph{Example 4: Power Series with Exponential Coefficients}

    Consider the power series
    \[
        \sum_{n=0}^{\infty} \frac{(2x)^n}{n}.
    \]

    Let
    \[
        a_n = \frac{(2x)^n}{n}.
    \]

    Apply the Root Test:
    \[
        \lim_{n \to \infty} \sqrt[n]{|a_n|}
        = \lim_{n \to \infty} \sqrt[n]{\frac{|2x|^n}{n}}
        = |2x| \cdot \lim_{n \to \infty} \sqrt[n]{\frac{1}{n}}.
    \]

    Since
    \[
        \lim_{n \to \infty} \sqrt[n]{\frac{1}{n}} = 1,
    \]
    we obtain
    \[
        \lim_{n \to \infty} \sqrt[n]{|a_n|} = |2x|.
    \]

    For convergence,
    \[
        |2x| < 1 \quad \Rightarrow \quad |x| < \frac{1}{2}.
    \]

    Thus, the radius of convergence is
    \[
        \boxed{R = \frac{1}{2}}.
    \]

    \textbf{Endpoint Testing:}

    \begin{itemize}
        \item At $x = \frac{1}{2}$:
              \[
                  \sum_{n=0}^{\infty} \frac{1}{n} \quad \text{diverges}.
              \]

        \item At $x = -\frac{1}{2}$:
              \[
                  \sum_{n=0}^{\infty} \frac{(-1)^n}{n} \quad \text{converges (alternating series)}.
              \]
    \end{itemize}

    Hence, the interval of convergence is
    \[
        \boxed{\left[-\frac{1}{2},\, \frac{1}{2}\right)}.
    \]
\end{examplebox}

\bigskip

\begin{examplebox}
    \paragraph{Example 5: Power Series with Polynomial Growth}

    Consider the power series
    \[
        \sum_{n=0}^{\infty} n^2 x^n.
    \]

    Let
    \[
        a_n = n^2 x^n.
    \]

    Apply the Root Test:
    \[
        \lim_{n \to \infty} \sqrt[n]{|a_n|}
        = \lim_{n \to \infty} \sqrt[n]{n^2 |x|^n}
        = |x| \cdot \lim_{n \to \infty} \sqrt[n]{n^2}.
    \]

    Since
    \[
        \lim_{n \to \infty} \sqrt[n]{n^2} = 1,
    \]
    we have
    \[
        \lim_{n \to \infty} \sqrt[n]{|a_n|} = |x|.
    \]

    For convergence,
    \[
        |x| < 1.
    \]

    Thus, the radius of convergence is
    \[
        \boxed{R = 1}.
    \]

    \textbf{Endpoint Testing:}

    \begin{itemize}
        \item At $x = 1$:
              \[
                  \sum_{n=0}^{\infty} n^2 \quad \text{diverges}.
              \]

        \item At $x = -1$:
              \[
                  \sum_{n=0}^{\infty} (-1)^n n^2 \quad \text{diverges}.
              \]
    \end{itemize}

    Hence, the interval of convergence is
    \[
        \boxed{(-1,\,1)}.
    \]
\end{examplebox}

\bigskip

\begin{examplebox}
    \paragraph{Example 6: Power Series with Factorials in the Denominator}

    Consider the power series
    \[
        \sum_{n=0}^{\infty} \frac{x^n}{(n!)^2}.
    \]

    Let
    \[
        a_n = \frac{x^n}{(n!)^2}.
    \]

    Apply the Root Test:
    \[
        \lim_{n \to \infty} \sqrt[n]{|a_n|}
        = \lim_{n \to \infty} \frac{|x|}{(n!)^{2/n}}.
    \]

    Since $(n!)^{1/n} \to \infty$ as $n \to \infty$, it follows that
    \[
        \lim_{n \to \infty} \sqrt[n]{|a_n|} = 0
    \]
    for all real $x$.

    Therefore, the series converges for all $x$, and
    \[
        \boxed{R = \infty}.
    \]

    The interval of convergence is
    \[
        \boxed{(-\infty, \infty)}.
    \]
\end{examplebox}

\subsubsection*{Endpoint Testing: Common Tests and When to Use Them}

After finding the radius of convergence $R$, the power series must be tested at the
endpoints
\[
    x = x_0 - R \quad \text{and} \quad x = x_0 + R.
\]

At each endpoint, the power series becomes a numerical series. Since the Ratio and Root
Tests are inconclusive at endpoints, other convergence tests must be used.

\bigskip

\paragraph{1. $p$-Series Test}

A series of the form
\[
    \sum_{n=1}^{\infty} \frac{1}{n^p}
\]
is called a \textbf{$p$-series}.

\begin{itemize}
    \item The series converges if $p > 1$.
    \item The series diverges if $p \le 1$.
\end{itemize}

\textbf{When to use:}
This test is used when the endpoint series simplifies to a rational expression involving
powers of $n$, such as
\[
    \sum \frac{1}{n^2}, \quad \sum \frac{1}{\sqrt{n}}, \quad \sum \frac{1}{n}.
\]

\textbf{Example:}
\[
    \sum_{n=1}^{\infty} \frac{1}{n^2} \quad \text{converges} \qquad (p = 2 > 1),
\]
\[
    \sum_{n=1}^{\infty} \frac{1}{n} \quad \text{diverges} \qquad (p = 1).
\]

\bigskip

\paragraph{2. Alternating Series Test}

An alternating series has the form
\[
    \sum_{n=1}^{\infty} (-1)^n b_n \quad \text{or} \quad \sum_{n=1}^{\infty} (-1)^{n+1} b_n,
\]
where $b_n > 0$ for all $n$.

The series converges if:
\begin{enumerate}
    \item $b_n$ is decreasing, and
    \item $\lim_{n \to \infty} b_n = 0$.
\end{enumerate}

\textbf{When to use:}
This test is used when substituting an endpoint produces alternating signs, typically
from terms such as $(-1)^n$.

\textbf{Important Note:}
An alternating series may converge even if it does not converge absolutely.

\textbf{Example:}
\[
    \sum_{n=1}^{\infty} \frac{(-1)^n}{n} \quad \text{converges (conditionally)}.
\]

\bigskip

\paragraph{3. Comparison Test}

Let $\sum a_n$ and $\sum b_n$ be series with $0 \le a_n \le b_n$ for all sufficiently large $n$.

\begin{itemize}
    \item If $\sum b_n$ converges, then $\sum a_n$ converges.
    \item If $\sum a_n$ diverges, then $\sum b_n$ diverges.
\end{itemize}

\textbf{When to use:}
This test is used when the endpoint series resembles a known series (such as a $p$-series)
but does not match it exactly.

\textbf{Example:}
\[
    \sum_{n=1}^{\infty} \frac{1}{n^2 + 1}
    \quad \text{converges by comparison with } \sum \frac{1}{n^2}.
\]

\bigskip

\paragraph{Choosing the Appropriate Test}

At an endpoint:
\begin{itemize}
    \item If the series resembles $\dfrac{1}{n^p}$, use the \textbf{$p$-series test}.
    \item If the series alternates in sign, try the \textbf{Alternating Series Test} first.
    \item If the series resembles a known convergent or divergent series but is not exact,
          use the \textbf{Comparison Test}.
\end{itemize}

Each endpoint must be tested \textbf{independently}. A power series may converge at one
endpoint and diverge at the other.
