\documentclass[12pt]{article}

% Packages
\usepackage{amsmath, amssymb, amsfonts}
\usepackage{graphicx}
\usepackage{hyperref}
\usepackage{geometry}
\usepackage{setspace}
\usepackage{enumitem}
\usepackage{physics}
\usepackage{tcolorbox}

\geometry{margin=1in}
\setstretch{1.15}

% Custom boxes for notes
\tcbset{
  colback=gray!5,
  colframe=black!70,
  arc=3mm,
  boxrule=0.8pt
}

\newtcolorbox{conceptbox}{title=Key Concept}
\newtcolorbox{examplebox}{title=Example}
\newtcolorbox{noteBox}{title=Engineering Note}

\title{Advanced Engineering Mathematics \\ \large Series}
\author{\textbf{Engr. Kaiveni Tom Dagcuta} \\ Computer Engineering}
\date{}

\begin{document}

\maketitle

\section*{Module 1: Infinite Series}

This lecture introduces the fundamental theory of infinite series as used in Advanced Engineering Mathematics. Emphasis is placed on convergence concepts and power series representations, which are essential in engineering analysis, modeling, and later topics such as differential equations and signal processing.

% Input subsection files
\subsection*{Series}

A \textbf{series} is the sum of the terms of a sequence. If $\{a_n\}$ is a sequence, then the corresponding series is written as
\[
    \sum_{n=1}^{\infty} a_n = a_1 + a_2 + a_3 + \cdots
\]

The value of a series is determined by the behavior of its \textbf{partial sums}
\[
    S_N = \sum_{n=1}^{N} a_n.
\]

\begin{itemize}
    \item The series \textbf{converges} if $\displaystyle \lim_{N \to \infty} S_N$ exists and is finite.
    \item The series \textbf{diverges} if this limit does not exist or is infinite.
\end{itemize}

\textbf{Necessary Condition for Convergence:}
\[
    \lim_{n \to \infty} a_n = 0.
\]
If this condition is not satisfied, the series must diverge.


\bigskip

\subsection*{1.1 Power Series and Their Convergence}

\subsubsection*{Definition of a Power Series}

A \textbf{power series} is an infinite series of the form
\[
    \sum_{n=0}^{\infty} c_n (x-x_0)^n
\]
where:
\begin{itemize}
    \item $c_n$ are constant coefficients,
    \item $x_0$ is a fixed real number called the \textbf{center} of the series,
    \item $(x-x_0)^n$ represents powers of the variable $x$.
\end{itemize}

Unlike ordinary numerical series, a power series depends on the value of $x$. As a result, a power series may converge for some values of $x$ and diverge for others.

\bigskip

\subsubsection*{General Behavior of Power Series}

For a given power series
\[
    \sum_{n=0}^{\infty} c_n (x-x_0)^n,
\]
there exists a real number $R \ge 0$, called the \textbf{radius of convergence}, such that:

\[
    \begin{cases}
        \text{The series converges absolutely if } |x-x_0| < R, \\
        \text{The series diverges if } |x-x_0| > R,             \\
        \text{The series may converge or diverge if } |x-x_0| = R.
    \end{cases}
\]

The radius of convergence can be determined from the coefficients of the series through:

\begin{center}
    \begin{tabular}{cc}
        \boxed{(a) \quad R = \dfrac{1}{\displaystyle \lim_{n \to \infty} \sqrt[n]{|c_n|}}} &
        \boxed{(b) \quad R = \dfrac{1}{\displaystyle \lim_{n \to \infty} \left| \dfrac{c_{n+1}}{c_n} \right|}}
    \end{tabular}
\end{center}

provided the limit exists.

\bigskip

The interval
\[
    (x_0 - R,\, x_0 + R)
\]
together with any endpoints where the series converges is called the \textbf{interval of convergence}.

\bigskip

\subsubsection*{Convergence, Divergence, and Absolute Convergence}

Let
\[
    \sum_{n=0}^{\infty} c_n (x-x_0)^n
\]
be a power series.

\begin{itemize}
    \item The series is said to \textbf{converge} at a value $x=b$ if the numerical series
          \[
              \sum_{n=0}^{\infty} c_n (b-x_0)^n
          \]
          converges.

    \item The series \textbf{diverges} at $x=b$ if the corresponding numerical series diverges.

    \item The series is \textbf{absolutely convergent} at $x=b$ if
          \[
              \sum_{n=0}^{\infty} \left| c_n (b-x_0)^n \right|
          \]
          converges.
\end{itemize}

\textbf{Important Result:}
If a power series converges at a point $x=b$, then it converges absolutely for all values of $x$ such that $|x-x_0| < |b-x_0|$.

\bigskip

\subsubsection*{Focus on Convergence of Power Series}

The convergence of a power series depends primarily on the distance of $x$ from the center $x_0$. The farther $x$ is from $x_0$, the more likely the series is to diverge.

To determine where a power series converges, the following steps are followed:

\begin{enumerate}
    \item Apply a convergence test to find the radius of convergence $R$.
    \item Determine the interval $|x-x_0| < R$.
    \item Test the endpoints $x = x_0 \pm R$ separately.
\end{enumerate}

\bigskip

\subsubsection*{Convergence Tests for Power Series}

\paragraph{1. Ratio Test}

Let the power series be
\[
    \sum_{n=0}^{\infty} c_n (x-x_0)^n,
\]
and define
\[
    a_n = c_n (x-x_0)^n.
\]

Compute
\[
    \lim_{n \to \infty} \left| \frac{a_{n+1}}{a_n} \right|.
\]

\begin{itemize}
    \item If the limit is less than 1, the series converges absolutely.
    \item If the limit is greater than 1, the series diverges.
    \item If the limit equals 1, the test is inconclusive.
\end{itemize}

\subsubsection*{Examples: Radius and Interval of Convergence Using the Ratio Test}

\begin{examplebox}
    \paragraph{Example 1: Power Series of $e^x$}

    Consider the power series
    \[
        \sum_{n=0}^{\infty} \frac{x^n}{n!}.
    \]

    Let
    \[
        a_n = \frac{x^n}{n!}.
    \]

    Applying the Ratio Test,
    \[
        \lim_{n \to \infty} \left| \frac{a_{n+1}}{a_n} \right|
        = \lim_{n \to \infty} \left| \frac{x^{n+1}}{(n+1)!} \cdot \frac{n!}{x^n} \right|
        = \lim_{n \to \infty} \frac{|x|}{n+1} = 0.
    \]

    Since the limit is zero for all real values of $x$, the series converges for every $x$.

    \[
        \boxed{R = \infty}
    \]

    Thus, the interval of convergence is
    \[
        \boxed{(-\infty, \infty)}.
    \]
\end{examplebox}

\bigskip

\begin{examplebox}
    \paragraph{Example 2: Geometric Series $\displaystyle \frac{1}{1-x}$}

    Consider the power series
    \[
        \sum_{n=0}^{\infty} x^n.
    \]

    Let
    \[
        a_n = x^n.
    \]

    Applying the Ratio Test,
    \[
        \lim_{n \to \infty} \left| \frac{a_{n+1}}{a_n} \right|
        = \lim_{n \to \infty} \left| \frac{x^{n+1}}{x^n} \right|
        = |x|.
    \]

    For convergence,
    \[
        |x| < 1.
    \]

    Thus, the radius of convergence is
    \[
        \boxed{R = 1}.
    \]

    \textbf{Endpoint Testing:}

    \begin{itemize}
        \item At $x = -1$:
              \[
                  \sum_{n=0}^{\infty} (-1)^n \quad \text{diverges}.
              \]

        \item At $x = 1$:
              \[
                  \sum_{n=0}^{\infty} 1 \quad \text{diverges}.
              \]
    \end{itemize}

    Hence, the interval of convergence is
    \[
        \boxed{(-1,\,1)}.
    \]
\end{examplebox}

\bigskip

\begin{examplebox}
    \paragraph{Example 3: Series Involving Factorials}

    Consider the power series
    \[
        \sum_{n=0}^{\infty} n! \, x^n.
    \]

    Let
    \[
        a_n = n! \, x^n.
    \]

    Applying the Ratio Test,
    \[
        \lim_{n \to \infty} \left| \frac{a_{n+1}}{a_n} \right|
        = \lim_{n \to \infty} \left| \frac{(n+1)! x^{n+1}}{n! x^n} \right|
        = \lim_{n \to \infty} (n+1)|x|.
    \]

    For convergence,
    \[
        (n+1)|x| < 1.
    \]

    As $n \to \infty$, this inequality holds only when $x = 0$.

    \[
        \boxed{R = 0}.
    \]

    Thus, the series converges only at $x = 0$, and the interval of convergence is
    \[
        \boxed{\{0\}}.
    \]
\end{examplebox}

\bigskip

\textbf{Summary of Results:}

\begin{center}
    \begin{tabular}{|c|c|c|}
        \hline
        \textbf{Series}                     & \textbf{Radius of Convergence} & \textbf{Interval of Convergence} \\
        \hline
        $\displaystyle \sum \frac{x^n}{n!}$ & $R = \infty$                   & $(-\infty, \infty)$              \\
        \hline
        $\displaystyle \sum x^n$            & $R = 1$                        & $(-1, 1)$                        \\
        \hline
        $\displaystyle \sum n! x^n$         & $R = 0$                        & $\{0\}$                          \\
        \hline
    \end{tabular}
\end{center}


\paragraph{2. Root Test}

The \textbf{Root Test} may also be applied:
\[
    \lim_{n \to \infty} \sqrt[n]{|a_n|}.
\]

The conclusions are the same as those of the Ratio Test.

\begin{examplebox}
    \paragraph{Example 4: Power Series with Exponential Coefficients}

    Consider the power series
    \[
        \sum_{n=0}^{\infty} \frac{(2x)^n}{n}.
    \]

    Let
    \[
        a_n = \frac{(2x)^n}{n}.
    \]

    Apply the Root Test:
    \[
        \lim_{n \to \infty} \sqrt[n]{|a_n|}
        = \lim_{n \to \infty} \sqrt[n]{\frac{|2x|^n}{n}}
        = |2x| \cdot \lim_{n \to \infty} \sqrt[n]{\frac{1}{n}}.
    \]

    Since
    \[
        \lim_{n \to \infty} \sqrt[n]{\frac{1}{n}} = 1,
    \]
    we obtain
    \[
        \lim_{n \to \infty} \sqrt[n]{|a_n|} = |2x|.
    \]

    For convergence,
    \[
        |2x| < 1 \quad \Rightarrow \quad |x| < \frac{1}{2}.
    \]

    Thus, the radius of convergence is
    \[
        \boxed{R = \frac{1}{2}}.
    \]

    \textbf{Endpoint Testing:}

    \begin{itemize}
        \item At $x = \frac{1}{2}$:
              \[
                  \sum_{n=0}^{\infty} \frac{1}{n} \quad \text{diverges}.
              \]

        \item At $x = -\frac{1}{2}$:
              \[
                  \sum_{n=0}^{\infty} \frac{(-1)^n}{n} \quad \text{converges (alternating series)}.
              \]
    \end{itemize}

    Hence, the interval of convergence is
    \[
        \boxed{\left[-\frac{1}{2},\, \frac{1}{2}\right)}.
    \]
\end{examplebox}

\bigskip

\begin{examplebox}
    \paragraph{Example 5: Power Series with Polynomial Growth}

    Consider the power series
    \[
        \sum_{n=0}^{\infty} n^2 x^n.
    \]

    Let
    \[
        a_n = n^2 x^n.
    \]

    Apply the Root Test:
    \[
        \lim_{n \to \infty} \sqrt[n]{|a_n|}
        = \lim_{n \to \infty} \sqrt[n]{n^2 |x|^n}
        = |x| \cdot \lim_{n \to \infty} \sqrt[n]{n^2}.
    \]

    Since
    \[
        \lim_{n \to \infty} \sqrt[n]{n^2} = 1,
    \]
    we have
    \[
        \lim_{n \to \infty} \sqrt[n]{|a_n|} = |x|.
    \]

    For convergence,
    \[
        |x| < 1.
    \]

    Thus, the radius of convergence is
    \[
        \boxed{R = 1}.
    \]

    \textbf{Endpoint Testing:}

    \begin{itemize}
        \item At $x = 1$:
              \[
                  \sum_{n=0}^{\infty} n^2 \quad \text{diverges}.
              \]

        \item At $x = -1$:
              \[
                  \sum_{n=0}^{\infty} (-1)^n n^2 \quad \text{diverges}.
              \]
    \end{itemize}

    Hence, the interval of convergence is
    \[
        \boxed{(-1,\,1)}.
    \]
\end{examplebox}

\bigskip

\begin{examplebox}
    \paragraph{Example 6: Power Series with Factorials in the Denominator}

    Consider the power series
    \[
        \sum_{n=0}^{\infty} \frac{x^n}{(n!)^2}.
    \]

    Let
    \[
        a_n = \frac{x^n}{(n!)^2}.
    \]

    Apply the Root Test:
    \[
        \lim_{n \to \infty} \sqrt[n]{|a_n|}
        = \lim_{n \to \infty} \frac{|x|}{(n!)^{2/n}}.
    \]

    Since $(n!)^{1/n} \to \infty$ as $n \to \infty$, it follows that
    \[
        \lim_{n \to \infty} \sqrt[n]{|a_n|} = 0
    \]
    for all real $x$.

    Therefore, the series converges for all $x$, and
    \[
        \boxed{R = \infty}.
    \]

    The interval of convergence is
    \[
        \boxed{(-\infty, \infty)}.
    \]
\end{examplebox}

\subsubsection*{Endpoint Testing: Common Tests and When to Use Them}

After finding the radius of convergence $R$, the power series must be tested at the
endpoints
\[
    x = x_0 - R \quad \text{and} \quad x = x_0 + R.
\]

At each endpoint, the power series becomes a numerical series. Since the Ratio and Root
Tests are inconclusive at endpoints, other convergence tests must be used.

\bigskip

\paragraph{1. $p$-Series Test}

A series of the form
\[
    \sum_{n=1}^{\infty} \frac{1}{n^p}
\]
is called a \textbf{$p$-series}.

\begin{itemize}
    \item The series converges if $p > 1$.
    \item The series diverges if $p \le 1$.
\end{itemize}

\textbf{When to use:}
This test is used when the endpoint series simplifies to a rational expression involving
powers of $n$, such as
\[
    \sum \frac{1}{n^2}, \quad \sum \frac{1}{\sqrt{n}}, \quad \sum \frac{1}{n}.
\]

\textbf{Example:}
\[
    \sum_{n=1}^{\infty} \frac{1}{n^2} \quad \text{converges} \qquad (p = 2 > 1),
\]
\[
    \sum_{n=1}^{\infty} \frac{1}{n} \quad \text{diverges} \qquad (p = 1).
\]

\bigskip

\paragraph{2. Alternating Series Test}

An alternating series has the form
\[
    \sum_{n=1}^{\infty} (-1)^n b_n \quad \text{or} \quad \sum_{n=1}^{\infty} (-1)^{n+1} b_n,
\]
where $b_n > 0$ for all $n$.

The series converges if:
\begin{enumerate}
    \item $b_n$ is decreasing, and
    \item $\lim_{n \to \infty} b_n = 0$.
\end{enumerate}

\textbf{When to use:}
This test is used when substituting an endpoint produces alternating signs, typically
from terms such as $(-1)^n$.

\textbf{Important Note:}
An alternating series may converge even if it does not converge absolutely.

\textbf{Example:}
\[
    \sum_{n=1}^{\infty} \frac{(-1)^n}{n} \quad \text{converges (conditionally)}.
\]

\bigskip

\paragraph{3. Comparison Test}

Let $\sum a_n$ and $\sum b_n$ be series with $0 \le a_n \le b_n$ for all sufficiently large $n$.

\begin{itemize}
    \item If $\sum b_n$ converges, then $\sum a_n$ converges.
    \item If $\sum a_n$ diverges, then $\sum b_n$ diverges.
\end{itemize}

\textbf{When to use:}
This test is used when the endpoint series resembles a known series (such as a $p$-series)
but does not match it exactly.

\textbf{Example:}
\[
    \sum_{n=1}^{\infty} \frac{1}{n^2 + 1}
    \quad \text{converges by comparison with } \sum \frac{1}{n^2}.
\]

\bigskip

\paragraph{Choosing the Appropriate Test}

At an endpoint:
\begin{itemize}
    \item If the series resembles $\dfrac{1}{n^p}$, use the \textbf{$p$-series test}.
    \item If the series alternates in sign, try the \textbf{Alternating Series Test} first.
    \item If the series resembles a known convergent or divergent series but is not exact,
          use the \textbf{Comparison Test}.
\end{itemize}

Each endpoint must be tested \textbf{independently}. A power series may converge at one
endpoint and diverge at the other.


\bigskip

\subsection*{1.2 Divergence of Series}

A series $\sum a_n$ is said to \textbf{diverge} if the sequence of partial sums
\[
    S_n = a_1 + a_2 + \cdots + a_n
\]
does not approach a finite limit as $n \to \infty$.

\bigskip

\paragraph{Important Observation}
A necessary condition for convergence is
\[
    \lim_{n \to \infty} a_n = 0.
\]
If this limit does not exist or is not zero, the series \textbf{must diverge}.
However, this condition alone does not guarantee convergence.

\bigskip

\paragraph{Key Divergence Tests}
\begin{itemize}
    \item $n$th-Term Test for Divergence
    \item Harmonic Series
    \item Comparison with Known Divergent Series
\end{itemize}

\bigskip

\subsubsection*{1. $n$th-Term Test for Divergence}

If
\[
    \lim_{n \to \infty} a_n \neq 0 \quad \text{or does not exist},
\]
then the series
\[
    \sum_{n=1}^{\infty} a_n
\]
\textbf{diverges}.

\textbf{Important Note:}
If $\lim_{n \to \infty} a_n = 0$, the test is inconclusive.

\bigskip

\paragraph{Example 1: $n$th-Term Test}

\[
    \sum_{n=1}^{\infty} \frac{n}{n+1}
\]

Since
\[
    \lim_{n \to \infty} \frac{n}{n+1} = 1 \neq 0,
\]
the series diverges by the $n$th-term test.

\bigskip

\paragraph{Example 2: Oscillating Terms}

\[
    \sum_{n=1}^{\infty} (-1)^n
\]

Here, the sequence $a_n = (-1)^n$ does not approach a limit. Therefore, the
series diverges by the $n$th-term test.

\bigskip

\subsubsection*{2. Harmonic Series}

The \textbf{harmonic series} is defined as
\[
    \sum_{n=1}^{\infty} \frac{1}{n}.
\]

Although
\[
    \lim_{n \to \infty} \frac{1}{n} = 0,
\]
the harmonic series \textbf{diverges}.

This example shows that the condition $\lim_{n \to \infty} a_n = 0$ is necessary
but not sufficient for convergence.

\bigskip

\paragraph{Example 3: Constant Multiple of the Harmonic Series}

\[
    \sum_{n=1}^{\infty} \frac{5}{n}
\]

Since this is a constant multiple of the harmonic series, it also diverges.

\bigskip

\subsubsection*{3. Comparison with Divergent Series}

If $0 \le a_n \le b_n$ for all sufficiently large $n$ and
\[
    \sum b_n \quad \text{diverges},
\]
then
\[
    \sum a_n \quad \text{also diverges}.
\]

\bigskip

\paragraph{Example 4: Comparison Test}

\[
    \sum_{n=1}^{\infty} \frac{1}{\sqrt{n}}
\]

This is a $p$-series with $p = \frac{1}{2} < 1$, which diverges. Therefore,
any series comparable to it will also diverge.

\bigskip

\paragraph{Example 5: Rational Function}

\[
    \sum_{n=1}^{\infty} \frac{n+1}{n}
\]

Since
\[
    \frac{n+1}{n} = 1 + \frac{1}{n},
\]
and the terms do not approach zero, the series diverges by the $n$th-term test.

\bigskip

\subsubsection*{Summary of Divergence Results}

\begin{itemize}
    \item If $\lim_{n \to \infty} a_n \neq 0$, the series diverges.
    \item The harmonic series diverges even though its terms approach zero.
    \item Series comparable to divergent $p$-series with $p \le 1$ diverge.
\end{itemize}


\bigskip

\subsection*{1.3 Absolute Convergence}

A series $\sum a_n$ is \textbf{absolutely convergent} if
\[
    \sum |a_n| \text{ converges.}
\]
Absolute convergence guarantees convergence of the original series.

\paragraph{Example 1}
\[
    \sum_{n=1}^{\infty} \frac{(-1)^n}{n^2}
\]
The absolute series
\[
    \sum_{n=1}^{\infty} \frac{1}{n^2}
\]
converges; hence the given series is absolutely convergent.

\paragraph{Example 2}
\[
    \sum_{n=1}^{\infty} (-1)^n \frac{1}{n}
\]
The absolute series diverges, but the original series converges conditionally.

\paragraph{Example 3}
\[
    \sum_{n=1}^{\infty} (-1)^n \frac{n}{n^2+1}
\]
The absolute series behaves like $\sum \frac{1}{n}$ and diverges. The original series converges conditionally.


\bigskip

\subsection*{1.4 Taylor Series}

Taylor series provide a way to represent a function as an infinite power series
centered at a point $x = x_0$. This representation allows complicated functions
to be approximated by polynomials near the center.

\bigskip

\subsubsection*{Definition of the Taylor Series}

If a function $f(x)$ has derivatives of all orders at $x = x_0$, then the
\textbf{Taylor series of $f(x)$ about $x = x_0$} is

\[
    f(x) = \sum_{n=0}^{\infty} \frac{f^{(n)}(x_0)}{n!}(x-x_0)^n.
\]

This is a power series centered at $x_0$, with coefficients determined by the
derivatives of $f(x)$ at $x = x_0$.

\bigskip

\subsubsection*{Interpretation of the Taylor Series}

Each term of the Taylor series incorporates information about the function at
the point $x = x_0$:
\begin{itemize}
    \item The constant term gives the value $f(x_0)$.
    \item The linear term depends on $f'(x_0)$ and matches the slope at $x=x_0$.
    \item Higher-degree terms improve the accuracy of the approximation near $x_0$.
\end{itemize}

Thus, Taylor polynomials provide increasingly accurate local approximations of
$f(x)$ as more terms are included.

\bigskip

\subsubsection*{Taylor Polynomials}

The \textbf{$n$th Taylor polynomial} for $f(x)$ about $x = x_0$ is defined as

\[
    P_n(x) = \sum_{k=0}^{n} \frac{f^{(k)}(x_0)}{k!}(x-x_0)^k.
\]

As $n$ increases, $P_n(x)$ better approximates $f(x)$ near $x=x_0$, provided the
Taylor series converges.

\bigskip

\subsubsection*{Example 1: Taylor Series of $e^x$ About $x_0 = 1$}

Since all derivatives of $e^x$ are equal to $e^x$, we have
\[
    f^{(n)}(1) = e.
\]

Substituting into the Taylor series formula gives
\[
    e^x = e \sum_{n=0}^{\infty} \frac{(x-1)^n}{n!}.
\]

Writing out the first few terms,
\[
    e^x = e \left[ 1 + (x-1) + \frac{(x-1)^2}{2!} + \frac{(x-1)^3}{3!} + \cdots \right].
\]

This series converges for all real values of $x$.

\bigskip

\subsubsection*{Example 2: Taylor Series of $\ln x$ About $x_0 = 1$}

Let $f(x) = \ln x$. Its derivatives are
\[
    f'(x) = \frac{1}{x}, \quad
    f''(x) = -\frac{1}{x^2}, \quad
    f'''(x) = \frac{2}{x^3}, \quad \dots
\]

Evaluating at $x = 1$,
\[
    f^{(n)}(1) = (-1)^{n+1}(n-1)!.
\]

Substituting into the Taylor series formula yields
\[
    \ln x = (x-1) - \frac{(x-1)^2}{2} + \frac{(x-1)^3}{3}
    - \frac{(x-1)^4}{4} + \cdots
\]

This series converges for
\[
    0 < x \le 2,
\]
which corresponds to $|x-1| \le 1$ with endpoint testing.

\bigskip

\subsubsection*{Example 3: Polynomial Approximation of $\sqrt{x}$ Near $x = 4$}

Let
\[
    f(x) = \sqrt{x}.
\]

Compute derivatives:
\[
    f(4) = 2, \quad
    f'(4) = \frac{1}{4}, \quad
    f''(4) = -\frac{1}{32}.
\]

Using these values, the second-degree Taylor polynomial about $x = 4$ is
\[
    \sqrt{x} \approx 2 + \frac{1}{4}(x-4) - \frac{1}{64}(x-4)^2.
\]

This polynomial provides a good approximation to $\sqrt{x}$ for values of $x$
close to $4$.

\bigskip

\subsubsection*{Example 4: Taylor Series of $\displaystyle \frac{1}{1+2x}$ About $x_0 = 1$}

Consider the function
\[
    f(x) = \frac{1}{1+2x}.
\]

We want the Taylor series centered at $x = 1$. First, rewrite the function
in terms of $(x-1)$.

\[
    1 + 2x = 1 + 2(1 + (x-1)) = 3 + 2(x-1).
\]

Thus,
\[
    f(x) = \frac{1}{3 + 2(x-1)}.
\]

Factor out the constant term:
\[
    f(x) = \frac{1}{3} \cdot \frac{1}{1 + \frac{2}{3}(x-1)}.
\]

Now use the geometric series formula
\[
    \frac{1}{1+u} = \sum_{n=0}^{\infty} (-1)^n u^n
    \quad \text{for } |u| < 1.
\]

Let
\[
    u = \frac{2}{3}(x-1).
\]

Then the Taylor series becomes
\[
    f(x) = \frac{1}{3}
    \sum_{n=0}^{\infty} (-1)^n \left(\frac{2}{3}(x-1)\right)^n.
\]

Writing out the first few terms,
\[
    \frac{1}{1+2x}
    = \frac{1}{3}
    - \frac{2}{9}(x-1)
    + \frac{4}{27}(x-1)^2
    - \frac{8}{81}(x-1)^3
    + \cdots
\]

\bigskip

\textbf{Interval of Convergence:}

The geometric series converges when
\[
    \left| \frac{2}{3}(x-1) \right| < 1.
\]

Solving,
\[
    |x-1| < \frac{3}{2}.
\]

Thus, the radius of convergence is
\[
    \boxed{R = \frac{3}{2}},
\]
and the interval of convergence is
\[
    \boxed{\left(-\frac{1}{2},\, \frac{5}{2}\right)}.
\]

\subsubsection*{Convergence and Validity of Taylor Series}

A Taylor series represents the original function only on the interval where the
series converges.

\begin{itemize}
    \item Inside the interval of convergence, the Taylor series converges to $f(x)$.
    \item Outside this interval, the series may diverge or converge to a different value.
\end{itemize}

Therefore, convergence is essential in determining where a Taylor series can be
used as a valid representation or approximation of a function.

\bigskip

\subsubsection*{Special Case: Maclaurin Series}

When $x_0 = 0$, the Taylor series is called a \textbf{Maclaurin series}:

\bigskip

\subsection*{1.5 Maclaurin Series}

A \textbf{Maclaurin series} is a special case of the Taylor series obtained by
choosing the center $x_0 = 0$. If a function $f(x)$ has derivatives of all orders
at $x = 0$, then its Maclaurin series is

\[
    f(x) = \sum_{n=0}^{\infty} \frac{f^{(n)}(0)}{n!}x^n.
\]

Maclaurin series are especially useful because of their simple form and their
connection to fundamental power series.

\bigskip

\subsubsection*{Interpretation of the Maclaurin Series}

The Maclaurin series represents a function as an infinite polynomial whose
coefficients are determined by the function and its derivatives at $x = 0$.

\begin{itemize}
    \item The constant term gives the value $f(0)$.
    \item The linear term matches the slope at $x = 0$.
    \item Higher-degree terms improve the accuracy near $x = 0$.
\end{itemize}

Thus, Maclaurin polynomials provide increasingly accurate approximations of
$f(x)$ for values of $x$ close to zero.

\bigskip

\subsubsection*{Common Maclaurin Series}

\begin{examplebox}
    \paragraph{Example 1: $e^x$}

    Since all derivatives of $e^x$ are equal to $e^x$, and $e^0 = 1$, we obtain
    \[
        e^x = 1 + x + \frac{x^2}{2!} + \frac{x^3}{3!} + \cdots
    \]

    This series converges for all real values of $x$.
\end{examplebox}

\bigskip

\begin{examplebox}
    \paragraph{Example 2: $\sin x$}

    The derivatives of $\sin x$ alternate between $\sin x$ and $\cos x$, yielding
    \[
        \sin x = x - \frac{x^3}{3!} + \frac{x^5}{5!} - \cdots
    \]

    This series also converges for all real values of $x$.
\end{examplebox}

\bigskip

\begin{examplebox}
    \paragraph{Example 3: $\cos x$}

    The derivatives of $\cos x$ alternate between $\cos x$ and $-\sin x$, giving
    \[
        \cos x = 1 - \frac{x^2}{2!} + \frac{x^4}{4!} - \cdots
    \]

    Again, the radius of convergence is infinite.
\end{examplebox}

\bigskip

\subsubsection*{Convergence of Maclaurin Series}

Since Maclaurin series are power series centered at $0$, their convergence is
determined using the same tests discussed earlier.

\begin{itemize}
    \item Many common Maclaurin series converge for all real $x$.
    \item Some converge only within a finite interval and must be tested at
          endpoints.
\end{itemize}

The interval of convergence determines where the series equals the original
function.

\bigskip

\subsubsection*{Polynomial Approximation Using Maclaurin Series}

Maclaurin series can be used to approximate functions near $x = 0$ by truncating
the series after a finite number of terms.

\paragraph{Example 4: Approximation of $\sin x$}

Using the first two nonzero terms,
\[
    \sin x \approx x - \frac{x^3}{3!}.
\]

This approximation is accurate for small values of $x$.

\bigskip



\end{document}
