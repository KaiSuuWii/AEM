\subsection*{1.2 Divergence of Series}

A series $\sum a_n$ is said to \textbf{diverge} if the sequence of partial sums
\[
    S_n = a_1 + a_2 + \cdots + a_n
\]
does not approach a finite limit as $n \to \infty$.

\bigskip

\paragraph{Important Observation}
A necessary condition for convergence is
\[
    \lim_{n \to \infty} a_n = 0.
\]
If this limit does not exist or is not zero, the series \textbf{must diverge}.
However, this condition alone does not guarantee convergence.

\bigskip

\paragraph{Key Divergence Tests}
\begin{itemize}
    \item $n$th-Term Test for Divergence
    \item Harmonic Series
    \item Comparison with Known Divergent Series
\end{itemize}

\bigskip

\subsubsection*{1. $n$th-Term Test for Divergence}

If
\[
    \lim_{n \to \infty} a_n \neq 0 \quad \text{or does not exist},
\]
then the series
\[
    \sum_{n=1}^{\infty} a_n
\]
\textbf{diverges}.

\textbf{Important Note:}
If $\lim_{n \to \infty} a_n = 0$, the test is inconclusive.

\bigskip

\paragraph{Example 1: $n$th-Term Test}

\[
    \sum_{n=1}^{\infty} \frac{n}{n+1}
\]

Since
\[
    \lim_{n \to \infty} \frac{n}{n+1} = 1 \neq 0,
\]
the series diverges by the $n$th-term test.

\bigskip

\paragraph{Example 2: Oscillating Terms}

\[
    \sum_{n=1}^{\infty} (-1)^n
\]

Here, the sequence $a_n = (-1)^n$ does not approach a limit. Therefore, the
series diverges by the $n$th-term test.

\bigskip

\subsubsection*{2. Harmonic Series}

The \textbf{harmonic series} is defined as
\[
    \sum_{n=1}^{\infty} \frac{1}{n}.
\]

Although
\[
    \lim_{n \to \infty} \frac{1}{n} = 0,
\]
the harmonic series \textbf{diverges}.

This example shows that the condition $\lim_{n \to \infty} a_n = 0$ is necessary
but not sufficient for convergence.

\bigskip

\paragraph{Example 3: Constant Multiple of the Harmonic Series}

\[
    \sum_{n=1}^{\infty} \frac{5}{n}
\]

Since this is a constant multiple of the harmonic series, it also diverges.

\bigskip

\subsubsection*{3. Comparison with Divergent Series}

If $0 \le a_n \le b_n$ for all sufficiently large $n$ and
\[
    \sum b_n \quad \text{diverges},
\]
then
\[
    \sum a_n \quad \text{also diverges}.
\]

\bigskip

\paragraph{Example 4: Comparison Test}

\[
    \sum_{n=1}^{\infty} \frac{1}{\sqrt{n}}
\]

This is a $p$-series with $p = \frac{1}{2} < 1$, which diverges. Therefore,
any series comparable to it will also diverge.

\bigskip

\paragraph{Example 5: Rational Function}

\[
    \sum_{n=1}^{\infty} \frac{n+1}{n}
\]

Since
\[
    \frac{n+1}{n} = 1 + \frac{1}{n},
\]
and the terms do not approach zero, the series diverges by the $n$th-term test.

\bigskip

\subsubsection*{Summary of Divergence Results}

\begin{itemize}
    \item If $\lim_{n \to \infty} a_n \neq 0$, the series diverges.
    \item The harmonic series diverges even though its terms approach zero.
    \item Series comparable to divergent $p$-series with $p \le 1$ diverge.
\end{itemize}
