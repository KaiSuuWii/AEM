\subsection*{1.4 Taylor Series}

Taylor series provide a way to represent a function as an infinite power series
centered at a point $x = x_0$. This representation allows complicated functions
to be approximated by polynomials near the center.

\bigskip

\subsubsection*{Definition of the Taylor Series}

If a function $f(x)$ has derivatives of all orders at $x = x_0$, then the
\textbf{Taylor series of $f(x)$ about $x = x_0$} is

\[
    f(x) = \sum_{n=0}^{\infty} \frac{f^{(n)}(x_0)}{n!}(x-x_0)^n.
\]

This is a power series centered at $x_0$, with coefficients determined by the
derivatives of $f(x)$ at $x = x_0$.

\bigskip

\subsubsection*{Interpretation of the Taylor Series}

Each term of the Taylor series incorporates information about the function at
the point $x = x_0$:
\begin{itemize}
    \item The constant term gives the value $f(x_0)$.
    \item The linear term depends on $f'(x_0)$ and matches the slope at $x=x_0$.
    \item Higher-degree terms improve the accuracy of the approximation near $x_0$.
\end{itemize}

Thus, Taylor polynomials provide increasingly accurate local approximations of
$f(x)$ as more terms are included.

\bigskip

\subsubsection*{Taylor Polynomials}

The \textbf{$n$th Taylor polynomial} for $f(x)$ about $x = x_0$ is defined as

\[
    P_n(x) = \sum_{k=0}^{n} \frac{f^{(k)}(x_0)}{k!}(x-x_0)^k.
\]

As $n$ increases, $P_n(x)$ better approximates $f(x)$ near $x=x_0$, provided the
Taylor series converges.

\bigskip

\subsubsection*{Example 1: Taylor Series of $e^x$ About $x_0 = 1$}

Since all derivatives of $e^x$ are equal to $e^x$, we have
\[
    f^{(n)}(1) = e.
\]

Substituting into the Taylor series formula gives
\[
    e^x = e \sum_{n=0}^{\infty} \frac{(x-1)^n}{n!}.
\]

Writing out the first few terms,
\[
    e^x = e \left[ 1 + (x-1) + \frac{(x-1)^2}{2!} + \frac{(x-1)^3}{3!} + \cdots \right].
\]

This series converges for all real values of $x$.

\bigskip

\subsubsection*{Example 2: Taylor Series of $\ln x$ About $x_0 = 1$}

Let $f(x) = \ln x$. Its derivatives are
\[
    f'(x) = \frac{1}{x}, \quad
    f''(x) = -\frac{1}{x^2}, \quad
    f'''(x) = \frac{2}{x^3}, \quad \dots
\]

Evaluating at $x = 1$,
\[
    f^{(n)}(1) = (-1)^{n+1}(n-1)!.
\]

Substituting into the Taylor series formula yields
\[
    \ln x = (x-1) - \frac{(x-1)^2}{2} + \frac{(x-1)^3}{3}
    - \frac{(x-1)^4}{4} + \cdots
\]

This series converges for
\[
    0 < x \le 2,
\]
which corresponds to $|x-1| \le 1$ with endpoint testing.

\bigskip

\subsubsection*{Example 3: Polynomial Approximation of $\sqrt{x}$ Near $x = 4$}

Let
\[
    f(x) = \sqrt{x}.
\]

Compute derivatives:
\[
    f(4) = 2, \quad
    f'(4) = \frac{1}{4}, \quad
    f''(4) = -\frac{1}{32}.
\]

Using these values, the second-degree Taylor polynomial about $x = 4$ is
\[
    \sqrt{x} \approx 2 + \frac{1}{4}(x-4) - \frac{1}{64}(x-4)^2.
\]

This polynomial provides a good approximation to $\sqrt{x}$ for values of $x$
close to $4$.

\bigskip

\subsubsection*{Example 4: Taylor Series of $\displaystyle \frac{1}{1+2x}$ About $x_0 = 1$}

Consider the function
\[
    f(x) = \frac{1}{1+2x}.
\]

We want the Taylor series centered at $x = 1$. First, rewrite the function
in terms of $(x-1)$.

\[
    1 + 2x = 1 + 2(1 + (x-1)) = 3 + 2(x-1).
\]

Thus,
\[
    f(x) = \frac{1}{3 + 2(x-1)}.
\]

Factor out the constant term:
\[
    f(x) = \frac{1}{3} \cdot \frac{1}{1 + \frac{2}{3}(x-1)}.
\]

Now use the geometric series formula
\[
    \frac{1}{1+u} = \sum_{n=0}^{\infty} (-1)^n u^n
    \quad \text{for } |u| < 1.
\]

Let
\[
    u = \frac{2}{3}(x-1).
\]

Then the Taylor series becomes
\[
    f(x) = \frac{1}{3}
    \sum_{n=0}^{\infty} (-1)^n \left(\frac{2}{3}(x-1)\right)^n.
\]

Writing out the first few terms,
\[
    \frac{1}{1+2x}
    = \frac{1}{3}
    - \frac{2}{9}(x-1)
    + \frac{4}{27}(x-1)^2
    - \frac{8}{81}(x-1)^3
    + \cdots
\]

\bigskip

\textbf{Interval of Convergence:}

The geometric series converges when
\[
    \left| \frac{2}{3}(x-1) \right| < 1.
\]

Solving,
\[
    |x-1| < \frac{3}{2}.
\]

Thus, the radius of convergence is
\[
    \boxed{R = \frac{3}{2}},
\]
and the interval of convergence is
\[
    \boxed{\left(-\frac{1}{2},\, \frac{5}{2}\right)}.
\]

\subsubsection*{Convergence and Validity of Taylor Series}

A Taylor series represents the original function only on the interval where the
series converges.

\begin{itemize}
    \item Inside the interval of convergence, the Taylor series converges to $f(x)$.
    \item Outside this interval, the series may diverge or converge to a different value.
\end{itemize}

Therefore, convergence is essential in determining where a Taylor series can be
used as a valid representation or approximation of a function.

\bigskip

\subsubsection*{Special Case: Maclaurin Series}

When $x_0 = 0$, the Taylor series is called a \textbf{Maclaurin series}: