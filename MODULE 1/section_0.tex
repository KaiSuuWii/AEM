\subsection*{Series}

A \textbf{series} is the sum of the terms of a sequence. If $\{a_n\}$ is a sequence, then the corresponding series is written as
\[
    \sum_{n=1}^{\infty} a_n = a_1 + a_2 + a_3 + \cdots
\]

The value of a series is determined by the behavior of its \textbf{partial sums}
\[
    S_N = \sum_{n=1}^{N} a_n.
\]

\begin{itemize}
    \item The series \textbf{converges} if $\displaystyle \lim_{N \to \infty} S_N$ exists and is finite.
    \item The series \textbf{diverges} if this limit does not exist or is infinite.
\end{itemize}

\textbf{Necessary Condition for Convergence:}
\[
    \lim_{n \to \infty} a_n = 0.
\]
If this condition is not satisfied, the series must diverge.

\paragraph{Example 1}
Determine if the series
\[
    \sum_{n=1}^{\infty} n
\]
converges or diverges.

\textbf{Solution:} The $n$th partial sum is
\[
    s_n = \sum_{i=1}^n i = \frac{n(n+1)}{2}.
\]
As $n \to \infty$,
\[
    \lim_{n\to\infty} \frac{n(n+1)}{2} = \infty,
\]
so the sequence of partial sums diverges and thus the series diverges.

\paragraph{Example 2}
Determine if the series
\[
    \sum_{n=2}^{\infty} \frac{1}{n^2 - 1}
\]
converges or diverges, and if it converges, find its value.

\textbf{Solution:} A known formula for the partial sums of this series is
\[
    s_n = \sum_{i=2}^n \frac{1}{i^2 - 1}
    = \frac{3}{4} - \frac{1}{2n} - \frac{1}{2(n+1)}.
\]
Taking the limit as $n\to\infty$,
\[
    \lim_{n\to\infty} s_n = \frac{3}{4}.
\]
Thus the series converges and
\[
    \sum_{n=2}^{\infty} \frac{1}{n^2 - 1} = \frac{3}{4}.
\]

\paragraph{Example 3}
Determine if the series
\[
    \sum_{n=0}^{\infty} (-1)^n
\]
converges or diverges.

\textbf{Solution:} The partial sums oscillate:
\[
    s_0 = 1,\quad
    s_1 = 0,\quad
    s_2 = 1,\quad
    s_3 = 0,\quad \ldots
\]
Since the sequence of partial sums does not approach a single value, this series diverges.

\paragraph{Example 4}
Determine if the series
\[
    \sum_{n=1}^{\infty} \frac{1}{3^{\,n-1}}
\]
converges or diverges. If it converges, find the value.

\textbf{Solution:} A formula for the partial sums is
\[
    s_n = \sum_{i=1}^{n} \frac{1}{3^{\,i-1}} = \frac{3}{2}\left(1-\frac{1}{3^n}\right).
\]
Taking the limit,
\[
    \lim_{n\to\infty} s_n = \frac{3}{2}.
\]
Therefore, the series converges with
\[
    \sum_{n=1}^{\infty} \frac{1}{3^{\,n-1}} = \frac{3}{2}.
\]

\bigskip
\textbf{Observation:} In Examples 2 and 4, the terms of the series approach zero and the series converges; in Examples 1 and 3, they do not approach a limit that yields a convergent series. Thus, the limit of the series terms is an important preliminary check for convergence before applying tests.
