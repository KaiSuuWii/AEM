\subsection*{1.5 Maclaurin Series}

A \textbf{Maclaurin series} is a special case of the Taylor series obtained by
choosing the center $x_0 = 0$. If a function $f(x)$ has derivatives of all orders
at $x = 0$, then its Maclaurin series is

\[
    f(x) = \sum_{n=0}^{\infty} \frac{f^{(n)}(0)}{n!}x^n.
\]

Maclaurin series are especially useful because of their simple form and their
connection to fundamental power series.

\bigskip

\subsubsection*{Interpretation of the Maclaurin Series}

The Maclaurin series represents a function as an infinite polynomial whose
coefficients are determined by the function and its derivatives at $x = 0$.

\begin{itemize}
    \item The constant term gives the value $f(0)$.
    \item The linear term matches the slope at $x = 0$.
    \item Higher-degree terms improve the accuracy near $x = 0$.
\end{itemize}

Thus, Maclaurin polynomials provide increasingly accurate approximations of
$f(x)$ for values of $x$ close to zero.

\bigskip

\subsubsection*{Common Maclaurin Series}

\begin{examplebox}
    \paragraph{Example 1: $e^x$}

    Since all derivatives of $e^x$ are equal to $e^x$, and $e^0 = 1$, we obtain
    \[
        e^x = 1 + x + \frac{x^2}{2!} + \frac{x^3}{3!} + \cdots
    \]

    This series converges for all real values of $x$.
\end{examplebox}

\bigskip

\begin{examplebox}
    \paragraph{Example 2: $\sin x$}

    The derivatives of $\sin x$ alternate between $\sin x$ and $\cos x$, yielding
    \[
        \sin x = x - \frac{x^3}{3!} + \frac{x^5}{5!} - \cdots
    \]

    This series also converges for all real values of $x$.
\end{examplebox}

\bigskip

\begin{examplebox}
    \paragraph{Example 3: $\cos x$}

    The derivatives of $\cos x$ alternate between $\cos x$ and $-\sin x$, giving
    \[
        \cos x = 1 - \frac{x^2}{2!} + \frac{x^4}{4!} - \cdots
    \]

    Again, the radius of convergence is infinite.
\end{examplebox}

\bigskip

\subsubsection*{Convergence of Maclaurin Series}

Since Maclaurin series are power series centered at $0$, their convergence is
determined using the same tests discussed earlier.

\begin{itemize}
    \item Many common Maclaurin series converge for all real $x$.
    \item Some converge only within a finite interval and must be tested at
          endpoints.
\end{itemize}

The interval of convergence determines where the series equals the original
function.

\bigskip

\subsubsection*{Polynomial Approximation Using Maclaurin Series}

Maclaurin series can be used to approximate functions near $x = 0$ by truncating
the series after a finite number of terms.

\paragraph{Example 4: Approximation of $\sin x$}

Using the first two nonzero terms,
\[
    \sin x \approx x - \frac{x^3}{3!}.
\]

This approximation is accurate for small values of $x$.

\bigskip

