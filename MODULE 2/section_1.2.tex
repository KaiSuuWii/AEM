\subsection{Why Ordinary Points Matter}

Classifying a point as ordinary is not merely a formal step—it determines whether standard solution techniques apply.

\begin{itemize}
    \item At an ordinary point, all coefficient functions behave well locally.
    \item The differential equation can be reduced to a standard form.
    \item Solutions exist, are unique, and vary smoothly with initial conditions.
\end{itemize}

\begin{conceptbox}
    At an ordinary point, local solutions behave ``nicely'' and do not exhibit blow-up, discontinuities, or undefined behavior.
\end{conceptbox}

This favorable behavior is what ultimately permits solutions to be constructed systematically using series expansions, which will be developed in the next section.

\begin{noteBox}
    If a point fails the analyticity test, standard solution assumptions break down and more advanced techniques are required.
\end{noteBox}