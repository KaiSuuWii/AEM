
\section{POWER SERIES}
\subsection{Why Power Series Solutions Work}

We know that:
\begin{itemize}
    \item Analytic functions can be represented as power series.
    \item Power series can be differentiated and integrated term-by-term within their radius of convergence.
\end{itemize}

Hence, near an ordinary point $x_0$, we may \emph{assume} a solution of the form
\begin{equation}
    y(x) = \sum_{n=0}^{\infty} a_n (x - x_0)^n.
\end{equation}

This assumption is not a guess—it is justified by the theory of power series.

\subsection{Derivatives of the Assumed Solution}

Differentiating term-by-term,
\begin{align}
    y'(x)  & = \sum_{n=1}^{\infty} n a_n (x - x_0)^{n-1},      \\
    y''(x) & = \sum_{n=2}^{\infty} n(n-1) a_n (x - x_0)^{n-2}.
\end{align}

These expressions are substituted directly into the differential equation.

\subsection{Method of Solution}

The power series method about an ordinary point follows a systematic process:

\begin{enumerate}[label=\arabic*.]
    \item Assume a power series solution centered at $x_0$.
    \item Compute $y'$ and $y''$.
    \item Substitute into the differential equation.
    \item Rewrite all terms using the same power of $(x - x_0)$.
    \item Equate coefficients of like powers.
    \item Obtain a \textbf{recurrence relation} for $a_n$.
    \item Use initial conditions (if given) to find constants.
\end{enumerate}

\subsection{Example 1: Simple Ordinary Point}

\begin{examplebox}
    Solve the differential equation
    \[
        y'' - y = 0
    \]
    about the ordinary point $x_0 = 0$ using a power series.
\end{examplebox}

\textbf{Step 1: Assume a power series solution}
\[
    y = \sum_{n=0}^{\infty} a_n x^n.
\]

\textbf{Step 2: Compute derivatives}
\begin{align*}
    y'  & = \sum_{n=1}^{\infty} n a_n x^{n-1},      \\
    y'' & = \sum_{n=2}^{\infty} n(n-1) a_n x^{n-2}.
\end{align*}

\textbf{Step 3: Substitute into the equation}
\[
    \sum_{n=2}^{\infty} n(n-1) a_n x^{n-2}
    - \sum_{n=0}^{\infty} a_n x^n = 0.
\]

\textbf{Step 4: Align powers of $x$}

Re-index the first sum by letting $n \to n+2$:
\[
    \sum_{n=0}^{\infty} (n+2)(n+1)a_{n+2} x^n
    - \sum_{n=0}^{\infty} a_n x^n = 0.
\]

\textbf{Step 5: Equate coefficients}
\[
    (n+2)(n+1)a_{n+2} - a_n = 0.
\]

\textbf{Recurrence Relation:}
\[
    a_{n+2} = \frac{a_n}{(n+2)(n+1)}.
\]

\subsection{Structure of the Solution}

The recurrence relation shows that:
\begin{itemize}
    \item Even-indexed coefficients depend only on $a_0$.
    \item Odd-indexed coefficients depend only on $a_1$.
\end{itemize}

Thus, the general solution is a linear combination of two power series:
\[
    y(x) = a_0 \left( 1 + \frac{x^2}{2!} + \frac{x^4}{4!} + \cdots \right)
    + a_1 \left( x + \frac{x^3}{3!} + \frac{x^5}{5!} + \cdots \right).
\]

\begin{noteBox}
    These series are recognized as the Maclaurin series for $\cosh x$ and $\sinh x$, respectively.
\end{noteBox}

\subsection{Example 2: Variable Coefficients with Ordinary Point}

\begin{examplebox}
    Solve the differential equation
    \[
        (1 + x)y'' - xy' - y = 0
    \]
    about $x_0 = 0$.
\end{examplebox}

Since
\[
    p(x) = 1 + x \quad \text{and} \quad p(0) = 1 \neq 0,
\]
the point $x_0 = 0$ is an \textbf{ordinary point}. Hence, a power series solution about $x = 0$ exists.

\subsubsection*{Power Series Assumption}

Assume a solution of the form
\[
    y(x) = \sum_{n=0}^{\infty} a_n x^n.
\]

Then,
\begin{align}
    y'(x)  & = \sum_{n=1}^{\infty} n a_n x^{n-1},     \\
    y''(x) & = \sum_{n=2}^{\infty} n(n-1)a_n x^{n-2}.
\end{align}

\subsubsection*{Substitution into the Differential Equation}

Substituting into
\[
    (1 + x)y'' - xy' - y = 0,
\]
we obtain
\begin{align*}
     & (1+x)\sum_{n=2}^{\infty} n(n-1)a_n x^{n-2}
    - x\sum_{n=1}^{\infty} n a_n x^{n-1}
    - \sum_{n=0}^{\infty} a_n x^n = 0.
\end{align*}

Distributing terms,
\begin{align*}
    \sum_{n=2}^{\infty} n(n-1)a_n x^{n-2}
    + \sum_{n=2}^{\infty} n(n-1)a_n x^{n-1}
    - \sum_{n=1}^{\infty} n a_n x^n
    - \sum_{n=0}^{\infty} a_n x^n = 0.
\end{align*}

\subsubsection*{Reindexing the Series}

Rewrite each sum in terms of $x^n$:
\begin{align*}
    \sum_{n=0}^{\infty} (n+2)(n+1)a_{n+2} x^n
    + \sum_{n=1}^{\infty} n(n-1)a_n x^n
    - \sum_{n=1}^{\infty} n a_n x^n
    - \sum_{n=0}^{\infty} a_n x^n = 0.
\end{align*}

\subsubsection*{Coefficient Comparison}

For $n = 0$:
\[
    2a_2 - a_0 = 0
    \quad \Rightarrow \quad
    a_2 = \frac{a_0}{2}.
\]

For $n \ge 1$:
\[
    (n+2)(n+1)a_{n+2}
    + \big[n(n-1) - n - 1\big]a_n = 0.
\]

Simplifying,
\[
    n(n-1) - n - 1 = n^2 - 2n - 1.
\]

Thus, the recurrence relation is
\[
    \boxed{
        a_{n+2} =
        \frac{2n + 1 - n^2}{(n+2)(n+1)}\,a_n,
        \qquad n \ge 0.
    }
\]

\subsubsection*{Structure of the Solution}

\begin{itemize}
    \item Even-indexed coefficients depend only on $a_0$.
    \item Odd-indexed coefficients depend only on $a_1$.
\end{itemize}

Hence, the general solution about $x = 0$ is
\[
    y(x) = a_0 y_1(x) + a_1 y_2(x),
\]
where $y_1$ and $y_2$ are linearly independent power series solutions.

\begin{noteBox}
    Variable-coefficient equations generally lead to non-terminating recurrence relations. Even when closed-form solutions do not exist, power series solutions remain valid within their radius of convergence.
\end{noteBox}


\subsection{Radius of Convergence}

From Chapter 15:
\begin{itemize}
    \item Power series solutions converge within a radius determined by the nearest singularity.
    \item The solution is guaranteed to be valid at least up to the closest point where $P(x)$ or $Q(x)$ becomes non-analytic.
\end{itemize}

\begin{conceptbox}
    For an ordinary point, the power series solution always exists and converges in some neighborhood of the point.
\end{conceptbox}

\subsection*{Summary}

\begin{itemize}
    \item Ordinary points allow direct application of power series methods.
    \item The method converts differential equations into algebraic recurrence relations.
    \item Solutions naturally connect to Taylor and Maclaurin series from Chapter 15.
    \item This technique forms the foundation for solving more complex equations near singular points.
\end{itemize}
