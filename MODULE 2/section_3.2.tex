
\subsection{Legendre Functions}

Legendre functions arise naturally in problems with spherical symmetry, such as gravitational fields, electrostatic potentials, and wave propagation in spherical coordinates.

\subsubsection{Legendre's Differential Equation}

The Legendre equation is
\[
    (1-x^2)y'' - 2xy' + n(n+1)y = 0.
\]

\begin{conceptbox}
    The solutions of Legendre's equation are called \textbf{Legendre functions}. When $n$ is a nonnegative integer, the polynomial solutions are called \textbf{Legendre polynomials}, denoted by $P_n(x)$.
\end{conceptbox}

\subsubsection{Legendre Polynomials}

For integer $n$, the solution $P_n(x)$ is a polynomial of degree $n$. The first few Legendre polynomials are:
\[
    P_0(x)=1,
    \qquad
    P_1(x)=x,
    \qquad
    P_2(x)=\frac{1}{2}(3x^2-1),
    \qquad
    P_3(x)=\frac{1}{2}(5x^3-3x).
\]

\subsubsection{Orthogonality Property}

Legendre polynomials satisfy the orthogonality condition
\[
    \int_{-1}^{1} P_m(x)P_n(x)\,dx = 0
    \qquad \text{for } m\neq n.
\]

\begin{noteBox}
    Orthogonality is crucial in Fourier-type expansions and in solving boundary value problems using series methods.
\end{noteBox}

\subsubsection{Applications of Legendre Functions}

Legendre functions appear in:
\begin{itemize}
    \item Electrostatic potential problems (Laplace's equation in spherical coordinates),
    \item Gravitational field modeling,
    \item Spherical harmonics,
    \item Quantum mechanics (angular momentum).
\end{itemize}
