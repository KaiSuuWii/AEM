\subsection{Why Ordinary Power Series Fail}

At a singular point, dividing the equation by $p(x)$ introduces non-analytic terms into the coefficients. As a result:
\begin{itemize}
    \item Standard power series substitutions lead to undefined expressions, or
    \item Recurrence relations fail to determine coefficients uniquely.
\end{itemize}

\begin{conceptbox}
    The breakdown of the ordinary power series method is caused by the behavior of the coefficients, not by the solution itself.
\end{conceptbox}

This observation motivates the search for a more flexible form of series solution.

\begin{examplebox}
    \textbf{Proof via Example: Ordinary power series may fail at a singular point.}

    Consider the differential equation
    \[
        x^2 y'' + x y' - y = 0
    \]
    about the point $x_0 = 0$.


    \textbf{Step 1: Show that $x=0$ is singular.}

    Here $p(x)=x^2$, so $p(0)=0$. Hence $x=0$ is a singular point.

    \bigskip

    \textbf{Step 2: Attempt an ordinary power series solution.}

    Assume (ordinary power series)
    \[
        y = \sum_{n=0}^{\infty} a_n x^n,
        \qquad
        y'=\sum_{n=1}^{\infty} n a_n x^{n-1},
        \qquad
        y''=\sum_{n=2}^{\infty} n(n-1)a_n x^{n-2}.
    \]

    Substitute into the ODE:
    \[
        x^2\sum_{n=2}^{\infty} n(n-1)a_n x^{n-2}
        + x\sum_{n=1}^{\infty} n a_n x^{n-1}
        -\sum_{n=0}^{\infty} a_n x^n =0.
    \]

    Simplify each term:
    \[
        \sum_{n=2}^{\infty} n(n-1)a_n x^{n}
        +\sum_{n=1}^{\infty} n a_n x^{n}
        -\sum_{n=0}^{\infty} a_n x^{n}=0.
    \]

    Now combine into one series (noting the $n=0$ term explicitly):
    \[
        (-a_0) + \sum_{n=1}^{\infty}\big(n(n-1)+n-1\big)a_n x^n = 0.
    \]

    Since the power series is identically zero, every coefficient must be zero.

    \bigskip

    \textbf{Step 3: Coefficient comparison.}

    For $x^0$:
    \[
        -a_0=0 \quad\Rightarrow\quad a_0=0.
    \]

    For $n\ge 1$:
    \[
        \big(n(n-1)+n-1\big)a_n = 0
        \quad\Rightarrow\quad
        (n^2-1)a_n=0.
    \]

    Thus:
    \[
        (n^2-1)a_n = 0
        \Rightarrow
        \begin{cases}
            a_n=0,               & n\neq 1, \\
            a_1\ \text{is free}, & n=1.
        \end{cases}
    \]

    So the ordinary power series solution becomes
    \[
        y = a_1 x.
    \]

    \bigskip

    \textbf{Step 4: Why this shows the ordinary power series method fails.}

    A second-order linear differential equation should admit two linearly independent solutions.
    However, the ordinary power series assumption around $x=0$ produced only \emph{one} solution,
    \[
        y = a_1 x,
    \]
    and forced all other coefficients to be zero. \\

    The loss of the second independent solution is not because the differential equation has only one solution, but because the assumed form
    \[
        y=\sum_{n=0}^{\infty}a_n x^n
    \]
    is too restrictive at a singular point.
    \\

    \textbf{Conclusion:}
    At singular points, dividing by $p(x)$ introduces non-analytic coefficients, and ordinary power series may fail to capture the full solution space. This motivates the need for a more flexible series form (Frobenius method), which allows non-integer exponents.

\end{examplebox}