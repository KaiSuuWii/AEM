\section{Solutions About Ordinary Points}

We consider second-order linear differential equations of the form
\begin{equation}
    p(x)y'' + q(x)y' + r(x)y = 0,
    \label{eq:general}
\end{equation}
where $p(x)$, $q(x)$, and $r(x)$ are functions of $x$.

\subsection{Ordinary Points}

\begin{conceptbox}
    A point $x = x_0$ is called an \textbf{ordinary point} of \eqref{eq:general} if
    \[
        p(x_0) \neq 0.
    \]
\end{conceptbox}

At an ordinary point, the differential equation may be divided by $p(x)$, yielding
\begin{equation}
    y'' + \frac{q(x)}{p(x)}y' + \frac{r(x)}{p(x)}y = 0.
    \label{eq:standard}
\end{equation}

Since $p(x_0) \neq 0$, the functions
\[
    \frac{q(x)}{p(x)}, \qquad \frac{r(x)}{p(x)}
\]
or
\begin{equation}
    y'' + P(x)y' + Q(x)y = 0,
\end{equation}
where
\[
    P(x) = \frac{q(x)}{p(x)}, \qquad Q(x) = \frac{r(x)}{p(x)}.
\]

\begin{conceptbox}
    A point $x = x_0$ is an \textbf{ordinary point} if both $P(x)$ and $Q(x)$ are analytic at $x_0$.
\end{conceptbox}

\subsection{Power Series Assumption}

Because all coefficient functions are analytic near $x_0$, we assume a solution of the form
\begin{equation}
    y(x) = \sum_{n=0}^{\infty} a_n (x - x_0)^n,
    \label{eq:series}
\end{equation}
where the coefficients $a_n$ are constants to be determined.

\begin{noteBox}
    This is not a heuristic guess. Analyticity of the coefficients ensures that solutions may be represented locally by convergent power series.
\end{noteBox}

\subsection{Derivatives of the Series}

Differentiating \eqref{eq:series} term-by-term,
\begin{align}
    y'(x)  & = \sum_{n=1}^{\infty} n a_n (x - x_0)^{n-1},     \\
    y''(x) & = \sum_{n=2}^{\infty} n(n-1)a_n (x - x_0)^{n-2}.
\end{align}

These expressions are substituted into \eqref{eq:standard}.

\subsection{Method of Coefficient Comparison}

After substitution, all resulting series are rewritten so that they involve the same power $(x - x_0)^n$. Since power series are unique representations, the coefficient of each power must vanish.

This produces a \textbf{recurrence relation} of the form
\[
    a_{n+k} = F(n)\,a_n,
\]
which determines all higher coefficients from a finite number of initial values.

\subsection{Example: Ordinary Point at the Origin}

\begin{examplebox}
    Solve
    \[
        y'' - y = 0
    \]
    about $x_0 = 0$ using a power series.
\end{examplebox}

Assume
\[
    y = \sum_{n=0}^{\infty} a_n x^n.
\]

Substitution and reindexing yield the recurrence relation
\[
    a_{n+2} = \frac{a_n}{(n+2)(n+1)}.
\]

This produces two linearly independent solutions corresponding to $a_0$ and $a_1$.

\subsection{Structure of the General Solution}

The recurrence relation shows that:
\begin{itemize}
    \item Even-indexed coefficients depend only on $a_0$.
    \item Odd-indexed coefficients depend only on $a_1$.
\end{itemize}

Thus, the general solution near an ordinary point is
\[
    y(x) = a_0 y_1(x) + a_1 y_2(x),
\]
where $y_1$ and $y_2$ are power series solutions.

\begin{conceptbox}
    A second-order linear differential equation about an ordinary point always admits two linearly independent power series solutions.
\end{conceptbox}

This mirrors the existence–uniqueness theory for initial value problems.

\subsection{Radius of Convergence of the Solution}

The power series solution does not necessarily converge for all $x$.

\begin{conceptbox}
    The radius of convergence of a power series solution about $x_0$ extends at least up to the nearest point where $p(x)$, $q(x)$, or $r(x)$ fails to be analytic or where $p(x)=0$.
\end{conceptbox}

\begin{itemize}
    \item Inside this interval, the solution is guaranteed to converge.
    \item At or beyond this boundary, the solution may diverge or require a different expansion point.
\end{itemize}

\begin{noteBox}
    In engineering applications, this often corresponds to physical boundaries, material discontinuities, or coordinate singularities.
\end{noteBox}

\subsection*{Summary}

\begin{itemize}
    \item Ordinary points occur where $p(x_0) \neq 0$.
    \item At such points, the equation can be reduced to standard form.
    \item Power series solutions exist and are unique.
    \item The radius of convergence is limited by the nearest singular behavior of the coefficients.
\end{itemize}
