
\subsection{Why Power Series Solutions Work}

\subsubsection{Power Series Assumption}
We know that:
\begin{itemize}
    \item Analytic functions can be represented as power series.
    \item Power series can be differentiated and integrated term-by-term within their radius of convergence.
\end{itemize}

Hence, near an ordinary point $x_0$, we may \emph{assume} a solution of the form
\begin{equation}
    y(x) = \sum_{n=0}^{\infty} a_n (x - x_0)^n.
\end{equation}

This assumption is not a guess—it is justified by the theory of power series.

\subsubsection{Derivatives of the Assumed Solution}

Differentiating term-by-term,
\begin{align}
    y'(x)  & = \sum_{n=1}^{\infty} n a_n (x - x_0)^{n-1},      \\
    y''(x) & = \sum_{n=2}^{\infty} n(n-1) a_n (x - x_0)^{n-2}.
\end{align}

These expressions are substituted directly into the differential equation.

\subsubsection{Method of Solution}

The power series method about an ordinary point follows a systematic process:

\begin{enumerate}[label=\arabic*.]
    \item Assume a power series solution centered at $x_0$.
    \item Compute $y'$ and $y''$.
    \item Substitute into the differential equation.
    \item Rewrite all terms using the same power of $(x - x_0)$.
    \item Equate coefficients of like powers.
    \item Obtain a \textbf{recurrence relation} for $a_n$.
    \item Use initial conditions (if given) to find constants.
\end{enumerate}
\begin{examplebox}
    Solve the differential equation
    \[
        y'' - y = 0
    \]
    about the ordinary point $x_0 = 0$ using a power series.


    \textbf{Step 1: Assume a power series solution}
    \[
        y = \sum_{n=0}^{\infty} a_n x^n.
    \]

    \textbf{Step 2: Compute derivatives}
    \begin{align*}
        y'  & = \sum_{n=1}^{\infty} n a_n x^{n-1},      \\
        y'' & = \sum_{n=2}^{\infty} n(n-1) a_n x^{n-2}.
    \end{align*}

    \textbf{Step 3: Substitute into the equation}
    \[
        \sum_{n=2}^{\infty} n(n-1) a_n x^{n-2}
        - \sum_{n=0}^{\infty} a_n x^n = 0.
    \]

    \textbf{Step 4: Align powers of $x$}

    Re-index the first sum by letting $n \to n+2$:
    \[
        \sum_{n=0}^{\infty} (n+2)(n+1)a_{n+2} x^n
        - \sum_{n=0}^{\infty} a_n x^n = 0.
    \]

    \textbf{Step 5: Equate coefficients}
    \[
        (n+2)(n+1)a_{n+2} - a_n = 0.
    \]

    \textbf{Recurrence Relation:}
    \[
        a_{n+2} = \frac{a_n}{(n+2)(n+1)}.
    \]

    \textbf{Structure of the Solution}

    The recurrence relation shows that:
    \begin{itemize}
        \item Even-indexed coefficients depend only on $a_0$.
        \item Odd-indexed coefficients depend only on $a_1$.
    \end{itemize}

    Thus, the general solution is a linear combination of two power series:
    \[
        y(x) = a_0 \left( 1 + \frac{x^2}{2!} + \frac{x^4}{4!} + \cdots \right)
        + a_1 \left( x + \frac{x^3}{3!} + \frac{x^5}{5!} + \cdots \right).
    \]

\end{examplebox}

\begin{noteBox}
    These series are recognized as the Maclaurin series for $\cosh x$ and $\sinh x$, respectively.
\end{noteBox}

\textbf{Another Example:}

\begin{examplebox}
    Solve the differential equation
    \[
        (1 + x)y'' - xy' - y = 0
    \]
    about $x_0 = 0$.

    Since
    \[
        p(x) = 1 + x \quad \text{and} \quad p(0) = 1 \neq 0,
    \]
    the point $x_0 = 0$ is an \textbf{ordinary point}. Hence, a power series solution about $x = 0$ exists. \\

    \textbf{Step 1: Power Series Assumption}

    Assume a solution of the form
    \[
        y(x) = \sum_{n=0}^{\infty} a_n x^n.
    \]

    Then,
    \begin{align}
        y'(x)  & = \sum_{n=1}^{\infty} n a_n x^{n-1},     \\
        y''(x) & = \sum_{n=2}^{\infty} n(n-1)a_n x^{n-2}.
    \end{align}

    \textbf{Step 2: Substitution into the Differential Equation}

    Substituting into
    \[
        (1 + x)y'' - xy' - y = 0,
    \]
    we obtain
    \begin{align*}
         & (1+x)\sum_{n=2}^{\infty} n(n-1)a_n x^{n-2}
        - x\sum_{n=1}^{\infty} n a_n x^{n-1}
        - \sum_{n=0}^{\infty} a_n x^n = 0.
    \end{align*}

    Distributing terms,
    \begin{align*}
        \sum_{n=2}^{\infty} n(n-1)a_n x^{n-2}
        + \sum_{n=2}^{\infty} n(n-1)a_n x^{n-1}
        - \sum_{n=1}^{\infty} n a_n x^n
        - \sum_{n=0}^{\infty} a_n x^n = 0.
    \end{align*}

    \textbf{Step 3: Reindexing the Series}


    From Step 2 we had
    \begin{align*}
        \sum_{n=2}^{\infty} n(n-1)a_n x^{n-2}
        + \sum_{n=2}^{\infty} n(n-1)a_n x^{n-1}
        - \sum_{n=1}^{\infty} n a_n x^n
        - \sum_{n=0}^{\infty} a_n x^n = 0.
    \end{align*}

    We rewrite \emph{each} sum in powers of $x^n$.

    \begin{itemize}
        \item For the first sum, let $n \to n+2$:
              \[
                  \sum_{n=2}^{\infty} n(n-1)a_n x^{n-2}
                  = \sum_{n=0}^{\infty} (n+2)(n+1)a_{n+2} x^n.
              \]

        \item For the second sum, let $m=n-1$ (so $m\ge 1$), then rename $m\to n$:
              \[
                  \sum_{n=2}^{\infty} n(n-1)a_n x^{n-1}
                  = \sum_{n=1}^{\infty} (n+1)n\,a_{n+1} x^n.
              \]

        \item The remaining sums already involve $x^n$:
              \[
                  -\sum_{n=1}^{\infty} n a_n x^n,
                  \qquad
                  -\sum_{n=0}^{\infty} a_n x^n.
              \]
    \end{itemize}

    Thus the equation becomes
    \begin{align*}
        \sum_{n=0}^{\infty} (n+2)(n+1)a_{n+2} x^n
        + \sum_{n=1}^{\infty} n(n+1)a_{n+1} x^n
        - \sum_{n=1}^{\infty} n a_n x^n
        - \sum_{n=0}^{\infty} a_n x^n = 0.
    \end{align*}
    \textbf{Step 4: Coefficient Comparison}

    For $n = 0$:
    \[
        2a_2 - a_0 = 0
        \quad \Rightarrow \quad
        a_2 = \frac{a_0}{2}.
    \]

    For $n \ge 1$:
    \[
        (n+2)(n+1)a_{n+2}
        + \big[n(n-1) - n - 1\big]a_n = 0.
    \]

    Simplifying,
    \[
        n(n-1) - n - 1 = n^2 - 2n - 1.
    \]

    Thus, the recurrence relation is
    \[
        \boxed{
            a_{n+2} =
            \frac{2n + 1 - n^2}{(n+2)(n+1)}\,a_n,
            \qquad n \ge 0.
        }
    \]

    \textbf{Step 5: Meaning of $a_0$ and $a_1$}

    From the assumed series
    \[
        y(x)=\sum_{n=0}^{\infty}a_n x^n,
    \]
    we have
    \[
        \boxed{a_0=y(0)}.
    \]
    Also,
    \[
        y'(x)=\sum_{n=1}^{\infty}n a_n x^{n-1}
        \quad\Rightarrow\quad
        \boxed{a_1=y'(0)}.
    \]
    So $a_0$ and $a_1$ are the two arbitrary constants determined by initial conditions.

    \textbf{Step 6: Write the two linearly independent series solutions}

    Set $\{a_0=1,\ a_1=0\}$ to define $y_1(x)$:
    \[
        y_1(x)=1+\frac{1}{2}x^2-\frac{1}{6}x^3+\frac{5}{24}x^4-\frac{19}{120}x^5+\frac{101}{720}x^6+\cdots
    \]

    Set $\{a_0=0,\ a_1=1\}$ to define $y_2(x)$:
    \[
        y_2(x)=x+\frac{1}{3}x^3-\frac{1}{6}x^4+\frac{1}{6}x^5-\frac{5}{36}x^6+\frac{31}{252}x^7+\cdots
    \]

    \textbf{Step 5: Structure of the Solution}

    \begin{itemize}
        \item Even-indexed coefficients depend only on $a_0$.
        \item Odd-indexed coefficients depend only on $a_1$.
    \end{itemize}

    Hence, the general solution about $x = 0$ is
    \[
        y(x) = a_0 y_1(x) + a_1 y_2(x),
    \]
    where $y_1$ and $y_2$ are linearly independent power series solutions.
\end{examplebox}

\begin{noteBox}
    Variable-coefficient equations generally lead to non-terminating recurrence relations. Even when closed-form solutions do not exist, power series solutions remain valid within their radius of convergence.
\end{noteBox}


\begin{examplebox}
    Find the first four terms in each portion of the series solution about $x_0=-2$ for
    \[
        y''-xy=0.
    \]


    \textbf{Step 1: Shift the expansion point.}
    Let
    \[
        t=x+2 \quad\Longrightarrow\quad x=t-2,
    \]
    and write $y$ as a power series in $t$:
    \[
        y(t)=\sum_{n=0}^{\infty} a_n t^n.
    \]
    Since $\dv{}{x}=\dv{}{t}$, we have
    \[
        y''(t)=\sum_{n=0}^{\infty}(n+2)(n+1)a_{n+2}t^n.
    \]

    Substitute into $y''-xy=0$:
    \[
        y''-(t-2)y=0
        \quad\Longrightarrow\quad
        y''+2y-ty=0.
    \]

    \textbf{Step 2: Substitute the series and align powers.}
    \[
        \sum_{n=0}^{\infty}(n+2)(n+1)a_{n+2}t^n
        +2\sum_{n=0}^{\infty}a_nt^n
        -\sum_{n=0}^{\infty}a_nt^{n+1}=0.
    \]
    Rewrite the last sum as an $t^n$-series:
    \[
        \sum_{n=0}^{\infty}a_nt^{n+1}=\sum_{n=1}^{\infty}a_{n-1}t^n.
    \]
    So we get
    \[
        \sum_{n=0}^{\infty}\Big((n+2)(n+1)a_{n+2}+2a_n\Big)t^n
        -\sum_{n=1}^{\infty}a_{n-1}t^n=0.
    \]

    \textbf{Step 3: Coefficient equations (recurrence).}

    For $n=0$:
    \[
        2a_2+2a_0=0 \quad\Longrightarrow\quad a_2=-a_0.
    \]

    For $n\ge 1$:
    \[
        (n+2)(n+1)a_{n+2}+2a_n-a_{n-1}=0
        \quad\Longrightarrow\quad
        \boxed{
            a_{n+2}=\frac{a_{n-1}-2a_n}{(n+2)(n+1)},\ \ n\ge 1.
        }
    \]

    \textbf{Step 4: Build the two portions (two linearly independent series).}

    \medskip
    \noindent\textbf{Portion 1 (take $a_0=1,\ a_1=0$).}
    Using the recurrence:
    \[
        a_2=-1,\quad a_3=\frac{1}{6},\quad a_4=\frac{1}{6},\ \dots
    \]
    So the first four terms are
    \[
        \boxed{
            y_1(t)=1 - t^2 + \frac{1}{6}t^3 + \frac{1}{6}t^4+\cdots
        }
    \]

    \medskip
    \noindent\textbf{Portion 2 (take $a_0=0,\ a_1=1$).}
    Using the recurrence:
    \[
        a_2=0,\quad a_3=-\frac{1}{3},\quad a_4=\frac{1}{12},\quad a_5=\frac{1}{30},\ \dots
    \]
    So the first four terms are
    \[
        \boxed{
            y_2(t)=t - \frac{1}{3}t^3 + \frac{1}{12}t^4 + \frac{1}{30}t^5+\cdots
        }
    \]

    \textbf{Final series solution about $x_0=-2$.}
    Since $t=x+2$,
    \[
        \boxed{
            \begin{aligned}
                y(x) =\; & a_0\!\left\{
                1-(x+2)^2+\frac{1}{6}(x+2)^3+\frac{1}{6}(x+2)^4+\cdots
                \right\}                  \\
                         & + a_1\!\left\{
                (x+2)-\frac{1}{3}(x+2)^3+\frac{1}{12}(x+2)^4+\frac{1}{30}(x+2)^5+\cdots
                \right\}
            \end{aligned}
        }
    \]
\end{examplebox}

\subsubsection{Radius of Convergence}

We know that:
\begin{itemize}
    \item Power series solutions converge within a radius determined by the nearest singularity.
    \item The solution is guaranteed to be valid at least up to the closest point where $P(x)$ or $Q(x)$ becomes non-analytic.
\end{itemize}

\begin{conceptbox}
    For an ordinary point, the power series solution always exists and converges in some neighborhood of the point.
\end{conceptbox}
