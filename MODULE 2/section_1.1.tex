\section{Solutions About Ordinary Points}

In this section, we introduce the \textbf{power series method} for solving linear differential equations near an \emph{ordinary point}. This method is foundational in engineering mathematics and serves as a bridge between differential equations and special functions encountered in applied sciences.



\subsection{Definition}
\begin{conceptbox}
    An \textbf{ordinary point} of a differential equation is a point where all coefficient functions are analytic (i.e., can be expressed as power series).
\end{conceptbox}
\subsubsection{General Form of the Differential Equation}

We consider second-order linear differential equations of the form
\begin{equation}
    p(x)y'' + q(x)y' + r(x)y = 0,
    \label{eq:general}
\end{equation}
where $p(x)$, $q(x)$, and $r(x)$ are functions of $x$. At an ordinary point, the differential equation may be divided by $p(x)$, yielding
\begin{equation}
    y'' + \frac{q(x)}{p(x)}y' + \frac{r(x)}{p(x)}y = 0.
    \label{eq:standard}
\end{equation}

Since $p(x_0) \neq 0$, the functions
\[
    \frac{q(x)}{p(x)}, \qquad \frac{r(x)}{p(x)}
\]
or
\begin{equation}
    y'' + P(x)y' + Q(x)y = 0,
\end{equation}
where
\[
    P(x) = \frac{q(x)}{p(x)}, \qquad Q(x) = \frac{r(x)}{p(x)}.
\]

\begin{conceptbox}
    A point $x = x_0$ is an \textbf{ordinary point} if both $P(x)$ and $Q(x)$ are analytic at $x_0$.
\end{conceptbox}

\begin{noteBox}
    In practice, most equations with polynomial coefficients have ordinary points everywhere except where division by zero occurs.
\end{noteBox}

\subsubsection{Test for Analyticity}

A function $f(x)$ is said to be \textbf{analytic at a point $x_0$} if it can be represented by a convergent power series in some open interval containing $x_0$. That is,
\[
    f(x) = \sum_{n=0}^{\infty} a_n (x - x_0)^n
\]
for all $x$ sufficiently close to $x_0$.

\begin{conceptbox}
    A practical test for analyticity is that the function and all of its derivatives exist and are finite in a neighborhood of the point.
\end{conceptbox}

In the context of the differential equation
\[
    y'' + P(x)y' + Q(x)y = 0,
\]
the point $x = x_0$ is an ordinary point if:
\begin{itemize}
    \item $P(x)$ is analytic at $x_0$, and
    \item $Q(x)$ is analytic at $x_0$.
\end{itemize}

\begin{noteBox}
    If $P(x)$ and $Q(x)$ are rational functions, then analyticity fails only where their denominators vanish. These locations mark the boundary between ordinary and singular points.
\end{noteBox}

This test allows us to classify points \emph{before} attempting to construct a solution.

\begin{examplebox}
    Determine whether $x = 0$ is an ordinary point of the differential equation
    \[
        x^2 y'' + (x+1)y' - y = 0.
    \]


    \textbf{Step 1: Write the equation in standard form}

    Divide the equation by $x^2$:
    \[
        y'' + \frac{x+1}{x^2}y' - \frac{1}{x^2}y = 0.
    \]

    Thus,
    \[
        P(x) = \frac{x+1}{x^2}, \qquad Q(x) = -\frac{1}{x^2}.
    \]

    \textbf{Step 2: Test analyticity at $x = 0$}

    Both $P(x)$ and $Q(x)$ contain the term $\frac{1}{x^2}$, which is undefined at $x = 0$. Therefore, neither function is analytic at $x = 0$.

    \textbf{Conclusion:}
    \[
        x = 0 \text{ is \emph{not} an ordinary point.}
    \]

\end{examplebox}



\begin{noteBox}
    The failure of analyticity is caused by division by zero. This point must be treated as a singular point.
\end{noteBox}
